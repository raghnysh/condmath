%%% Notes on condensed mathematics

\documentclass{article}

\usepackage{notes}

\addbibresource{condmath.bib}

\title{Notes on condensed mathematics}

\hypersetup{pdftitle={Notes on condensed mathematics}}

\hypersetup{pdfsubject={Mathematics}}

\hypersetup{pdfkeywords={condensed mathematics, topological algebra,
    profinite spaces, category theory, abelian categories, derived
    categories, infinity categories, topos theory}}

\author{Raghavendra Nyshadham\thanks{\cczero} \\
  {\normalsize\nolinkurl{rn@raghnysh.com}}}

\hypersetup{pdfauthor={Raghavendra Nyshadham (rn@raghnysh.com)}}

\date{2024-01-14}

\begin{document}

\begin{titlingpage}
  \maketitle

  \begin{abstract}
    This document is a set of rough notes on condensed mathematics.
    The document is a work in progress.  The files of the current
    version of the document are available at
    \url{https://github.com/raghnysh/condmath}.

    \mscnumbers{20E18, 18E10, 22A99, 18F10, 18G80, 18N60}

    \ghtopics{algebraic-geometry, category-theory, topology,
      homological-algebra}
  \end{abstract}
\end{titlingpage}

\tableofcontents

\section{Introduction}
\label{sec:113nrd0o}

This document is a set of rough notes on condensed mathematics, a
subject that was developed in the references \textcite{bib:872u2noz},
\textcite{bib:iy49ytm3}, and \textcite{bib:7k2n4jtn}.  The document is
a work in progress.  The files of the current version of the document
are available at \url{https://github.com/raghnysh/condmath}.

\section{Set theory}
\label{sec:pafnta4o}

This section contains some preliminaries on set theory: ordinals,
transfinite induction, cardinals, etc.  The description of these
topics here is informal because nothing is gained for these notes by
making the development more rigorous.

\subsection{Ordered sets}
\label{sec:5f3q5o8v}

This subsection is a recapitulation of some basic notions about orders
on sets.  It is essentially a summary of \S1 of Chapter~III in
\textcite{bib:lmhdqwpw}.

The \firstterm{negation} of a relation \(R\) on a set \(X\) is the
relation \(R^\neg\) on \(X\) that is defined by setting, for all
\(x, y \in X\), \(x R^\neg y\) if \(x\) is not related to \(y\) with
respect to \(R\), that is, if the statement \(x R y\) is false.  The
\firstterm{opposite} of \(R\) is the relation \(R^{\mathrm{op}}\) on
\(X\) that is defined by setting \(x R^{\mathrm{op}} y\) if \(y R x\).
The \firstterm{irreflexive kernel} of \(R\) is the relation \(R^\neq\)
on \(X\) that is defined by setting \(x R^\neq y\) if \(x R y\) and
\(x \neq y\).  The \firstterm{reflexive closure} of \(R\) is the
relation \(R^=\) on \(X\) that is defined by setting \(x R^= y\) if
\(x R y\) or \(x = y\); it is the smallest reflexive relation on \(X\)
that contains \(R\).

It is obvious that negation and forming the opposite are involutive
while the irreflexive kernel and reflexive closure are idempotent,
that is,
\begin{displaymath}
  (R^\neg)^\neg = R, \quad
  (R^{\mathrm{op}})^{\mathrm{op}} = R, \quad
  (R^\neq)^\neq = R^\neq, \quad
  (R^=)^= = R^=,
\end{displaymath}
and that taking the opposite commutes with the other three operations:
\begin{displaymath}
  (R^{\mathrm{op}})^\neg = (R^\neg)^{\mathrm{op}}, \quad
  (R^{\mathrm{op}})^\neq = (R^\neq)^{\mathrm{op}}, \quad
  (R^{\mathrm{op}})^= = (R^=)^{\mathrm{op}}.
\end{displaymath}

An \firstterm{order} on \(X\) is a relation \(R\) on it which is:
\begin{enumerate}
\item \firstterm{reflexive}: \(x R x\) for every element \(x\) of
  \(X\);
\item \firstterm{anti-symmetric}: if \(x, y \in X\), and \(x R y\) and
\(y R x\), then \(x = y\); and
\item \firstterm{transitive}: if \(x, y, z \in X\), and \(x R y\) and
  \(y R z\), then \(x R z\).
\end{enumerate}
An \firstterm{ordered set} is a set together with an order on it.

Usually the symbol \(\leq\) (or a similar symbol such as \(\preceq\)
or \(\sqsubseteq\)) is used as notation for an order.  If \(\leq\) is
used to denote an order, the opposite of \(\leq\) is denoted by
\(\geq\).  The irreflexive kernel of \(\leq\) is denoted by \(<\), and
that of \(\geq\) by \(>\).  The negations of \(\leq, \geq, <, >\) are
denoted by \(\nleq, \ngeq, \nless, \ngtr\), respectively.  The
corresponding notations when another symbol is used instead of
\(\leq\) are analogous to these notations.

\begin{theorem}
  \label{thm:ztywp662}
  Taking the irreflexive kernel is a bijection from the set of orders
  on \(X\) onto the set of all relations \(<\) on \(X\) which are:
  \begin{enumerate}
  \item \firstterm{irreflexive}: there is no element \(x\) of \(X\)
    such that \(x < x\);
  \item \emph{transitive}: if \(x,y,z \in X\), and \(x < y\) and
    \(y < z\), then \(x < z\).
  \end{enumerate}
  The inverse of this bijection maps any relation with these two
  properties to its reflexive closure.
\end{theorem}

The irreflexive kernel \(<\) of an order \(\leq\) on \(X\) is
obviously irreflexive; if \(x,y,z \in X\), and \(x < y\) and
\(y < z\), the transitivity of \(\leq\) implies that \(x \leq z\), and
because \(x \neq y\) the anti-symmetry of \(\leq\) implies that
\(x \neq z\), so \(<\) is transitive.  Conversely, if \(<\) is an
irreflexive and transitive relation on \(X\), its reflexive closure
\(\leq\) is obviously reflexive; if \(x \leq y\) and \(y \leq z\),
either of the equalities \(x = y\) and \(y = z\) implies that
\(x \leq z\), so \(\leq\) is transitive; lastly, if \(x \leq y\) and
\(y \leq x\), the transitivity of \(\leq\) and the irreflexivity of
\(<\) imply that \(x = y\); thus, \(\leq\) is an order.  It is obvious
that the maps \(\leq \mapsto <\) and \(< \mapsto \leq\) are inverses
of each other.

The restriction of an order \(\leq\) on \(X\) to any subset \(Y\) of
\(X\) is an order on \(Y\), which is said to be \firstterm{induced} by
the order on \(X\).  A family \((X_i)_{i \in I}\) of sets and an order
\(\leq_i\) on each \(X_i\) together induce on the set
\(X = \prod_{i \in I}X_i\) an order \(\leq\) called the
\firstterm{product} order, which is defined by setting for any two
elements \(x= (x_i)\) and \(y = (y_i)\) of \(X\), \(x \leq y\) if
\(x_i \leq_i y_i\) for all \(i\).  The product order specialises to an
order on the set \(Y^X\) of all functions from a set \(X\) to a set
\(Y\), in which any two functions \(f,g\) satisfy the condition
\(f \leq g\) if and only if \(f(x) \leq g(x)\) for all \(x \in X\).

A function \(f\) from an ordered set \(X\) to an ordered set \(Y\) is
called \firstterm{increasing} (respectively, \firstterm{decreasing})
if \(f(x) \leq f(x')\) (respectively, \(f(x) \geq f(x')\)) for all
\(x, x' \in X\) such that \(x \leq x'\).  It is called
\firstterm{strictly increasing} (respectively, \firstterm{strictly
  decreasing}) if \(f(x) < f(x')\) (respectively, \(f(x) > f(x')\))
for all \(x, x' \in X\) such that \(x < x'\).  These definitions imply
that an injective increasing (respectively, decreasing) function is
strictly increasing (respectively, strictly decreasing).  An
\firstterm{isomorphism} from \(X\) to \(Y\) is an increasing bijection
from \(X\) onto \(Y\) whose inverse is also increasing.  An
isomorphism from \(X\) to itself is called an \firstterm{automorphism}
of \(X\).

An element \(a\) of an ordered set \(X\) is called \firstterm{minimal}
(respectively, \firstterm{maximal}) if it is the only element \(x\) of
\(X\) such that \(x \leq a\) (respectively, \(x \geq a\)).  It is
called a \firstterm{minimum} (respectively, \firstterm{maximum}) of
\(X\) if \(a \leq x\) (respectively, \(a \geq x\)) for all
\(x \in X\).  Anti-symmetry implies that \(X\) has at most one
minimum, and at most one maximum.  If the minimum (respectively,
maximum) of \(X\) exists, then it is the unique minimal element
(respectively, maximal element) of \(X\).

An element \(a\) of \(X\) is called a \firstterm{lower bound}
(respectively, \firstterm{upper bound}) of a subset \(Y\) of \(X\) if
\(a \leq y\) (respectively, \(a \geq y\)) for all \(y \in Y\).  A
subset of \(X\) is said to be \firstterm{bounded below} (respectively,
\firstterm{bounded above}) if it has a lower bound (respectively,
upper bound).  The maximum of the set of lower bounds of \(Y\) is
called the \firstterm{infimum} of \(Y\); the minimum of the set of
upper bounds of \(Y\) is called the \firstterm{supremum} of \(Y\).
The infimum and supremum of \(Y\) are unique when they exist, and are
respectively denoted by \(\inf Y\) and \(\sup Y\).

An order on a set \(X\) is called \firstterm{directed} (respectively,
\firstterm{codirected}) if every subset of \(X\) with two elements is
bounded above (respectively, bounded below).  A \firstterm{directed
  set} (respectively, \firstterm{codirected set}) is an ordered set
whose order is directed (respectively, codirected).  A maximal element
of a directed set is its maximum, and a minimal element of a
codirected set is its minimum.

A \firstterm{lattice} is an ordered set \(X\) with the property that
every subset of \(X\) with two elements has an infimum and a supremum.
The product of a family of lattices is again a lattice.

An order on a set \(X\) is said to be \firstterm{total} if for any two
elements \(x, y\) of \(X\), either \(x \leq y\) or \(y \leq x\).  An
order on \(X\) is total if and only if it has the property of
\firstterm{trichotomy}: for any two elements \(x, y\) of \(X\),
exactly one of the conditions \(x < y\), \(x = y\), \(x > y\) is true.
A \firstterm{totally ordered set} is an ordered set whose order is
total.  Every strictly increasing surjection from a totally ordered
set \(X\) onto any ordered set \(Y\) is an isomorphism; in particular,
any increasing bijection from \(X\) onto \(Y\) is an isomorphism.
Zorn's lemma, which is equivalent to the axiom of choice, states that
an ordered set in which every totally ordered subset is bounded above
has a maximal element.

For any two elements \(a, b\) of an ordered set \(X\), any of the sets
\begin{align*}
  [a, b]& = \{ x \in X \,\vert\, a \leq x \leq b \},&
  [a, b)& = \{ x \in X \,\vert\, a \leq x < b \}, \\
  (a, b]& = \{ x \in X \,\vert\, a < x \leq b \},&
  (a, b)& = \{ x \in X \,\vert\, a < x < b \},
\end{align*}
or the sets
\begin{align*}
    [a, \infty)& = \{ x \in X \,\vert\, a \leq x \},&
  (a, \infty)& = \{ x \in X \,\vert\, a < x \}, \\
  (-\infty, b]& = \{ x \in X \,\vert\, x \leq b \},&
  (-\infty, b)& = \{ x \in X \,\vert\, x < b \},
\end{align*}
or the set \(X\) itself, is called an \firstterm{interval}.  A subset
\(Y\) of \(X\) is called a \firstterm{segment} if for every
\(a \in Y\), the interval \((-\infty, a]\) is contained in \(Y\).  The
union and the intersection of a family of segments of \(X\) are
evidently segments, as are the empty set and \(X\).

\subsection{Well-ordered sets}
\label{sec:62tryw49}

An order on a set \(X\) is called a \firstterm{well-order} if every
nonempty subset of \(X\) has a minimum.  A well-order is total: if
\(x,y\) are two elements of \(X\), and if, for instance, \(x\) is the
minimum of the set \(\{ x, y \}\), then \(x \leq y\).  A well-order on
\(X\) induces a well-order on every subset of \(X\).  Zermelo's
theorem, which is equivalent to the axiom of choice, states that every
set has a well-order.  It is notationally convenient, especially when
the set \(X\) in question comes with an already given order, to
rephrase Zermelo's theorem as follows: for every set \(X\), there
exist a well-ordered set \(\Lambda\) and a bijection between
\(\Lambda\) and \(X\).

A \firstterm{well-ordered set} is an ordered set whose order is a
well-order.  Every subset of a well-ordered set that is bounded above
has a supremum.

\begin{theorem}
  \label{thm:3uwc4uyw}
  Taking the irreflexive kernel is a bijection from the set of
  well-orders on a set \(X\) onto the set of relations \(<\) on \(X\)
  which have the following properties:
  \begin{enumerate}
  \item For any two distinct elements \(x, y\) of \(X\), either
    \(x < y\) or \(y < x\).
  \item Every nonempty subset \(Y\) of \(X\) has an element \(a\) such
    that no element \(x\) of \(Y\) satisfies the condition \(x < a\).
  \end{enumerate}
  The inverse of this bijection maps any relation with these
  properties to its reflexive closure.
\end{theorem}

Let \(\leq\) be a well-order on \(X\), and let \(<\) be its
irreflexive kernel.  As \(\leq\) is a total order, it has the
trichotomy property: for any two elements \(x, y\) of \(X\), exactly
one of the conditions \(x < y\), \(x = y\), \(y < x\) is true.  In
particular, for any two distinct elements \(x, y\) of \(X\), either
\(x < y\) or \(y < x\).  If \(a\) is the minimum of a nonempty subset
\(Y\) of \(X\), then for all \(x \in Y\), we have \(a \leq x\); the
trichotomy of \(\leq\) implies that \(x \nless a\).

Conversely, let \(<\) be a relation on \(X\) with the properties
stated in the exercise.  We will first check that \(<\) is irreflexive
and transitive.

If \(x\) is any element of \(X\), the singleton \(Y = \{ x \}\) has an
element \(a\) such that \(y \nless a\) for all \(y \in Y\).
Necessarily, \(a = x\), so \(x \nless x\).  Therefore, \(<\) is
irreflexive.

We will now verify that for any \(x, y \in X\), exactly one of the
conditions \(x < y\), \(x = y\), \(y < x\) holds.  The first
hypothesis on \(<\) implies that at least one of these conditions is
true.  If \(x = y\), then the irreflexivity of \(<\) implies that
\(x \nless y\) and \(y \nless x\).  If \(x < y\), we have to rule out
\(x = y\) and \(y < x\); the irreflexivity of \(<\) implies that
\(x \neq y\); let \(a\) be an element of the set \(Y = \{ x, y \}\)
such that \(t \nless a\) for all \(t \in Y\); as \(x < y\),
\(a \neq y\), so \(a = x\); therefore, \(y \nless x\).  Similarly, if
\(y < x\), then \(x \neq y\) and \(x \nless y\).  It follows that
exactly one of the conditions \(x < y\), \(x = y\), \(y < x\) holds.

Suppose \(x, y, z \in X\) are such that \(x < y\) and \(y < z\).  Let
\(a\) be an element of the set \(Y = \{ x, y, z \}\) such that
\(a \nless t\) for all \(t \in Y\).  As \(x < y\), \(a \neq y\), and
as \(y < z\), \(a \neq z\), so \(a = x\).  Thus, \(z \nless x\).
Also, \(x \neq z\), for otherwise the hypotheses \(x < y\) and
\(y < z\) would imply that two of the conditions \(x < y\), \(x = y\),
\(y < x\) hold.  Thus, \(x \neq z\) and \(z \nless x\).  It follows
from the first hypothesis on \(<\) that \(x < z\).  Therefore, the
relation \(<\) is transitive.

As \(<\) is irreflexive and transitive, by \Cref{thm:3uwc4uyw}, its
reflexive closure \(\leq\) is an order on \(X\).  It remains to check
that every nonempty subset \(Y\) of \(X\) has a minimum with respect
to \(\leq\).  Let \(a\) be an element of \(Y\) such that
\(x \nless a\) for all \(x \in Y\).  Then the first hypothesis on
\(<\) implies that for every \(x \in Y\), we have either \(x = a\) or
\(a < x\), that is, \(a \leq x\).  Thus, \(a\) is the minimum of \(Y\)
with respect to \(\leq\).  This verifies that \(\leq\) is a well-order
on \(X\).

\begin{theorem}
  \label{thm:htdzbbwv}
  Every segment \(Y\) of a well-ordered set \(X\) that is a proper
  subset of \(X\) is an interval \((-\infty, a)\) for a unique element
  \(a\) of \(X\); in fact, \(a\) is the minimum of the complement of
  \(Y\) in \(X\).
\end{theorem}

The complement \(Z\) of a proper segment \(Y\) of \(X\), being
nonempty, has a minimum \(a\).  For any element \(y\) of \(Y\), the
interval \((-\infty,y]\) is contained in \(Y\), so the element \(a\)
of \(X\) that lies outside \(Y\) is not in that interval; the
trichotomy property implies that \(a > y\).  Conversely, any element
\(y\) of \(X\) that is \(<\) the minimum \(a\) of \(Z\) lies outside
\(Z\), and is hence in \(Y\).  Therefore, \(Y = (-\infty,a)\).  If
\(a'\) is any element of \(X\) such that \(Y = (-\infty, a')\), then
\(a'\), which is the minimum of the complement \([a', \infty)\) of the
interval \((-\infty, a')\), is also the minimum of the complement of
\(Y\); thus, \(a' = a\).

\begin{theorem}
  \label{thm:b4yv5x3a}
  The function \(x \mapsto (-\infty, x)\) is an isomorphism from any
  well-ordered set \(X\) onto the set of proper segments of \(X\)
  ordered by inclusion.  In particular, the set of segments of \(X\)
  is well-ordered by inclusion.
\end{theorem}

The transitivity of \(<\) and \Cref{thm:htdzbbwv} imply that the
stated map is an increasing bijection from the totally ordered set
\(X\) onto the ordered set \(S\) of proper segments of \(X\), and is
hence an isomorphism.  It follows that \(S\) is well-ordered by
inclusion.  As every element of \(S\) is contained in \(X\), the set
of all segments of \(X\) is also well-ordered.

\begin{theorem}
  \label{thm:36vndfrs}
  Any segment \(Y\) of a well-ordered set \(X\) satisfies exactly one
  of the following conditions:
  \begin{enumerate}
  \item \(Y\) is an interval \((-\infty, a]\) for a unique
    \(a \in X\).
  \item \(Y\) is the union of the intervals \((-\infty, y)\) as \(y\)
    runs over \(Y\).
  \end{enumerate}
\end{theorem}

In case \(Y\) has a maximum \(a\), it is contained in the interval
\((-\infty, a]\); as \(a\) belongs to the segment \(Y\),
\((-\infty, a]\) is contained in \(Y\); therefore,
\(Y = (-\infty, a]\); anti-symmetry implies that \(a\) is the unique
element of \(X\) for which this equality is true.  On the other hand,
if \(Y\) does not have a maximum, for every element \(t\) of \(Y\),
there is an element \(y\) of \(Y\) such that \(y > t\), and because
\(Y\) is a segment, \((-\infty, y)\) is contained in \(Y\); thus,
\(Y\) is the union of the segments \((-\infty, y)\) as \(y\) runs over
\(Y\).

\begin{theorem}
  \label{thm:y3go6w2h}
  Let \(S\) be a set of segments of a well-ordered set \(X\) such
  that:
  \begin{enumerate}
  \item if \(x \in X\) and \((-\infty, x) \in S\), then
    \((-\infty, x] \in S\); and
  \item the union of any family of elements of \(S\) belongs to \(S\).
  \end{enumerate}
  Then every segment of \(X\) belongs to \(S\).
\end{theorem}

If the set \(T\) of segments of \(X\) that do not belong to \(S\) is
non-empty, then \Cref{thm:b4yv5x3a} implies that \(T\) has a minimum
\(Y\) under inclusion.  By \Cref{thm:36vndfrs}, this segment \(Y\) is
either \((-\infty, a]\) for some \(a \in X\), or the union of the
segments \((-\infty, y)\) as \(y\) runs over \(Y\).  In the first
case, the segment \((-\infty, a)\) is a proper subset of the minimum
\(Y\) of \(T\), and hence belongs to \(S\); the first hypothesis on
\(S\) implies that \(Y \in S\), a contradiction.  In the second case,
the segments \((-\infty, y)\) for \(y \in Y\) are proper subsets of
\(Y\), and hence belong to \(S\); the second hypothesis on \(S\)
implies that \(Y \in S\), again a contradiction.

An \firstterm{inductive} subset of an ordered set \(X\) is a subset
\(Y\) of \(X\) with the property that any element \(x\) of \(X\) such
that \((-\infty, x) \subset Y\) belongs to \(Y\).

\begin{theorem}[Principle of transfinite induction]
  \label{thm:5vxgawq0}
  The only inductive subset of a well-ordered set \(X\) is \(X\)
  itself.
\end{theorem}

If the complement \(Z\) of an inductive subset \(Y\) of \(X\) is
nonempty, then \(Z\) has a minimum \(a\).  Trichotomy implies that
every element \(x\) of \((-\infty, a)\) belongs to \(Y\); as \(Y\) is
inductive, \(a \in Y\), a contradiction.

Here is another proof, one that uses \Cref{thm:y3go6w2h}.  If an
inductive subset \(Y\) of \(X\) contains the interval \((-\infty, x)\)
for some \(x \in X\), it contains \(x\) also, hence
\((-\infty, x] \subset Y\).  On the other hand, the union of any
family of segments contained in \(Y\) is obviously a subset of \(Y\).
Therefore, every segment of \(X\) is contained in \(Y\).  In
particular, \(X \subset Y\).

Given a function of sets \(p : E \to X\), a \firstterm{section} of
\(p\) on a subset \(Y\) of \(X\) is a function \(s : Y \to E\) such
that \(p(s(x)) = x\) for all \(x \in Y\).  The set of all sections of
\(p\) on \(Y\) is denoted below by the symbol \(\Gamma(Y, p)\).

\begin{theorem}[Definition by transfinite recursion]
  \label{thm:srxrngd8}
  Let \(X\) be a well-ordered set, and \(p : E \to X\) a function from
  a set \(E\) to \(X\).  Suppose that for every element \(x\) of
  \(X\), \(f_x\) is a function from \(\Gamma((-\infty, x), p)\) to
  \(p^{-1}(x)\).  Then there exists a unique section \(u\) of \(p\) on
  \(X\) such that \(u(x) = f_x(u \vert_{(-\infty, x)})\) for all
  \(x \in X\).
\end{theorem}

Call a section \(u\) of \(p\) on a segment \(Y\) of \(X\)
\emph{inductive} if \(u(x) = f_x(u \vert_{(-\infty, x)})\) for all
\(x \in Y\).  The set of points where two inductive sections \(u\) and
\(v\) of \(p\) on \(Y\) agree is an inductive subset of the
well-ordered set \(Y\), so transfinite induction (\Cref{thm:5vxgawq0})
implies that \(u = v\).  The restriction of an inductive section of
\(p\) on \(Y\) to a segment contained in \(Y\) is also inductive.

Let \(S\) be the set of segments \(Y\) of \(X\) such that there is an
inductive section of \(p\) on \(Y\).  We will show that \(S\)
satisfies the conditions of \Cref{thm:y3go6w2h}.

Suppose \(a \in X\) and \(Y = (-\infty, a)\) belongs to \(S\).  Let
\(u\) be an inductive section of \(p\) on \(Y\).  Define a section
\(v\) of \(p\) on \(Z = (-\infty, a]\) as follows:
\begin{displaymath}
  v(x) =
  \begin{cases}
    u(x) & \text{if} ~ x \in Y \\
    f_a(u) & \text{if} ~ x = a.
  \end{cases}
\end{displaymath}
As \(v\) extends \(u\),
\begin{displaymath}
  v(x) = u(x) = f_x(u \vert_{(-\infty, x)}) =
  f_x(v \vert_{(-\infty, x)})
\end{displaymath}
for all \(x \in Y\), and
\begin{displaymath}
  v(a) = f_a(u) = f_a(v \vert_{(-\infty, a)}).
\end{displaymath}
Therefore, \(v\) is inductive, and \(Z \in S\).

Suppose, on the other hand, that \(Y\) is the union of a family
\((Y_i)_{i \in I}\) of elements of \(S\).  On each \(Y_i\), there is
an inductive section \(u_i\) of \(p\).  For all \(i, j \in I\), the
restriction \(u_{ij}\) of \(u_i\) to the segment
\(Y_{ij} = Y_i \cap Y_j\) is inductive.; as the restriction \(u_{ji}\)
of \(u_j\) to \(Y_{ij}\) is also inductive, the uniqueness statement
above implies that \(u_{ij} = u_{ji}\).  There is, therefore, a unique
function \(u : Y \to E\) which restricts to \(u_i\) on each \(Y_i\).
It is obviously an inductive section of \(p\), hence \(Y \in S\).

\Cref{thm:y3go6w2h} implies that every segment of \(X\) belongs to
\(S\).  In particular, \(X \in S\).

\begin{theorem}[Definition by transfinite recursion]
  \label{thm:5kg5ewo1}
  Let \(X\) be a well-ordered set, and \(T\) any set.  Suppose that
  for every element \(x\) of \(X\), \(f_x\) is a function from
  \(T^{(-\infty, x)}\) to \(T\).  Then there exists a unique function
  \(u : X \to T\) such that \(u(x) = f_x(u \vert_{(-\infty, x)})\) for
  all \(x \in X\).
\end{theorem}

Any fibre of the first projection \(p : X \times T \to X\) is
canonically identified with \(T\), and a section of \(p\) on a subset
\(Y\) of \(X\) is canonically identified with a function from \(Y\) to
\(T\).  Therefore, the present statement follows from
\Cref{thm:srxrngd8}.

\begin{theorem}
  \label{thm:hw1jxum0}
  If \(f\) is a strictly increasing function from a well-ordered set
  \(X\) to itself, then \(x \leq f(x)\) for all \(x \in X\).
\end{theorem}

If the set \(M\) of \(x \in X\) such that \(x > f(x)\) is non-empty,
it has a minimum \(a\).  Let \(b = f(a)\).  As \(a \in M\), \(a > b\);
as \(f\) is strictly increasing, \(f(a) > f(b)\), that is,
\(b > f(b)\).  Thus, \(b \in M\) and \(b < a\), a contradiction.

\begin{theorem}
  \label{thm:yvftewxp}
  The identity function is the only automorphism of any well-ordered
  set.
\end{theorem}

Let \(f\) be an automorphism of a well-ordered set \(X\), and \(g\)
its inverse.  Being increasing bijections, they are strictly
increasing.  \Cref{thm:hw1jxum0} implies that for any \(x \in X\),
\(x \leq f(x)\) and \(x \leq g(x)\); putting \(f(x)\) in the place of
\(x\) in the second of these inequalities gives
\(f(x) \leq g(f(x)) = x\); by anti-symmetry, \(f(x) = x\).

\begin{theorem}
  \label{thm:uvocd4lt}
  There is at most one isomorphism from a well-ordered set to another.
\end{theorem}

If \(f\) and \(g\) are two isomorphisms from a well-ordered set \(X\)
to another well-ordered set \(Y\), then \(g \circ f^{-1}\) is an
automorphism of \(X\).  It must be the identity map of \(X\) by
\Cref{thm:yvftewxp}, so \(f = g\).

\begin{theorem}
  \label{thm:q14zcdkw}
  If \(f\) is an isomorphism from a well-ordered set \(X\) onto a
  segment \(Y\) of \(X\), then \(Y = X\) and \(f\) is the identity
  function of \(X\).
\end{theorem}

Let \(f' : X \to X\) be the composite of \(f\) and the inclusion map
from \(Y\) to \(X\).  It is increasing and injective, hence strictly
increasing.  \Cref{thm:hw1jxum0} implies that for any \(x \in X\),
\(x \leq f'(x) = f(x)\); as \(Y = f(X)\) is a segment of \(X\),
\(x \in Y\).  Therefore, \(Y = X\).  It follows from
\Cref{thm:yvftewxp} that \(f\) is the identity map of \(X\).

\begin{theorem}
  \label{thm:7z7bwher}
  Let \(X\) and \(Y\) be well-ordered sets, and suppose that \(f\) is
  an isomorphism from \(X\) onto a segment \(Y'\) of \(Y\), and \(g\)
  is an isomorphism from \(Y\) onto a segment \(X'\) of \(X\).  Then
  \(X' = X\), \(Y' = Y\), and \(g = f^{-1}\).
\end{theorem}

The composite \(g \circ f\) is an isomorphism from \(X\) onto the
segment \(X'\) of \(X\).  By \Cref{thm:q14zcdkw}, \(X' = X\) and
\(g \circ f\) is the identity function of \(X\).  Similarly,
\(Y' = Y\) and \(f \circ g\) is the identity function of \(Y\).

\begin{theorem}
  \label{thm:vy95iuyi}
  For any two well-ordered sets \(X\) and \(Y\), either \(X\) is
  isomorphic to a segment of \(Y\), or \(Y\) is isomorphic to a
  segment of \(X\).
\end{theorem}

Let \(X_0\) be the set of elements \(x\) of \(X\) for which there
exists an element \(y\) of \(Y\) such that the interval
\((-\infty, x)\) of \(X\) is isomorphic to the interval
\((-\infty, y)\) of \(Y\).  If \(y, y' \in Y\) and \((-\infty, y)\) is
isomorphic to \((-\infty, y'\), then \(y = y'\); for if, for instance,
\(y < y'\), then \((-\infty, y')\) is isomorphic to its segment
\((-\infty, y)\), so by \Cref{thm:q14zcdkw},
\((-\infty, y) = (-\infty, y')\), a contradiction.  Thus, for every
\(x \in X_0\), there is a unique element \(y\) of \(Y\) such that
\((-\infty, x)\) is isomorphic to \((-\infty, y)\).  We thus have a
function \(f : X_0 \to Y\) such that \((-\infty, x)\) is isomorphic to
\((-\infty, f(x))\).  Its image \(Y_0\) consists of the points
\(y \in Y\) for which there is an \(x \in X\) such that
\((-\infty, x)\) is isomorphic to \((-\infty, y)\).  Let
\(g : X_0 \to Y_0\) be the function induced by \(f\).

Suppose \(x \in X_0\) and \(x' < x\).  Let \(u\) be an isomorphism
from \((-\infty, x)\) onto \((-\infty, g(x))\).  Then \(u\) and
\(u^{-1}\) are strictly increasing, so \(u((-\infty, x'))\) equals
\((-\infty, u(x'))\).  Thus, \(u\) restricts to an isomorphism from
\((-\infty, x')\) onto \((-\infty, u(x'))\).  This means that
\(x' \in X_0\) and \(g(x') = u(x') < g(x)\).  Therefore, \(X_0\) is a
segment of \(X\), and \(g\) is strictly increasing.  As \(g\) is
surjective as well and \(X_0\) is totally ordered, \(g\) is an
isomorphism.  In particular, \(Y_0\) is a segment of \(Y\).  Thus,
\(g\) is an isomorphism from the segment \(X_0\) of \(X\) onto the
segment \(Y_0\) of \(Y\).

It therefore suffices to prove that either \(X_0 = X\) or \(Y_0 = Y\).
If \(X_0 \neq X\) and \(Y_0 \neq Y\), \Cref{thm:htdzbbwv} implies that
\(X_0 = (-\infty, a)\) and \(Y_0 = (-\infty, b)\) for some \(a \in X\)
and \(b \in Y\).  This means that \(g\) is an isomorphism from
\((-\infty, a)\) onto \((-\infty, b)\).  Therefore,
\(a \in X_0 = (-\infty, a)\), a contradiction.

\begin{theorem}
  \label{thm:xhbclgwz}
  Every subset of a well-ordered set \(X\) is isomorphic to a segment
  of \(X\).
\end{theorem}

Suppose \(Y\) is a subset of \(X\) that is not isomorphic to any
segment of \(X\).  By \Cref{thm:vy95iuyi}, there is an isomorphism
\(f\) from \(X\) onto a proper segment \(Y'\) of \(Y\).
\Cref{thm:htdzbbwv} gives an element \(b\) of \(Y\) such that
\begin{displaymath}
  Y' = \{ y \in Y \,\vert\, y < b \}.
\end{displaymath}
Let \(g : X \to X\) denote the composite of \(f\) and the inclusion
from \(Y'\) to \(X\).  It is strictly increasing, so
\Cref{thm:hw1jxum0} implies that \(b \leq g(b)\), a contradiction
because \(g(b) \in Y'\).

Suppose \((X_i)_{i \in I}\) is a family of ordered sets indexed by a
well-ordered set \(I\).  Define a relation \(<\) on the product set
\(X = \prod_{i \in I} X_i\) by setting \(x < y\) if \(x \neq y\) and
\(x_j < y_j\), where \(x = (x_i)\), \(y = (y_i)\), and \(j\) is the
minimum of the set of all \(i \in I\) satisfying \(x_i \neq y_i\).
The reflexive closure of \(<\) is an order on \(X\), which is called
the \firstterm{lexicographic} order.  It is a total order if every
\(X_i\) is totally ordered.

An element \(x\) of an ordered set \(X\) is called a
\firstterm{successor} of an element \(w\) of \(X\) if \(w < x\) and
there is no element \(t\) of \(X\) such that \(w < t < x\); this
condition is equivalent to the condition that \(w < x\) and for any
element \(t\) of \(X\) such that \(w \leq t \leq x\), we have either
\(w = t\) or \(t = x\).  An element of \(X\) is called a
\firstterm{successor element} of \(X\) if it is a successor of some
element of \(X\).  An element of \(X\) is called a \firstterm{limit
  element} of \(X\) if it is not a successor element of \(X\).  It is
obvious that the minimum of \(X\), if it exists, is a limit element of
\(X\).

For any element \(w\) of an ordered set \(X\), let
\begin{displaymath}
  w^+ =
  \begin{cases}
    \min \, (w, \infty)
    & \text{if the interval} ~ (w, \infty) ~
      \text{has a minimum} \\
    w
    & \text{if} ~ (w, \infty) ~
      \text{does not have a minimum}.
  \end{cases}
\end{displaymath}
Similarly, for any element \(x\) of \(X\), let
\begin{displaymath}
  x^- =
  \begin{cases}
    \max \, (-\infty, x)
    & \text{if the interval} ~ (-\infty, x) ~
      \text{has a maximum} \\
    x
    & \text{if} ~ (-\infty, x) ~
      \text{does not have a maximum}.
  \end{cases}
\end{displaymath}
It is obvious that for any \(w \in X\), \(w \leq w^+\) with equality
if and only if the interval \((w, \infty)\) does not have a minimum.
Similarly, for any \(x \in X\), \(x^- \leq x\) with equality if and
only if the interval \((-\infty, x)\) does not have a maximum.

\begin{theorem}
  \label{thm:q8h11s4b}
  Let \(X\) be a totally ordered set, \(P(X)\) the set of elements
  \(w\) of \(X\) such that \(w < w^+\), and \(S(X)\) the set of
  elements \(x\) of \(X\) such that \(x^- < x\).
  \begin{enumerate}
  \item For any element \(w\) of \(X\), \(w^+\) is the infimum of the
    interval \((w, \infty)\).  In particular, if \(w, t\) are elements
    of \(X\) such that \(w < t\), then \(w^+ \leq t\).
  \item For any element \(x\) of \(X\), \(x^-\) is the supremum of the
    interval \((-\infty, x)\).  In particular, if \(x, t\) are
    elements of \(X\) such that \(t < x\), then \(t \leq x^-\).
  \item An element \(w\) of \(X\) has a successor if and only if it
    belongs to \(P(X)\); in that case, \(w^+\) is the unique successor
    of \(w\).  Similarly, an element \(x\) of \(X\) is a successor
    element if and only if it belongs to \(S(X)\); in that case,
    \(x^-\) is the unique element of \(X\) which has \(x\) as a
    successor.  In particular, an element of \(X\) has at most one
    successor, and is the successor of at most one element of \(X\).
  \item The function \(w \mapsto w^+\) from \(X\) to \(X\) induces an
    isomorphism of ordered sets from \(P(X)\) onto \(S(X)\).  The
    inverse of this bijection maps an element \(x\) of \(S(X)\) to
    \(x^-\).
  \item An element \(x\) of \(X\) is a limit element of \(X\) if and
    only if \(x^- = x\), that is, if and only if the interval
    \((-\infty, x)\) does not have a maximum.  In particular, if \(x\)
    is a limit element of \(X\), then the set \((-\infty, x)\) is
    either empty or infinite, depending on whether \(x\) is the
    minimum of \(X\) or not.
  \item If \(X\) is well-ordered, then an element of \(X\) has a
    successor if and only if it is not the maximum of \(X\).
  \end{enumerate}
\end{theorem}

Let \(w \in X\).  If the interval \((w, \infty)\) has a minimum \(x\),
then \(w^+ = x\) is obviously the infimum of \((w, \infty)\).  Suppose
\((w, \infty)\) does not have a minimum.  It is obvious that \(w\) is
a lower bound of \((w, \infty)\).  Let \(t\) be an arbitrary lower
bound of \((w, \infty)\).  If \(w < t\), then \(t \in (w, \infty)\);
as \(t\) is a lower bound of \((w, \infty)\), this implies that \(t\)
is the minimum of \((w, \infty)\), contradicting the hypothesis that
\((w, \infty)\) does not have a minimum.  Therefore, \(w \nless t\).
As \(X\) is totally ordered, it follows that \(t \leq w\).  Therefore,
\(w^+ = w\) is the infimum of \((w, \infty)\).  It follows that in
both the cases, \(w^+\) is the infimum of \((w, \infty)\).  If
\(w, t \in X\) are such that \(w < t\), then \(t \in (w, \infty)\); as
\(w^+\) is the infimum of \((w, \infty)\), it is a lower bound of
\((w, \infty)\), hence \(w^+ \leq t\).  This verifies the first part
of the theorem.

The proof of the second part of the theorem is similar to that of the
first part.

We will now show that an element \(x\) of \(X\) is a successor of an
element \(w\) of \(X\) if and only if \(w < w^+ = x\) if and only if
\(w = x^- < x\).  Suppose \(x\) is a successor of \(w\); then
\(w < x\), so \(x \in (w, \infty)\); let \(t\) be any element of
\((w, \infty)\); if \(t < x\), then \(w < t < x\), contradicting the
assumption that \(x\) is a successor of \(w\); therefore, the
trichotomy property of \(X\) implies that \(x \leq t\); it follows
that \(x\) is the minimum of \((w, \infty)\); thus, \(w^+ = x\) and
\(w < x = w^+\).  Suppose \(w < w^+ = x\); then \(w^+\), hence \(x\),
is the minimum of \((w, \infty)\); in particular, \(x\) belongs to
\((w, \infty)\), so \(w < x\), hence \(w\) belongs to
\((-\infty, x)\); let \(t\) be any element of \((-\infty, x)\); then
\(t < x\); as \(x\) is the minimum of \((w, \infty)\), \(x \leq s\)
for all \(s \in (w, \infty)\), so \(t \notin (w, \infty)\); therefore,
\(t \leq w\); it follows that \(w\) is the maximum of
\((-\infty, x)\); thus, \(x^- = w\) and \(x^- = w < x\).  Suppose
\(w = x^- < x\); then \(w < x\), and \(x^-\), hence \(w\), is the
maximum of \((-\infty, x)\); let \(t\) be any element of \(X\) such
that \(w \leq t \leq x\); suppose \(t \neq x\); then \(t < x\), so
\(t \in (-\infty, x)\); as \(w\) is the maximum of \((-\infty, x)\),
this implies that \(t \leq w\); therefore, \(w = t\); it follows that
\(x\) is a successor of \(w\).  This verifies the equivalences stated
at the beginning of this paragraph.

For any element \(w\) of \(X\), the equivalence
\begin{displaymath}
  x ~ \text{is a successor of} ~ w
  \Leftrightarrow
  w < w^+ = x,
\end{displaymath}
valid for any element \(x\) of \(X\), says that \(w\) has a successor
if and only if \(w \in P(X)\), and in that case, \(w^+\) is the unique
successor of \(w\).  Similarly, for any element \(x\) of \(X\), the
equivalence
\begin{displaymath}
  x ~ \text{is a successor of} ~ w
  \Leftrightarrow
  w = x^- < x,
\end{displaymath}
valid for any element \(w\) of \(X\), says that \(x\) is a successor
element of \(X\) if and only if \(x \in S(X)\), and in that case,
\(x^-\) is the unique element of \(X\) which has \(x\) as a successor.
So the third part of the theorem is proved.

Thus, the function \(w \mapsto w^+\) from \(X\) to \(X\) induces a
function \(\sigma : P(X) \to S(X)\), and the function
\(x \mapsto x^-\) from \(X\) to \(X\) induces a function
\(\pi : S(X) \to P(X)\).  If \(w\) is any element of \(P(X)\), then
\(x = w^+\) is a successor element of \(X\) because it is the
successor of \(w\), so \(x^-\) is the unique element of \(X\) which
has \(x\) as a successor, hence \(x^- = w\), that is, \((w^+)^- = w\);
thus, \(\tau \circ \sigma = \mathbb{1}_{P(X)}\).  Similarly, if \(x\)
is any element of \(S(X)\), then \(w = x^-\) has a successor, namely,
\(x\), so \(w^+\) is the unique successor of \(w\); therefore,
therefore, \(w^+ = x\), that is, \((x^-)^+ = x\); this means that
\(\sigma \circ \pi = \mathbb{1}_{S(X)}\).  Thus, the functions
\(\sigma\) and \(\pi\) are mutual inverses.  If \(w_1, w_2\) are
elements of \(P(X)\) such that \(w_1 < w_2\), then by the first part
of the theorem, \(w_1^+ \leq w_2\); as \(w_2 \in P(X)\),
\(w_2 < w_2^+\); therefore, \(w_1^+ < w_2^+\).  Thus, \(\sigma\) is a
strictly increasing bijection from \(P(X)\) onto \(S(X)\), and is
hence an isomorphism from \(P(X)\) onto \(S(X)\).  This checks the
fourth part of the theorem.

By the third part of the theorem, \(S(X)\) is the set of successor
elements of \(X\), so the set of limit elements of \(X\) is the set
\begin{align*}
  X \setminus S(X)
  &=
    \{ x \in X \,\vert\, x^- = x \} \\
  &=
    \{ x \in X \,\vert\,
    (-\infty, x) ~ \text{does not have a maximum} \}.
\end{align*}
An element \(x\) of \(X\) is the minimum of \(X\) if and only if
\((-\infty, x)\) is empty.  On the other hand, if \((-\infty, x)\) is
nonempty and does not have a maximum, then it must be infinite because
every nonempty finite subset of a totally ordered set has a maximum.
This verifies the fifth part of the theorem.

Lastly, if \(X\) is well-ordered and \(w \in X\), then the set
\((w, \infty)\) has a minimum if and only if it is nonempty, that is,
if and only if \(w\) is not the maximum of \(X\) in case it exists.
The last part of the theorem follows.

\begin{theorem}[Principle of transfinite induction]
  \label{thm:g5v2vli5}
  Let \(A\) be a subset of a well-ordered set \(X\) satisfying the
  following conditions:
  \begin{enumerate}
  \item If \(x\) is a successor element of \(X\), and \(x^- \in A\),
    then \(x \in A\).
  \item If \(x\) is a limit element of \(X\), and
    \((-\infty, x) \subset A\), then \(x \in A\).
  \end{enumerate}
  Then \(A\) equals \(X\).
\end{theorem}

Suppose \(x\) is an element of \(X\) such that
\((-\infty, x) \subset A\).  We will show that \(x \in A\).  If \(x\)
is a successor element of \(X\), then by \Cref{thm:q8h11s4b},
\(x^- < x\); as \((-\infty, x) \subset A\), it follows that
\(x^- \in A\); so the first hypothesis on \(A\) implies that
\(x \in A\).  On the other hand, if \(x\) is a limit element of \(X\),
then as \((-\infty, x) \subset A\), the second hypothesis on \(A\)
implies that \(x \in A\).  Thus, in both the cases, \(x \in A\).
Therefore, \(A\) is an inductive subset of \(X\), so by
\Cref{thm:5vxgawq0}, \(A = X\).

\begin{theorem}[Definition by transfinite recursion]
  \label{thm:83vd5knr}
  Let \(X\) be a well-ordered set, and \(p : E \to X\) a function from
  a set \(E\) to \(X\).  Suppose that:
  \begin{enumerate}
  \item For every successor element \(x\) of \(X\), \(f_x\) is a
    function from \(p^{-1}(x^-)\) to \(p^{-1}(x)\).
  \item For every limit element \(x\) of \(X\), \(g_x\) is a function
    from \(\Gamma((-\infty, x), p)\) to \(p^{-1}(x)\).
  \end{enumerate}
  Then there exists a unique section \(u\) of \(p\) on \(X\) such
  that:
  \begin{enumerate}
  \item \(u(x) = f_x(u(x^-))\) for every successor element \(x\) of
    \(X\), and
  \item \(u(x) = g_x(u \vert_{(-\infty, x)})\) for every limit element
    \(x\) of \(X\).
  \end{enumerate}
\end{theorem}

For every element \(x\) of \(X\), define a function \(h_x\) from
\(\Gamma((-\infty, x), p)\) to \(p^{-1}(x)\) as follows:
\begin{enumerate}
\item If \(x\) is a successor element of \(X\), let
  \(h_x(s) = f_x(s(x^-))\) for every section \(s\) of \(p\) on
  \((-\infty, x)\).
\item If \(x\) is a limit element of \(X\), let \(h_x = g_x\).
\end{enumerate}
Then transfinite recursion (\Cref{thm:srxrngd8}) gives a unique
section \(u\) of \(p\) such that
\(u(x) = h_x(u \vert_{(-\infty, x)})\) for all \(x \in X\).  Thus, if
\(x\) is a successor element of \(X\), then
\begin{displaymath}
  u(x) =
  h_x(u \vert_{(-\infty, x)}) =
  f_x(u \vert_{(-\infty, x)} (x^-)) =
  f_x(u(x^-)),
\end{displaymath}
and if \(x\) is a limit element of \(X\), then
\begin{displaymath}
  u(x) = h_x(u \vert_{(-\infty, x)}) = g_x(u \vert_{(-\infty, x)}).
\end{displaymath}
Therefore, \(u\) has the required properties.

Suppose \(u'\) is another section of \(p\) which has the required
properties.  Then for every successor element \(x\) of \(X\), we have
\begin{displaymath}
  u'(x) =
  f_x(u'(x^-)) =
  f_x(u' \vert_{(-\infty, x)}(x^-)) =
  h_x(u' \vert_{(-\infty, x)}),
\end{displaymath}
while for every limit element \(x\) of \(X\), we have
\begin{displaymath}
  u'(x) = g_x(u' \vert_{(-\infty, x)}) = h_x(u' \vert_{(-\infty, x)}).
\end{displaymath}
Thus, \(u'(x) = h_x(u' \vert_{(-\infty, x)})\) for all \(x \in X\).
As \(u\) is the only section of \(p\) on \(X\) with this property, we
get \(u' = u\).

It is convenient to expand \Cref{thm:83vd5knr} as follows.

\begin{theorem}[Definition by transfinite recursion]
  \label{thm:hpleq0rr}
  Let \(X\) be a nonempty well-ordered set, and \(p : E \to X\) a
  function from a set \(E\) to \(X\).  Suppose that:
  \begin{enumerate}
  \item \(e\) is an element of \(p^{-1}(a)\), where \(a\) is the
    minimum of \(X\).
  \item For every successor element \(x\) of \(X\), \(f_x\) is a
    function from \(p^{-1}(x^-)\) to \(p^{-1}(x)\).
  \item For every limit element \(x\) of \(X\) other than the minimum
    \(a\) of \(X\), \(g_x\) is a function from
    \(\Gamma((-\infty, x), p)\) to \(p^{-1}(x)\).
  \end{enumerate}
  Then there exists a unique section \(u\) of \(p\) on \(X\) such
  that:
  \begin{enumerate}
  \item \(u(a) = e\),
  \item \(u(x) = f_x(u(x^-))\) for every successor element \(x\) of
    \(X\), and
  \item \(u(x) = g_x(u \vert_{(-\infty, x)})\) for every limit element
    \(x\) of \(X\) other than \(a\).
  \end{enumerate}
\end{theorem}

As the set \((-\infty, a)\) is empty, \(\Gamma((-\infty, a), p)\) is a
singleton consisting of the unique function from the empty set to
\(E\).  So a function from \(\Gamma((-\infty, a), p)\) to
\(p^{-1}(a)\) is canonically identified with an element of
\(p^{-1}(a)\).  Therefore, the theorem follows from
\Cref{thm:83vd5knr}.

\begin{theorem}[Definition by transfinite recursion]
  \label{thm:a61v7g3m}
  Let \(X\) be a well-ordered set, and \(T\) any set.  Suppose that
  for every successor element \(x\) of \(X\), \(f_x\) is a function
  from \(T\) to \(T\), and that for every limit element \(x\) of
  \(X\), \(g_x\) is a function from \(T^{(-\infty, x)}\) to \(T\).
  Then there exists a unique function \(u : X \to T\) such that
  \(u(x) = f_x(u(x^-))\) for every successor element \(x\) of \(X\),
  and \(u(x) = g_x(u \vert_{(-\infty, x)})\) for every limit element
  \(x\) of \(X\).
\end{theorem}

This theorem can be deduced from \Cref{thm:83vd5knr} just as
(\ref{thm:5kg5ewo1}) was deduced from (\ref{thm:srxrngd8}).

\begin{theorem}[Definition by transfinite recursion]
  \label{thm:e3bitdp0}
  Let \(X\) be a nonempty well-ordered set, and \(T\) any set.
  Suppose that \(a\) is an element of \(T\), that for every successor
  element \(x\) of \(X\), \(f_x\) is a function from \(T\) to \(T\),
  and that for every limit element \(x\) of \(X\) other than the
  minimum \(a\) of \(X\), \(g_x\) is a function from
  \(T^{(-\infty, x)}\) to \(T\).  Then there exists a unique function
  \(u : X \to T\) such that \(u(a) = e\), where \(a\) is the minimum
  of \(X\), \(u(x) = f_x(u(x^-))\) for every successor element \(x\)
  of \(X\), and \(u(x) = g_x(u \vert_{(-\infty, x)})\) for every limit
  element \(x\) of \(X\) other than \(a\).
\end{theorem}

This theorem is an expansion of \Cref{thm:a61v7g3m} just as
(\ref{thm:hpleq0rr}) was an expansion of (\ref{thm:83vd5knr}).

\subsection{Ordinals}
\label{sec:t851d2lg}

A set \(\alpha\) is called \firstterm{transitive} if every element of
\(\alpha\) is a subset of \(\alpha\); that is, if \(\beta \in \alpha\)
and \(\gamma \in \beta\), then \(\gamma \in \alpha\).

An \firstterm{ordinal} is a set \(\alpha\) with the following
properties:
\begin{enumerate}
\item \(\alpha\) is a transitive set.
\item For any two distinct elements \(\beta, \gamma\) of \(\alpha\),
  either \(\beta \in \gamma\) or \(\gamma \in \beta\).
\item Every nonempty subset \(x\) of \(\alpha\) has an element
  \(\beta\) such that \(\beta \cap x = \emptyset\).
\end{enumerate}

The empty set is obviously an ordinal.  For any natural number \(n\),
let \([n]\) denote the set of natural numbers that are less than
\(n\).  Then the function \(n \mapsto [n]\) is an injection from the
set \(\mathbb{N}\) of natural numbers to its power set
\(\mathcal{P}(\mathbb{N})\).  We will identify \(\mathbb{N}\) with its
image \(\omega\) by this injection.  With this identification done,
every natural number \(n\) is the set of all the natural numbers that
are less than \(n\):
\begin{align*}
  0 & = \emptyset, \\
  1 &= \{ 0 \} = \{ \emptyset \}, \\
  2 &= \{ 0, 1 \} = \{ \emptyset, \{ \emptyset \} \}, \\
  3 &= \{ 0, 1, 2 \} =
      \{
      \emptyset, \{ \emptyset \}, \{ \emptyset, \{ \emptyset \} \}
      \},
\end{align*}
and so on.  Now every natural number is an ordinal, and the set
\(\omega\) of all the natural numbers is itself an ordinal.

If \(\alpha\) is a nonempty ordinal, then the third condition in the
definition of ordinals gives an element \(\beta\) of \(\alpha\) such
that \(\beta \cap \alpha = \emptyset\); as \(\alpha\) is a transitive
set, \(\beta\) is a subset of \(\alpha\), so
\(\beta \cap \alpha = \beta\), hence \(\beta = \emptyset\).  Thus the
empty set is an element of every nonempty ordinal.

\begin{theorem}
  \label{thm:7cc2cckf}
  Let \(\alpha\) be an ordinal.
  \begin{enumerate}
  \item (Well-ordering of \(\alpha\)).  Let \(\leq\) be the reflexive
    closure of the relation \(\in\) on \(\alpha\): for any two
    elements \(\beta, \gamma\) of \(\alpha\), \(\beta \leq \alpha\) if
    \(\beta \in \gamma\) or \(\beta = \gamma\).  Then \(\leq\) is a
    well-order on \(\alpha\).
  \item (Irreflexivity of \(\in\)).  For every \(\beta \in \alpha\),
    we have \(\beta \notin \beta\).  In particular,
    \(\alpha \notin \alpha\).
  \item (Trichotomy).  For any two elements \(\beta, \gamma\) of
    \(\alpha\), exactly one of the conditions \(\beta \in \gamma\),
    \(\beta = \gamma\), \(\gamma \in \beta\) is true.
  \item (Transitivity of elements).  Every element of \(\beta\) is a
    transitive set.
  \end{enumerate}
\end{theorem}

The definition of ordinals implies that the relation \(\in\) has the
following properties:
\begin{enumerate}
\item For any two distinct elements \(\beta, \gamma\) of \(\alpha\),
  either \(\beta \in \gamma\) or \(\gamma \in \beta\).
\item Every nonempty subset \(x\) of \(\alpha\) has an element
  \(\beta\) such that no element \(\gamma\) of \(x\) satisfies the
  condition \(\gamma \in \beta\).
\end{enumerate}
By \Cref{thm:3uwc4uyw}, the reflexive closure \(\leq\) of \(\in\) is a
well-order on \(\alpha\).  As \(\in\) is the irreflexive kernel of the
order \(\leq\) on \(\alpha\), by \Cref{thm:ztywp662} it is irreflexive
and transitive.  The irreflexivity of \(\in\) on \(\alpha\) means that
\(\beta \notin \beta\) for all \(\beta \in \alpha\).  In particular,
if \(\alpha \in \alpha\), then substituting \(\alpha\) for \(\beta\)
in the previous sentence, we have \(\alpha \notin \alpha\), a
contradiction; therefore, \(\alpha \notin \alpha\).  The transitivity
of \(\in\) on \(\alpha\) implies that if
\(\beta, \gamma, \delta \in \alpha\) are such that
\(\beta \in \gamma\) and \(\gamma \in \delta\), then
\(\beta \in \delta\), so the element \(\delta\) of \(\alpha\) is
transitive.

The reflexive closure of the relation \(\in\) on an ordinal \(\alpha\)
is called the \firstterm{canonical order} on \(\alpha\).  It is a
well-order on \(\alpha\) by \Cref{thm:7cc2cckf}.  We will always
denote it by \(\leq\).  Following our general notation for the
irreflexive kernel of an order, we will denote the irreflexive kernel
of \(\leq\) by \(<\); thus, \(<\) is the same as the relation \(\in\)
on \(\alpha\): for all \(\beta, \gamma \in \alpha\),
\(\beta < \gamma\) is just another notation for \(\beta \in \gamma\).

\begin{theorem}
  \label{thm:86se6zr0}
  Let \(\alpha\) be an ordinal, and \(\beta, \gamma\) two elements of
  \(\alpha\).  Then \(\beta \leq \gamma\) if and only if \(\beta\) is
  a subset of \(\gamma\).  Also, \(\beta < \gamma\) if and only if
  \(\beta \in \gamma\) if and only if \(\beta\) is a proper subset of
  \(\gamma\).
\end{theorem}

We have already said that the relations \(<\) and \(\in\) on
\(\alpha\) are the same, so \(\beta < \gamma\) just means that
\(\beta \in \gamma\).  As for the other equivalence,
\(\beta \leq \gamma\) if and only if either \(\beta \in \gamma\) or
\(\beta = \gamma\).  So it suffices to prove that \(\beta \in \gamma\)
if and only if \(\beta\) is a proper subset of \(\gamma\).  We will
use the properties of \(\alpha\) from \Cref{thm:7cc2cckf}.  If
\(\beta \in \gamma\), then as \(\gamma\) is a transitive set by the
transitivity of the elements of \(\alpha\), \(\beta\) is a subset of
\(\gamma\); also, the irreflixivity of \(\in\) on \(\alpha\) and the
condition \(\beta \in \gamma\) imply that \(\beta \neq \gamma\);
therefore, \(\beta\) is a proper subset of \(\gamma\).  Conversely, if
\(\beta\) is a proper subset of \(\gamma\), then
\(\beta \neq \gamma\), hence \(\gamma\) is not a subset of \(\beta\);
as \(\beta\) is a transitive set, this implies that
\(\gamma \notin \beta\); therefore, by the trichotomy property of
\(\alpha\), \(\beta \in \gamma\).

\begin{theorem}
  \label{thm:9hac1s0p}
  A set \(\alpha\) is an ordinal if and only if there is a well-order
  on \(\alpha\) such that \(\beta = (-\infty, \beta)\) for all
  \(\beta \in \alpha\).  In that case, such a well-order coincides
  with the canonical order on the ordinal \(\alpha\), and is, in
  particular, unique.
\end{theorem}

Suppose that \(\alpha\) has a well-order \(\sqsubseteq\) with the
property that \(\beta = (-\infty, \beta)\) for every element \(\beta\)
of \(\alpha\).  We will check that it satisfies the three conditions
that together define an ordinal.  If \(\beta\) is an element of
\(\alpha\), then \(\beta = (-\infty, \beta)\), so as
\((-\infty, \beta)\) is a subset of \(\alpha\), \(\beta\) is a subset
of \(\alpha\); therefore, \(\alpha\) is a transitive set.  If
\(\beta, \gamma\) are distinct elements of \(\alpha\), then as the
well-order \(\sqsubseteq\) is total, its trichotomy implies that
either \(\beta \sqsubset \gamma\) or \(\gamma \sqsubset \beta\); if
\(\beta \sqsubset \gamma\), then
\(\beta \in (-\infty, \gamma) = \gamma\), and similarly if
\(\gamma \sqsubset \beta\), then \(\gamma \in \beta\); therefore,
either \(\beta \in \gamma\) or \(\gamma \in \beta\).  Lastly, any
nonempty subset \(x\) of \(\alpha\) has a minimum \(\beta\) with
respect to the well-order \(\sqsubseteq\); thus,
\(\beta \sqsubseteq \gamma\) for all \(\gamma \in x\); the totality of
\(\sqsubseteq\) implies that for all \(\gamma \in x\), we have
\(\gamma \nsqsubset \beta\), so \(\gamma \notin (-\infty, \beta) =
\beta\); therefore, \(\beta \cap x = \emptyset\).  It follows that
\(\alpha\) is an ordinal.  Now, for any two elements \(\beta, \gamma\)
of \(\alpha\), we have \(\beta \sqsubset \gamma\) if and only if
\(\beta \in (-\infty, \gamma)\); as \((-\infty, \gamma) = \gamma\),
this means that \(\beta \sqsubset \gamma\) if and only if \(\beta \in
\gamma\); therefore, \(\beta \sqsubseteq \gamma\) if and only if
\(\beta \in \gamma\) or \(\beta = \gamma\), that is, if and only if
\(\beta \leq \gamma\), where \(\leq\) is the canonical order on the
ordinal \(\alpha\).  Therefore, \(\sqsubseteq\) coincides with the
canonical order on the ordinal \(\alpha\).

Conversely, suppose that \(\alpha\) is an ordinal, and let \(\leq\)
denote its canonical order.  By \Cref{thm:7cc2cckf}, \(\leq\) is a
well-order on \(\alpha\).  For any element \(\beta\) of \(\alpha\), we
have
\begin{displaymath}
  (-\infty, \beta) =
  \{ \gamma \in \alpha \,\vert\, \gamma < \beta \} =
  \{ \gamma \in \alpha \,\vert\, \gamma \in \beta \} =
  \alpha \cap \beta = \beta
\end{displaymath}
because the irreflexive kernel \(<\) of \(\leq\) is the same as the
relation \(\in\) on \(\alpha\), and the element \(\beta\) of the
transitive set \(\alpha\) is a subset of \(\alpha\).

\begin{theorem}
  \label{thm:5qmhh3to}
  Every element of an ordinal is an ordinal.
\end{theorem}

Let \(\beta\) be an element of an ordinal \(\alpha\).  By
\Cref{thm:7cc2cckf}, \(\beta\) is a transitive set.  If
\(\gamma, \delta\) are distinct elements of \(\beta\), then as
\(\alpha\) is a transitive set, \(\gamma, \delta\) are distinct
elements of \(\alpha\) too; therefore, either \(\gamma \in \delta\) or
\(\delta \in \gamma\).  If \(x\) is a nonempty subset of \(\beta\),
then as \(\alpha\) is a transitive set, \(x\) is a nonempty subset of
\(\alpha\) too; therefore, \(x\) has an element \(\gamma\) such that
\(\gamma \cap x = \emptyset\).  Thus, \(\beta\) satisfies all the
conditions that define an ordinal.

\begin{theorem}
  \label{thm:lebsg6gf}
  An ordinal \(\alpha\) is an element of an ordinal \(\beta\) if and
  only if \(\alpha\) is a proper subset of \(\beta\).
\end{theorem}

If \(\alpha\) is an element of \(\beta\), then as \(\beta\) is a
transitive set, \(\alpha\) is a subset of \(\beta\); by
\Cref{thm:7cc2cckf}, \(\alpha \notin \alpha\); as
\(\alpha \in \beta\), we therefore have \(\alpha \neq \beta\); thus,
\(\alpha\) is a proper subset of \(\beta\).

Suppose conversely that \(\alpha\) is a proper subset of \(\beta\).
Then the proper subset \(x = \beta \setminus \alpha\) of \(\beta\) has
an element \(\gamma\) such that \(\gamma \cap x = \emptyset\).  We
will verify that \(\gamma = \alpha\).  The relation
\(\gamma \cap x = \emptyset\) implies that
\(\gamma \cap \beta \subset \alpha\).  As \(\gamma\) is an element of
the transitive set \(\beta\), \(\gamma \subset \beta\), so
\(\gamma \cap \beta = \gamma\).  Therefore, \(\gamma \subset \alpha\).
To check the reverse inclusion, let \(\delta\) be an element of
\(\alpha\).  As \(\alpha\) is a subset of \(\beta\),
\(\delta \in \beta\).  Thus, both \(\gamma\) and \(\delta\) belong to
\(\beta\).  If \(\gamma = \delta\), then as \(\delta \in \alpha\),
\(\gamma \in \alpha\), a contradiction because
\(\gamma \in x = \alpha \setminus \beta\); therefore,
\(\gamma \neq \delta\).  On the other hand, if \(\gamma \in \delta\),
then as \(\delta\) is an element of the transitive set \(\alpha\), we
have \(\gamma \in \alpha\), leading to the same contradiction as in
the previous case; therefore, \(\gamma \notin \delta\).  It follows
that \(\delta \in \gamma\).  Thus, \(\alpha \subset \gamma\), hence
\(\gamma = \alpha\).  As \(\gamma \in \beta\), this implies that
\(\alpha \in \beta\).

\begin{theorem}
  \label{thm:xut6v700}
  Any two ordinals \(\alpha\) and \(\beta\) satisfy exactly one of the
  conditions \(\alpha \in \beta\), \(\alpha = \beta\),
  \(\beta \in \alpha\).
\end{theorem}

Suppose \(\alpha \notin \beta\) and \(\alpha \neq \beta\).
\Cref{thm:lebsg6gf} implies that \(\alpha\) is not a proper subset of
\(\beta\) because \(\alpha \notin \beta\).  As \(\alpha \neq \beta\),
this means that \(\alpha\) is not a subset of \(\beta\).  Therefore,
the set \(x = \alpha \setminus \beta\) is a nonempty subset of the
ordinal \(\alpha\), and hence has an element \(\gamma\) such that
\(\gamma \cap x = \emptyset\).  Thus,
\(\gamma \cap \alpha \subset \beta\).  As \(\gamma \in \alpha\) and
\(\alpha\) is transitive, \(\gamma \subset \alpha\), so
\(\gamma \cap \alpha = \gamma\).  Therefore, \(\gamma \subset \beta\).
Now as \(\gamma \in x\), \(\gamma \notin \beta\).  By
\Cref{thm:lebsg6gf}, \(\gamma\) is not a proper subset of \(\beta\).
It follows that the subset \(\gamma\) of \(\beta\) equals \(\beta\).
Since \(\gamma \in \alpha\), we get \(\beta \in \alpha\).  So at least
one of the three stated conditions holds.

If \(\alpha \in \beta\), then as \(\alpha \notin \alpha\)
(\Cref{thm:7cc2cckf}), \(\alpha \neq \beta\); also
\(\beta \notin \alpha\), for if \(\beta \in \alpha\), then as
\(\alpha\) is transitive, the relations \(\alpha \in \beta\) and
\(\beta \in \alpha\) imply that \(\alpha \in \alpha\), a
contradiction.  Similarly, if \(\beta \in \alpha\), then
\(\alpha \notin \beta\) and \(\alpha \neq \beta\).  Lastly, if
\(\alpha = \beta\), then as \(\alpha \notin \alpha\), we have
\(\alpha \notin \beta\) and \(\beta \notin \alpha\).  Therefore, the
three stated conditions are mutually exclusive.

\begin{theorem}
  \label{thm:4ca41efr}
  Every nonempty set \(X\) of ordinals has a unique element \(\alpha\)
  such that \(\alpha \cap X = \emptyset\).  In fact, \(\alpha\) is the
  intersection of the elements of \(X\).
\end{theorem}

Let \(\xi\) be an element of the nonempty set \(X\).  If
\(\xi \cap X = \emptyset\), let \(\alpha = \xi\).  Suppose
\(\xi \cap X \neq \emptyset\).  Then as \(\xi\) is an ordinal, there
is an element \(\alpha\) of the nonempty subset \(\xi \cap X\) of
\(\xi\) such that \(\alpha \cap \xi \cap X = \emptyset\).  The
transitivity of \(\xi\) implies that its element \(\alpha\) is a
subset of it, so \(\alpha \cap \xi = \alpha\).  Therefore, \(\alpha\)
is an element of \(X\) such that \(\alpha \cap X = \emptyset\).  This
verifies that there is an element \(\alpha\) of \(X\) such that
\(\alpha \cap X = \emptyset\).

To show that \(\alpha\) is the intersection of the elements of \(X\),
it suffices to verify that it is a subset of every element \(\beta\)
of \(X\).  As \(\alpha \cap X = \emptyset\), \(\beta\) does not belong
to \(\alpha\).  Therefore, by \Cref{thm:xut6v700}, either
\(\alpha \in \beta\) or \(\alpha = \beta\).  According to
\Cref{thm:lebsg6gf}, \(\alpha \in \beta\) means that \(\alpha\) is a
proper subset of \(\beta\).  Therefore, \(\alpha\) is either a proper
subset of \(\beta\) or equals \(\beta\), that is, \(\alpha\) is a
subset of \(\beta\).  It follows that \(\alpha\) is the intersection
of the elements of \(X\).

\begin{theorem}
  \label{thm:22wi7nhu}
  A set is an ordinal if and only if is transitive and all its
  elements are ordinals.
\end{theorem}

If a set \(\alpha\) is an ordinal, then it is transitive by
definition, and all its elements are ordinals by \Cref{thm:5qmhh3to}.

Conversely, suppose \(\alpha\) is a transitive set all of whose
elements are ordinals.  \Cref{thm:xut6v700} implies that if
\(\beta, \gamma\) are distinct elements of \(X\), then either
\(\beta \in \gamma\) or \(\gamma \in \beta\).  Every nonempty subset
\(x\) of \(\alpha\), by \Cref{thm:4ca41efr}, has an element \(\beta\)
such that \(\beta \cap x = \emptyset\).  Therefore, \(\alpha\) is an
ordinal.

\begin{theorem}
  \label{thm:9s24zgf9}
  Let \(X\) be a set of ordinals.  Then the union of the elements of
  \(X\) is an ordinal.
\end{theorem}

Let \(\alpha\) denote the union of the elements of \(X\).  If
\(\beta\) is an element of \(\alpha\), then there is an element
\(\beta'\) of \(X\) such that \(\beta \in \beta'\); as \(\beta'\) is
an ordinal, it is transitive, so \(\beta\) is a subset of \(\beta'\);
by the definition of the union, \(\beta'\) is a subset of \(\alpha\);
therefore, \(\beta\) is a subset of \(\alpha\); thus, \(\alpha\) is a
transitive set.  Every element \(\beta\) of \(\alpha\) is an element
of some element \(\beta'\) of \(X\); as \(\beta'\) is an ordinal, by
\Cref{thm:5qmhh3to}, \(\beta\) is an ordinal too; thus, every element
of \(\alpha\) is an ordinal.  It follows from \Cref{thm:22wi7nhu} that
\(\alpha\) is an ordinal.

\begin{theorem}[Principle of transfinite induction]
  \label{thm:6ehprsvw}
  Let \(\alpha\) be an ordinal, and \(x\) a subset of \(\alpha\) with
  the property that for every element \(\beta\) of \(\alpha\) such
  that \(\beta\) is a subset of \(x\), we have \(\beta \in x\).  Then
  \(x\) equals \(\alpha\).
\end{theorem}

Let \(\leq\) denote the canonical order on \(\alpha\) as usual.  Thus,
the irreflexive kernel \(<\) of \(\leq\) is the same as the relation
\(\in\) on \(\alpha\).  This means that \(\beta = (-\infty, \beta)\)
for all \(\beta \in \alpha\).  Therefore, the hypothesis on \(x\) says
just that \(x\) is an inductive subset of \(\alpha\) with respect to
\(\leq\).  As \(\leq\) is a well-order on \(\alpha\)
(\Cref{thm:7cc2cckf}), by transfinite induction for well-ordered sets
(\Cref{thm:5vxgawq0}), \(x = \alpha\).

\begin{theorem}[Definition by transfinite recursion]
  \label{thm:fa9kx7s9}
  Let \(\alpha\) be an ordinal, and \(p : E \to \alpha\) a function
  from a set \(E\) to \(\alpha\).  Suppose that for every element
  \(\beta\) of \(\alpha\), \(f_\beta\) is a function from
  \(\Gamma(\beta, p)\) to \(p^{-1}(\beta)\).  Then there is a unique
  section \(u\) of \(p\) on \(\alpha\) such that
  \(u(\beta) = f_\beta(u \vert_\beta)\) for all \(\beta \in \alpha\).
\end{theorem}

The notations \(\Gamma(\beta, p)\) and \(u \vert_\beta\), for an
element \(\beta\) of \(\alpha\), in the statement of the theorem make
sense because \(\beta\) is a subset of \(\alpha\) as the ordinal
\(\alpha\) is a transitive set.  The statement itself follows from
transfinite recursion for well-ordered sets (\Cref{thm:srxrngd8}),
just as \Cref{thm:6ehprsvw} followed from transfinite induction for
well-ordered sets.

\begin{theorem}[Defintion by transfinite recursion]
  \label{thm:3y2h6l46}
  Let \(\alpha\) be an ordinal, and \(T\) any set.  Suppose that for
  every element \(\beta\) of \(\alpha\), \(f_\beta\) is a function
  from \(T^\beta\) to \(T\).  Then there exists a unique function
  \(u : \alpha \to T\) such that \(u(\beta) = f_\beta(u \vert_\beta)\)
  for all \(\beta \in \alpha\).
\end{theorem}

This theorem too follows immediately from the corresponding statement
for well-ordered sets (\Cref{thm:5kg5ewo1}).

An ordinal \(\alpha\) is called a \firstterm{successor} of an ordinal
\(\xi\) if \(\xi \in \alpha\) and there is no element \(\tau\) of
\(\alpha\) such that \(\xi \in \tau\); by \Cref{thm:lebsg6gf}, this is
equivalent to the condition that \(\xi\) is a proper subset of
\(\alpha\), and for any ordinal \(\tau\) such that
\(\xi \subset \tau \subset \alpha\), we have either \(\xi = \tau\) or
\(\tau = \alpha\).  An ordinal is called a \firstterm{successor
  ordinal} if it is a successor of some ordinal.  An ordinal is called
a \firstterm{limit ordinal} if it is not a successor ordinal.  It is
obvious that the empty set is a limit ordinal.

For any ordinal \(\xi\), let \(\xi^+\) denote the set
\(\xi \cup \{ \xi \}\); we have \(\xi \notin \xi\) by
\Cref{thm:7cc2cckf}, so \(\xi\) is a proper subset of \(\xi^+\).  For
any ordinal \(\alpha\), let \(\alpha^-\) denote the union of the
elements of \(\alpha\); as \(\alpha\) is a transitive set, every
element of \(\alpha\) is a subset of \(\alpha\), so \(\alpha^-\) is
also a subset of \(\alpha\).

\begin{theorem}
  \label{thm:8odofa55}
  \begin{enumerate}
  \item For any ordinal \(\xi\), the set \(\xi^+\) is the unique
    ordinal with the following properties:
    \begin{enumerate}
    \item \(\xi^+\) is a subset of every ordinal \(\nu\) such that
      \(\xi \in \nu\), and
    \item If \(\tau\) is an ordinal that is a subset of every ordinal
      \(\nu\) such that \(\xi \in \nu\), then \(\tau\) is a subset of
      \(\xi^+\).
    \end{enumerate}
  \item For any ordinal \(\alpha\), the set \(\alpha^-\) is the unique
    ordinal with the following properties:
    \begin{enumerate}
    \item Every element of \(\alpha\) is a subset of \(\alpha^-\), and
    \item If \(\tau\) is an ordinal such that every element of
      \(\alpha\) is a subset of \(\tau\), then \(\alpha^-\) is a
      subset of \(\tau\).
    \end{enumerate}
  \item For every ordinal \(\xi\), \(\xi^+\) is the unique successor
    of \(\xi\).  On the other hand, an ordinal \(\alpha\) is a
    successor ordinal if and only if \(\alpha^- \in \alpha\); in that
    case, \(\alpha^-\) is the unique ordinal which has \(\alpha\) as a
    successor. In particular, every ordinal has a unique successor,
    and is the successor of at most one ordinal.
  \item For all ordinals \(\xi\), we have \((\xi^+)^- = \xi\); for
    every successor ordinal \(\alpha\), we have
    \((\alpha^-)^+ = \alpha\).  Also, if \(\xi, \nu\) are ordinals,
    then \(\xi \in \nu\) if and only if \(\xi^+ \in \nu^+\);
    similarly, if \(\alpha, \beta\) are successor ordinals, then
    \(\alpha \in \beta\) if and only if \(\alpha^- \in \beta^-\).
  \item An ordinal \(\alpha\) is a limit ordinal if and only if
    \(\alpha^- = \alpha\).  In particular, a limit ordinal is either
    the empty set or an infinite set.
  \end{enumerate}
\end{theorem}

Let \(\xi\) be an ordinal.  If \(\lambda\) is an element of \(\xi\),
then either \(\lambda \in \xi\) or \(\lambda = \xi\); in the first
case, \(\lambda\) is a subset of \(\xi\) because \(\xi\) is a
transitive set, so \(\lambda\) is a subset of \(\xi^+\); in the second
case, \(\lambda \subset \xi^+\) because \(\xi\) is a subset of
\(\xi^+\); therefore, \(\xi^+\) is a transitive set.  By
\Cref{thm:5qmhh3to}, every element of \(\xi^+\) is an ordinal.
Therefore, by \Cref{thm:22wi7nhu}, \(\xi^+\) is an ordinal.  Suppose
\(\nu\) is an ordinal such that \(\xi \in \nu\); let
\(\sigma \in \xi^+\); then either \(\sigma \in \xi\) or
\(\sigma = \xi\); if \(\sigma \in \xi\), then as \(\xi \in \nu\) and
the ordinal \(\nu\) is a transitive set, \(\sigma \in \nu\); on the
other hand, if \(\sigma = \xi\), then \(\sigma \in \nu\) because
\(\xi \in \nu\); thus in both the cases, \(\sigma \in \nu\); it
follows that \(\xi^+\) is a subset of \(\nu\).  Let \(\tau\) be an
ordinal that is a subset of every ordinal \(\nu\) such that
\(\xi \in \nu\); then as \(\xi \in \xi^+\), \(\tau\) is a subset of
\(\xi^+\).  So \(\xi^+\) is an ordinal which has the two stated
properties.  If \(\alpha\) is any ordinal which has these properties
with \(\alpha\) in the place of \(\xi^+\), then it is obvious that
\(\xi^+\) and \(\alpha\) are subsets of each other, and are hence
equal.  So \(\xi^+\) is the unique ordinal which has these properties.
This verifies the first part of the theorem.

Let \(\alpha\) be an ordinal.  Then \(\alpha^-\) is the union of the
elements of \(\alpha\), so by \Cref{thm:5qmhh3to} and
\Cref{thm:9s24zgf9}, \(\alpha^-\) is an ordinal.  The definition of
\(\alpha^-\) implies that every element of \(\alpha\) is a subset of
\(\alpha^-\).  If \(\tau\) is an ordinal such that every element of
\(\alpha\) is a subset of \(\tau\), then it is obvious that the union
\(\alpha^-\) of the elements of \(\alpha\) is also a subset of
\(\tau\).  So \(\alpha^-\) has the two stated properties.  If \(\xi\)
is any ordinal which has these properties with \(\xi\) substituted for
\(\alpha^-\), then it is obvious that \(\alpha^-\) and \(\xi\) are
subsets of each other, and are hence equal.  Thus, \(\alpha^-\) is the
unique ordinal with these properties.  The second part of the theorem
is verified.

We will now show that an ordinal \(\alpha\) is a successor of an
ordinal \(\xi\) if and only if \(\xi^+ = \alpha\) if and only if
\(\xi = \alpha^-\) and \(\alpha^- \in \alpha\).  Suppose \(\alpha\) is
a successor of \(\xi\); then \(\xi \in \alpha\); by the first part of
the theorem, \(\xi^+\) is a subset of \(\alpha\); therefore, by
\Cref{thm:lebsg6gf}, either \(\xi^+ \in \alpha\) or
\(\xi^+ = \alpha\); if \(\xi^+ \in \alpha\), then we have
\(\xi \in \xi^+\) and \(\xi^+ \in \alpha\), which is a contradiction
because \(\alpha\) is a successor of \(\xi\); so \(\xi^+ = \alpha\).
Suppose \(\xi^+ = \alpha\); then as \(\xi \in \xi^+\),
\(\xi \in \alpha\); therefore, \(\xi\) is a subset of the union
\(\alpha^-\) of the elements of \(\alpha\); let \(\tau\) be any
element of \(\alpha\); as \(\xi^+ = \alpha\), either \(\tau \in \xi\)
or \(\tau = \xi\); by \Cref{thm:lebsg6gf}, this means that either
\(\tau\) is a proper subset of \(\xi\), or \(\tau = \xi\), that is,
\(\tau\) is a subset of \(\xi\); thus, every element of \(\alpha\) is
a subset of \(\xi\), so \(\alpha^-\) is a subset of \(\xi\);
therefore, \(\xi = \alpha^-\), and, as \(\xi \in \alpha\),
\(\alpha^- \in \alpha\).  Suppose \(\xi = \alpha^-\) and
\(\alpha^- \in \alpha\); then \(\xi \in \alpha\); suppose there is an
ordinal \(\tau\) such that \(\xi \in \tau\) and \(\tau \in \alpha\);
as \(\xi = \alpha^-\), \(\xi\) is the union of the elements of
\(\alpha\); therefore, as \(\tau \in \alpha\), \(\tau\) is a subset of
\(\xi\); by \Cref{thm:lebsg6gf}, either \(\tau \in \xi\) or
\(\tau = \xi\); if \(\tau \in \xi\), then as \(\xi \in \tau\) and
\(\tau\) is transitive, we get \(\tau \in \tau\); on the other hand,
if \(\tau = \xi\), then as \(\xi \in \tau\), we again have
\(\tau \in \tau\); thus, in both the cases we have the relation
\(\tau \in \tau\) which contradicts \Cref{thm:7cc2cckf}; it follows
that there is no ordinal \(\tau\) such that \(\xi \in \tau\) and
\(\tau \in \alpha\); therefore, \(\alpha\) is a successor of \(\xi\).
This verifies the equivalences stated at the beginning of this
paragraph.

For any ordinal \(\xi\), the equivalence
\begin{displaymath}
  \alpha ~ \text{is a successor of} ~ \xi
  \Leftrightarrow
  \xi^+ = \alpha,
\end{displaymath}
valid for any ordinal \(\alpha\), says that \(\alpha^+\) is the unique
successor of \(\xi\).  Similarly, for any ordinal \(\alpha\), the
equivalence
\begin{displaymath}
  \alpha ~ \text{is a successor of} ~ \xi
  \Leftrightarrow
  \xi = \alpha^- ~ \text{and} ~ \alpha^- \in \alpha,
\end{displaymath}
valid for any ordinal \(\xi\), says that \(\alpha\) is a successor
ordinal if and only if \(\alpha^- \in \alpha\), and in that case
\(\alpha^-\) is the unique ordinal which has \(\alpha\) as a
successor.  This checks the third part of the theorem.

If \(\xi\) is any ordinal, then \(\alpha = \xi^+\) is a successor
ordinal because it is the successor of \(\xi\), so \(\alpha^-\) is the
unique ordinal which has \(\alpha\) as its successor, hence
\(\alpha^- = \xi\), that is, \((\xi^+)^- = \xi\).  On the other hand,
if \(\alpha\) is any successor ordinal, then it is the successor of
\(\xi = \alpha^-\), so by the uniqueness of successors,
\(\xi^+ = \alpha\), that is, \((\alpha^-)^+ = \alpha\).  If
\(\xi, \nu\) are ordinals such that \(\xi \in \nu\), then by the first
part of the theorem, \(\xi^+\) is a subset of \(\nu\); as \(\nu\) is a
proper subset of \(\nu^+\), it follows that \(\xi^+\) is a proper
subset of \(\nu^+\); by \Cref{thm:lebsg6gf}, \(\xi^+ \in \nu^+\).
Conversely, if \(\xi^+ \in \nu^+\), then \(\nu \notin \xi\) because by
the previous sentence, if \(\nu \in \xi\), then \(\nu^+ \in \xi^+\),
so by the transitivity of \(\nu^+\), \(\nu^+ \in \nu^+\),
contradicting \Cref{thm:7cc2cckf}; also, \(\xi \neq \nu\), for if
\(\xi = \nu\), then \(\xi^+ = \nu^+\), so we again get the
contradiction \(\xi^+ \in \xi^+\); therefore, by \Cref{thm:xut6v700},
\(\xi \in \nu\).  Thus, \(\xi \in \nu\) if and only if
\(\xi^+ \in \nu^+\).  Similarly, if \(\alpha, \beta\) are successor
ordinals, then \(\alpha = (\alpha^-)^+\) and \(\beta = (\beta^-)^+\),
so by the previous sentence, \(\alpha \in \beta\) if and only if
\(\alpha^- \in \beta^-\).  The fourth part of the theorem follows.

By the third part of of the theorem, and \Cref{thm:lebsg6gf}, an
ordinal \(\alpha\) is a successor ordinal if and only if \(\alpha^-\)
is a proper subset of \(\alpha\).  As \(\alpha^-\) is a subset of
\(\alpha\), this means hat \(\alpha\) is a limit ordinal if and only
if \(\alpha^- = \alpha\).  If \(\alpha\) is a nonempty finite ordinal,
then as the canonical order \(\leq\) on \(\alpha\) is total,
\(\alpha\) has a maximum \(\beta\) with respect to \(\leq\); every
element of \(\alpha\) is contained in \(\beta\) by
\Cref{thm:86se6zr0}, so \(\beta\) is the union of the elements of
\(\alpha\), that is, \(\beta = \alpha^-\); so \(\alpha^- \in \alpha\),
hence \(\alpha\) is a successor ordinal.  Thus, every nonempty finite
ordinal is a successor ordinal.  It follows that a limit ordinal is
either empty or infinite.  This verifies the last part of the theorem.

\subsection{Exercises}
\label{sec:q1c2w2jy}

\begin{exercise}
  \label{exe:swjbrepg}
  Verify that a maximal element of a directed set is its maximum, and
  a minimal element of a codirected set is its minimum.
\end{exercise}

\begin{exercise}
  \label{exe:yowl3axc}
  Show that a product of lattices is a lattice.
\end{exercise}

\begin{exercise}
  \label{exe:4knqif5b}
  Prove that an order is total if and only if it has the trichotomy
  property.
\end{exercise}

\begin{exercise}
  \label{exe:v1tw1ijd}
  Show that every strictly increasing surjection from a totally
  ordered set \(X\) onto any ordered set \(Y\) is an isomorphism, and
  that any increasing bijection from \(X\) onto \(Y\) is an
  isomorphism.
\end{exercise}

\begin{exercise}
  \label{exe:gsdhavsp}
  Let \(X, Y\) be two well-ordered sets, and \(f, g\) two increasing
  functions from \(X\) to \(Y\) such that \(f(X)\) is a segment of
  \(Y\) and \(g\) is strictly increasing.  Show that
  \(f(x) \leq g(x)\) for all \(x \in X\).
\end{exercise}

\begin{exercise}
  \label{exe:rpm18mof}
  This exercise is about a simple topology on an ordered set.
  \begin{enumerate}
  \item \label{item:xzh0qsyg} Check that the union and the
    intersection of a family of segments of an ordered set \(X\) are
    segments, as are the empty set and \(X\).  Thus, the segments of
    \(X\) form a topology on \(X\) in which any intersection of open
    sets is open.
  \item \label{item:ef4d89pd} Suppose \(X\) is a well-ordered set, and
    let \(Y\) be any ordered set.  For any segment \(U\) of \(X\), let
    \(F(U)\) be the set of strictly increasing functions from \(U\) to
    \(Y\).  Show that this defines a subsheaf \(F\) of the sheaf of
    \(Y\)-valued functions on the topological space \(X\) defined in
    (\ref{item:xzh0qsyg}).
  \item With \(X\) and \(Y\) as in (\ref{item:ef4d89pd}), for any
    segment \(U\) of \(X\), let \(F(U)\) be the set of functions
    \(f : U \to Y\) such that the subset \(f(U)\) of \(Y\) is totally
    ordered.  Verify that this also defines a subsheaf \(F\) of the
    sheaf of \(Y\)-valued functions on \(X\).
  \end{enumerate}
\end{exercise}

\begin{exercise}
  \label{exe:oh59gz4i}
  Prove the following reformulation of transfinite recursion
  (\Cref{thm:srxrngd8}).  Let \(X\) be a well-ordered set, and
  \((E_x)_{x \in X}\) a family of sets indexed by \(X\).  Suppose that
  for every \(x \in X\), \(f_x\) is a function from the product set
  \(\prod_{y \in (-\infty, x)} E_y\) to \(E_x\).  Then there is a
  unique element \(e\) of \(\prod_{x \in X} E_x\) such that
  \(e_x = f_x((e_y)_{y \in (-\infty, x)})\) for all \(x \in X\).
\end{exercise}

\begin{exercise}
  \label{exe:eghnwow2}
  Here is a proof of \Cref{thm:vy95iuyi} using transfinite recursion.
  Suppose \(X\) and \(Y\) are well-ordered sets such that \(Y\) is not
  isomorphic to any segment of \(X\).  Use \Cref{thm:5kg5ewo1} to
  get a function \(u : X \to Y\) such that \(u(x)\) is the minimum of
  \(Y \setminus u((-\infty, x))\) for all \(x \in X\).  Prove that
  \(u\) is an isomorphism from \(X\) onto a segment of \(Y\).
\end{exercise}

\begin{exercise}
  \label{exe:vgms7ihx}
  Here is another proof, from \textcite[\S2.5,
  Theorem~3]{bib:lmhdqwpw}, of \Cref{thm:vy95iuyi}, which uses Zorn's
  lemma.  Let \(X\) and \(Y\) be well-ordered sets.  Consider the set
  \(P\) of all pairs \((A,f)\) consisting of a segment \(A\) of \(X\)
  and a function \(f : A \to Y\) which is an isomorphism onto a
  segment of \(Y\).  Define an order on \(P\) by setting
  \((A,f) \leq (A',f')\) if \(A \subset A'\) and \(f = f' \vert_A\).
  Apply Zorn's lemma to \(P\) to show that either \(X\) is isomorphic
  to a segment of \(Y\), or \(Y\) is isomorphic to a segment of \(X\).
\end{exercise}

\begin{exercise}
  \label{exe:9d868h6m}
  This exercise gives a proof of Zorn's lemma using Zermelo's theorem
  and transfinite recursion \parencite[Chapter~II,
  Theorem~9.4]{bib:h8xsxpp7}.  Let X be an ordered set in which every
  totally ordered subset is bounded above.
  \begin{enumerate}
  \item \label{item:ebwyq3lr} Show that if \(M\) is a maximal totally
    ordered subset of \(X\), that is, a totally ordered subset of
    \(X\) which is not properly contained in another totally ordered
    subset of \(X\), then any upper bound of \(M\) is a maximal
    element of \(X\).
  \item Zermelo's theorem gives a well-ordered set \(\Lambda\) and a
    bijection \(h\) from the set \(\Lambda\) to the set \(X\).  Use
    transfinite recursion (\Cref{thm:5kg5ewo1}) to get a family
    \((M_\lambda)_{\lambda \in \Lambda}\) of totally ordered subsets
    of \(X\) such that, for any \(\lambda \in \Lambda\), \(M_\lambda\)
    equals \((\cup_{\mu < \lambda} M_\mu) \cup \{ h(\lambda) \}\) if
    this set is a totally ordered subset of \(X\), and is empty
    otherwise.  Prove that the union of the \(M_\lambda\) is a maximal
    totally ordered subset of \(X\).
  \end{enumerate}
\end{exercise}

\begin{exercise}
  \label{exe:5qa65h5w}
  Here is another proof of Zorn's lemma using Zermelo's theorem and
  transfinite recursion \parencite[Part~II, Chapter~14,
  Solution~1]{bib:uw2wsr3k}.  Let X be an ordered set in which every
  well-ordered subset is bounded above.  Zermelo's theorem gives a
  well-ordered set \(\Lambda\) and a bijection \(h\) from the set
  \(\Lambda\) to the set \(X\).  Use transfinite recursion to define a
  subset \(M\) of \(X\) such that any element \(x\) of \(X\) belongs
  to \(M\) if and only if \(w < x\) for all \(w \in M\) such that
  \(h^{-1}(w) < h^{-1}(x)\).  Prove that \(M\) is totally ordered, and
  that any upper bound of \(M\) is a maximal element of \(X\).
\end{exercise}

\begin{exercise}
  \label{exe:x9ewqbv6}
  Prove the following generalisation of transfinite recursion
  (\Cref{thm:srxrngd8}).  Let \(X\) be a well-ordered set.  Make the
  set \(X\) a topological space by giving it the topology whose open
  sets are the segments of \(X\) (\Cref{exe:rpm18mof}).  Let \(F\) be
  a sheaf of sets on \(X\).  Suppose that for every point \(x \in X\),
  \(f_x\) is a function from \(F((-\infty, x))\) to
  \(F((-\infty, x])\) which is a section of the restriction map:
  \(f_x(s) \vert_{(-\infty, x)} = s\) for all
  \(s \in F((-\infty, x))\).  Show that there exists a unique element
  \(u \in F(X)\) such that
  \(u \vert_{(-\infty, x]} = f_x(u \vert_{(-\infty, x)})\) for all
  \(x \in X\).
\end{exercise}

\begin{exercise}
  \label{exe:k8tmok20}
  A final proof of Zorn's lemma using Zermelo's theorem and
  transfinite recursion, this time from \textcite[Chapter~XVI, \S4,
  Theorem~1]{bib:embpmrus}.  Let \(X\) be an ordered set in which
  every well-ordered subset is bounded above.  Suppose that \(X\) has
  no maximal element.  Arrive at a contradiction as follows.
  \begin{enumerate}
  \item Verify that every well-ordered subset \(A\) of \(X\) has a
    strict upper bound, that is, an element \(b \in X\) such that
    \(a < b\) for all \(a \in A\).
  \item Zermelo's theorem gives a well-ordered set \(\Lambda\) and a
    bijection \(h\) from the set \(\Lambda\) onto the set \(X\).  Use
    transfinite recursion (\Cref{exe:x9ewqbv6}) to get a strictly
    increasing function \(u\) from \(\Lambda\) to \(X\) with the the
    property that \(h^{-1}(u(\lambda)) \leq h^{-1}(b)\) for all
    \(\lambda \in \Lambda\) and for every strict upper bound \(b\) of
    \(u((-\infty, \lambda))\) in \(X\).
  \item Verify that \(h^{-1} \circ u\) is a strictly increasing
    function from \(\Lambda\) to \(\Lambda\).
  \item Show that \(u(\Lambda)\) is a well-ordered subset of \(X\)
    which does not have a strict upper bound.
  \end{enumerate}
\end{exercise}

\begin{exercise}[{\textcite[Chapter~4, \S2]{bib:lsf6zq1g}}]
  \label{exe:8yw1mw3a}
  Let \(X\) be an ordered set, \(a\) an element of \(X\), and \(f\) a
  function from \(X\) to itself.  A subset \(Y\) of \(X\) is called
  \firstterm{superinductive} for the pair \((a, f)\) if it satisfies
  the following conditions:
  \begin{enumerate}
  \item \(a \in Y\).
  \item \(f(Y) \subset Y\).
  \item If \(Z\) is a nonempty totally ordered subset of \(Y\), and
    \(b\) the supremum of \(Z\) in \(X\), then \(b \in Y\).
  \end{enumerate}
  A subset \(A\) of \(X\) is called \firstterm{minimally
    superinductive} for \((a, f)\) if it is superinductive for
  \((a, f)\), and no proper subset of \(A\) is superinductive for
  \((a, f)\).  Show that \(X\) has a unique subset \(A\) that is
  minimally superinductive for \((a, f)\).  Verify also that if
  \(x \leq f(x)\) for all \(x \in X\), then \(a\) is the minimum of
  \(A\).
\end{exercise}

\begin{exercise}[{\textcite[Chapter~4, Theorem~2.1]{bib:lsf6zq1g}}]
  \label{exe:9l6q0c66}
  Let \(X\) be an ordered set, \(a\) an element of \(X\), and \(f\) a
  function from \(X\) to itself.  Suppose that \(X\) is minimally
  superinductive for \((a, f)\).  Let \(R\) be a subset of
  \(X \times X\) with the following properties:
  \begin{enumerate}
  \item \label{item:u0r5igra} For all \(x \in X\), \((x, a) \in R\).
  \item \label{item:3jc58w0r} If \(x, y \in X\) are such that
    \((x, y)\) and \((y, x)\) belong to \(R\), then \((x, f(y))\)
    belongs to \(R\).
  \item \label{item:fh1pwe19} If \(x\) is an element of \(X\), \(Y\) a
    nonempty totally ordered subset of \(X\), \(b\) the supremum of
    \(Y\) in \(X\), and \((x, y) \in R\) for all \(y \in Y\), then
    \((x, b) \in R\).
  \end{enumerate}
  Prove that if \(x \in X\) is such that \((y, x) \in R\) for all
  \(y \in X\), then \((x, y) \in R\) for all \(y \in R\).  Deduce from
  this statement that \(R = X \times X\).
\end{exercise}

\begin{exercise}[{\textcite[Th\'eor\`eme~1]{bib:je0dhslq}}]
  \label{exe:yf8euudc}
  Let \(X\) be an ordered set, \(a\) an element of \(X\), and \(f\) a
  function from \(X\) to itself.  Suppose that \(x \leq f(x)\) for all
  \(x \in X\), and that \(X\) is minimally superinductive for
  \((a, f)\).  Prove that for all \(x,y \in X\), either
  \(f(x) \leq y\) or \(y \leq x\); in particular, \(X\) is totally
  ordered.
\end{exercise}

\begin{exercise}
  \label{exe:htz5ftfy}
  Let \(X\) be an ordered set, \(a\) an element of \(X\), and \(f\) a
  function from \(X\) to itself.  Suppose that \(x \leq f(x)\) for all
  \(x \in X\), and that \(X\) is minimally superinductive for
  \((a, f)\).  Let \(x, y \in X\).  Prove the following statements:
  \begin{enumerate}
  \item If \(x \leq y \leq f(x)\), then either \(x = y\) or
    \(y = f(x)\).
  \item If \(x < y\), then \(f(x) \leq y\).
  \item The function \(f : X \to X\) is increasing.
  \end{enumerate}
\end{exercise}

\begin{exercise}[{\textcite[Corollaire~1 of Th\'eor\`eme~1]{bib:je0dhslq}}]
  \label{exe:ycuba4tr}
  Let \(X\) be an ordered set, \(a\) an element of \(X\), and \(f\) a
  function from \(X\) to itself.  Suppose that \(x \leq f(x)\) for all
  \(x \in X\).  Let \(A\) be the unique subset of \(X\) that is
  minimally superinductive for \((a, f)\) (\Cref{exe:8yw1mw3a}), and
  let \(b \in X\).  Prove the equivalence of the following statements:
  \begin{enumerate}
  \item \label{item:31iv9t84} \(b \in A\) and \(f(b) = b\).
  \item \label{item:zg0eno8k} \(b\) is the maximum of \(A\).
  \item \label{item:zhfu99x4} \(b\) is the supremum of \(A\).
  \end{enumerate}
  In particular, \(f\) has at most one fixed point in \(A\).
\end{exercise}

\begin{exercise}[Bourbaki-Witt fixed point theorem]
  \label{exe:al7fffh1}
  Let \(X\) be an ordered set, and \(f\) a function from \(X\) to
  itself.  Suppose that every totally ordered subset of \(X\) has a
  supremum, and that \(x \leq f(x)\) for all \(x \in X\).  Prove that
  for every \(x \in X\), there exists an element \(b\) of \(X\) such
  that \(x \leq b\) and \(f(b) = b\); in particular, \(f\) has a fixed
  point in \(X\).
\end{exercise}

\begin{exercise}[{\textcite[Corollaire~2 of
    Th\'eor\`eme~1]{bib:je0dhslq}}]
  \label{exe:313bq61a}
  Let \(X\) be an ordered set, \(a\) an element of \(X\), and \(f\) a
  function from \(X\) to itself.  Suppose that \(x \leq f(x)\) for all
  \(x \in X\), and that \(X\) is minimally superinductive for
  \((a, f)\).
  \begin{enumerate}
  \item Show that if a subset \(Y\) of \(X\) does not have an infimum
    in \(X\), then \(Y\) is empty.
  \item Prove that \(X\) is well-ordered.
  \end{enumerate}
\end{exercise}

\begin{exercise}[{\textcite[p.~437]{bib:je0dhslq}}]
  \label{exe:m0vdltb2}
  This exercise gives a derivation of Zermelo's theorem from the axiom
  of choice using the previous few exercises on superinductivity.  Let
  \(X\) be an arbitrary set.  Denote by \(\mathcal{P}\) the power set
  of \(X\), and by \(\mathcal{P}'\) the set of elements of
  \(\mathcal{P}\) that are distinct from \(X\).  The axiom of choice
  gives a function \(g : \mathcal{P}' \to X\) with the property that
  \(g(Y) \in X \setminus Y\) for every \(Y \in \mathcal{P}'\).  Order
  \(\mathcal{P}\) by inclusion: for all \(Y, Z \in \mathcal{P}\),
  \(Y \leq Z\) if \(Y\) is a subset of \(Z\).  Define a function
  \(f : \mathcal{P} \to \mathcal{P}\) by setting
  \begin{displaymath}
    f(Y) =
    \begin{cases}
      Y \cup \{ g(Y) \} & \text{if} ~ Y \neq X \\
      X & \text{if} ~ Y = X
    \end{cases}
  \end{displaymath}
  for all \(Y \in \mathcal{P}\).  Thus, \(Y \leq f(Y)\) for all
  \(Y \in \mathcal{P}\).  Let \(\mathcal{A}\) be the unique subset of
  \(\mathcal{P}\) that is minimally superinductive for the pair
  \((\emptyset, f)\) (\Cref{exe:8yw1mw3a}).
  \begin{enumerate}
  \item Verify that \(X \in \mathcal{A}\).
  \item Let \(\mathcal{A}'\) denote the set of elements of
    \(\mathcal{A}\) that are distinct from \(X\).  Show that the
    function \(g : \mathcal{P}' \to X\) restricts to a bijection from
    \(\mathcal{A}'\) onto \(X\).  Deduce from this statement that
    there is a well-order on the set \(X\).
  \end{enumerate}
\end{exercise}

\begin{exercise}
  \label{exe:rw9lhmv9}
  Let \(X\) be a set, and \((X_i)_{i \in I}\) a partition of \(X\).
  Denote by \(p\) the function from \(X\) to \(I\) which maps any
  element \(x\) of \(X\) to the unique index \(i \in I\) such that
  \(x \in X_i\).  Suppose that we are given an order on each of the
  sets \(X_i\), and on the index set \(I\).
  \begin{enumerate}
  \item Show that there is a unique order on \(X\) which induces the
    given order on each of the \(X_i\), and satisfies the condition
    that \(x \leq y\) for all \(x, y \in X\) such that
    \(p(x) < p(y)\).
  \item If the orders on the \(X_i\) and \(I\) are well-orders, verify
    that the above order on \(X\) is also a well-order.
  \end{enumerate}
\end{exercise}

\begin{exercise}
  \label{exe:mfln3otu}
  Let \(X\) be a set.  Let \(S\) be the set of pairs \((Y, \leq)\)
  consisting of a subset \(Y\) of \(X\) and a well-order \(\leq\) on
  \(Y\).  For any two elements \(s = (Y, \leq)\) and
  \(s' = (Y', \leq')\) of \(S\), define \(s \sqsubseteq s'\) if \(Y\)
  is a subset of \(Y'\), \(\leq\) is the order that is induced on
  \(Y\) by \(\leq'\), and \(Y\) is a segment of \(Y'\) with respect to
  \(\leq'\).
  \begin{enumerate}
  \item Check that \(\sqsubseteq\) is an order on \(S\).
  \item Let \(T = (s_i)_{i \in I}\) be a family of elements of \(S\)
    which is totally ordered with respect to \(\sqsubseteq\): for any
    two indices \(i, j \in I\), either \(s_i \sqsubseteq s_j\) or
    \(s_j \sqsubseteq s_i\).  Write \(s_i = (Y_i, \leq_i)\) for every
    \(i\).  Let \(Y\) be the union of the sets \(Y_i\) as \(i\) runs
    over \(I\).
    \begin{enumerate}
    \item Verify that there is a unique order \(\leq\) on \(Y\) which
      induces \(\leq_i\) on each \(Y_i\).
    \item Prove that \(\leq\) is a well-order on \(Y\).
    \item Show that the element \(s = (Y, \leq)\) of \(S\) is a
      supremum of the family \(T\): \(s_i \sqsubseteq s\) for all
      \(i\), and if \(s'\) is another element of \(S\) such that
      \(s_i \sqsubseteq s'\) for all \(i\), then \(s \sqsubseteq s'\).
    \end{enumerate}
  \end{enumerate}
\end{exercise}

\begin{exercise}
  \label{exe:fgbbb6v2}
  Use the Bourbaki-Witt fixed point theorem (\Cref{exe:al7fffh1}) to
  derive Zermelo's theorem from the axiom of choice as follows.  Let
  \(X\) be a set, and let \(S\) be the ordered set of
  \Cref{exe:mfln3otu}.  Apply the Bourbaki-Witt fixed point theorem to
  a suitable function on \(S\) to deduce that there is a well-order on
  \(X\).
\end{exercise}

\begin{exercise}
  \label{exe:64qxlchj}
  Let \(X\) be an ordered set.  Denote by \(\mathcal{S}\) the set of
  totally ordered subsets of \(X\).  Order \(\mathcal{S}\) by
  inclusion: for \(Y, Z \in \mathcal{S}\), \(Y \leq Z\) if \(Y\) is a
  subset of \(Z\).  Show that every totally ordered subset of
  \(\mathcal{S}\) has a supremum in \(\mathcal{S}\).
\end{exercise}

\begin{exercise}
  \label{exe:h4ww3mr2}
  This exercise shows how to deduce Zorn's lemma from the axiom of
  choice using the Bourbaki-Witt fixed point theorem
  (\Cref{exe:al7fffh1}).  Let \(X\) be an ordered set in which every
  totally ordered subset is bounded above.  Apply the Bourbaki-Witt
  fixed point theorem to an appropriate function on the ordered set
  \(\mathcal{S}\) of \Cref{exe:64qxlchj} to deduce that \(X\) has a
  maximal element.
\end{exercise}

\begin{exercise}
  \label{exe:51daga9q}
  Here is another proof of Zorn's lemma from the axiom of choice using
  the Bourbaki-Witt fixed point theorem (\Cref{exe:al7fffh1}).  Let
  \(X\) be an ordered set.  Denote by \(\mathcal{S}\) the set of
  well-ordered subsets of \(X\).  For \(Y, Z \in \mathcal{S}\), define
  \(Y \sqsubseteq Z\) if \(Y\) is a segment of \(Z\).
  \begin{enumerate}
  \item \label{item:j6hjwyxg} Show that \(\sqsubseteq\) is an order on
    \(\mathcal{S}\), and that every totally ordered subset of
    \(\mathcal{S}\) has a supremum in \(\mathcal{S}\).
  \item \label{item:41f56yw7} Check that if \(M\) is a maximal element
    of \(\mathcal{S}\), then any upper bound of \(M\) is a maximal
    element of \(X\).
  \item Suppose that every well-ordered subset of \(X\) is bounded
    above.  Apply the Bourbaki-Witt fixed point theorem to a suitable
    function on \(\mathcal{S}\) to deduce that \(X\) has a maximal
    element.
  \end{enumerate}
\end{exercise}

\begin{exercise}
  \label{exe:ylp1zcc8}
  Let \(X\) be a totally ordered set, \(w\) an element of \(X\) that
  has a successor, and \(Y\) a subset of \(X\) such that
  \(w^+ \in Y\).  Show that \(w^+\) is the minimum of the set of
  elements \(t\) of \(Y\) such that \(w < t\).
\end{exercise}

\begin{exercise}
  \label{exe:ddc4bdfk}
  Let \(\xi\) be an ordinal, and \(X\) a set of ordinals such that
  \(\xi^+ \in X\).  Prove that \(\xi^+\) is the intersection of the
  elements \(\tau\) of \(X\) such that \(\xi \in \tau\).
\end{exercise}

\begin{exercise}[Burali-Forti paradox]
  \label{exe:9pk4eh7k}
  Prove that there is no set \(X\) such that every ordinal is an
  element of \(X\).
\end{exercise}

\begin{exercise}
  \label{exe:7tvca54w}
  Prove the following statements for any two ordinals \(\alpha\) and
  \(\beta\).
  \begin{enumerate}
  \item If \(\alpha^- \in \beta^-\), then \(\alpha \in \beta\).
  \item If \(\alpha \in \beta\), then either \(\alpha^- \in \beta^-\),
    or \(\alpha\) is a limit ordinal and \(\beta = \alpha^+\).
  \end{enumerate}
\end{exercise}

\subsection{Solutions}
\label{sec:fzhm4l61}

\begin{solution}[\ref{exe:swjbrepg}]
  \label{sol:pcqde5hb}
  If \(a\) is a maximal element of a directed set \(X\), then for any
  \(x \in X\), the set \(\{ a, x \}\) has an upper bound \(y\).  As
  \(a\) is maximal, the relation \(a \leq y\) implies that \(a = y\);
  and as \(x \leq y\), it follows that \(x \leq a\).  Thus, \(a\) is a
  maximum of \(X\).  The second statement of the exercise follows from
  the first because the opposite of a directed order is codirected.
\end{solution}

\begin{solution}[\ref{exe:yowl3axc}]
  \label{sol:rfwhck5v}
  Let \(X\) be the product of a family \((X_i)_{i \in I}\) of
  lattices.  Suppose \(x = (x_i)\) and \(y = (y_i)\) are two elements
  of \(X\).  As each \(X_i\) is a lattice, the set
  \(Y_i = \{ x_i, y_i \}\) has a supremum \(a_i\) and an infimum
  \(b_i\) in \(X_i\).  It is obvious that the element \(a = (a_i)\) of
  \(X\) is an upper bound of the set \(Y = \{ x, y \}\); if
  \(z = (z_i)\) is any upper bound of \(Y\), then \(z_i\) is an upper
  bound of \(Y_i\), hence \(a_i \leq z_i\) for all \(i\); thus,
  \(a \leq z\); it follows that \(a\) is the supremum of \(Y\).
  Similarly, \((b_i)\) is the infimum of \(Y\).
\end{solution}

\begin{solution}[\ref{exe:4knqif5b}]
  \label{sol:ien6orgc}
  An order with the trichotomy property is obviously total.
  Conversely, if an order \(\leq\) on on a set \(X\) is total, and if
  \(x,y \in X\) are such that \(x \nless y\) and \(x \neq y\), then
  \(x \nleq y\), hence \(x \geq y\); as \(x \neq y\), \(x > y\); thus,
  at least one of the three alternatives \(x < y\), \(x = y\),
  \(x > y\) is valid; it is obvious that \(x = y\) rules out the other
  two alternatives; if \(x < y\), then \(x \neq y\), and by
  anti-symmetry \(x \ngtr y\); similarly, if \(x > y\), then
  \(x \neq y\) and \(x \nless y\); thus, \(\leq\) has the trichotomy
  property.
\end{solution}

\begin{solution}[\ref{exe:v1tw1ijd}]
  \label{sol:rd6deawb}
  If \(f : X \to Y\) is a strictly increasing surjection, and if
  \(x, x' \in X\) and \(f(x) = f(x')\), then \(x \nless x'\) and
  \(x \ngtr x'\); so the trichotomy of the order on \(X\) implies that
  \(x = x'\); thus, \(f\) is injective, and hence a bijection; if
  \(y, y' \in Y\) and \(y \leq y'\), then the anti-symmetry of the
  order on \(Y\), and the strictly increasing property of \(f\) imply
  that \(f^{-1}(y) \ngtr f^{-1}(y')\); therefore, again by trichotomy,
  \(f^{-1}(y) \leq f^{-1}(y')\); thus, \(f\) is an isomorphism.  An
  increasing bijection from \(X\) to \(Y\) is strictly increasing
  because it is injective, so it is an isomorphism by the first
  statement.
\end{solution}

\begin{solution}[\ref{exe:gsdhavsp}]
  \label{sol:ja6mjzt3}
  If the set \(M\) of \(x \in X\) such that \(f(x) > g(x)\) is
  non-empty, it has a minimum \(a\).  The segment \(f(X)\) of \(Y\)
  contains \(f(a)\), hence it contains the interval
  \((-\infty, f(a)]\).  Therefore, the element \(g(a)\) of this
  interval belongs to \(f(X)\): \(g(a) = f(z)\) for some \(z \in X\).
  As \(f\) is increasing and \(f(z) < f(a)\), trichotomy gives
  \(z < a\), so \(z \notin M\): \(f(z) \leq g(z)\).  The hypothesis
  that \(g\) is strictly increasing and the relation \(z < a\) imply
  that \(g(z) < g(a)\).  It follows that \(f(z) < g(a)\), a
  contradiction because \(f(z) = g(a)\).
\end{solution}

\begin{solution}[\ref{exe:rpm18mof}]
  \label{sol:aeu4zzog}
  If \(U\) is either the empty set or \(X\), it is a segment because
  it satisfies the condition that \((-\infty, a] \subset U\) for every
  \(a \in U\) vacuously in the first case and by the definition of
  containment in the second case.  If \((U_i)_{i \in I}\) is a family
  of segments of \(X\), and \(y\) an element of their union \(U\),
  then there is an \(i \in I\) such that \(y \in U_i\); as \(U_i\) is
  a segment, it contains \((-\infty, y]\), hence so does its superset
  \(U\).  Similarly, if \(y\) belongs to the intersection \(V\) of the
  \(U_i\), then for every \(i\), the segment \(U_i\) contains \(y\),
  hence it contains \((-\infty, y]\); it follows that \(V\) contains
  \((-\infty, y]\).

  Suppose \(X\) is a well-ordered set, and let \(Y\) be any ordered
  set.  For any segment \(U\) of \(X\), let \(F(U)\) be the set of
  strictly increasing functions from \(U\) to \(Y\).  Let
  \((U_i)_{i \in I}\) be a family of segments of \(X\), \(U\) their
  union, and \(f : U \to Y\) any function.  If \(f \in F(U)\), then it
  is obvious that its restriction to any subset of \(U\) is also
  strictly increasing, so \(f \vert_{U_i} \in F(U_i)\) for all \(i\).
  Conversely, if \(f \vert_{U_i} \in F(U_i)\) for every \(i\), and if
  \(x, x' \in U\) are such that \(x < x'\), then there is an index
  \(i \in I\) such that \(x' \in U_i\); as \(U_i\) is a segment,
  \(x \in U_i\), and as \(f \vert_{U_i}\) is strictly increasing,
  \(f(x) < f(x')\); therefore, \(f \in F(U)\).  It follows that \(F\)
  is a subsheaf of the sheaf of \(Y\)-valued functions on the
  topological space \(X\).

  Now suppose that \(X\) and \(Y\) are as above, and let \(F(U)\) be
  the set of all functions \(f : U \to Y\) such that \(f(U)\) is a
  totally ordered subset of \(Y\).  Let \((U_i)_{i \in I}\) be a
  family of segments of \(X\), \(U\) their union, and \(f : U \to Y\)
  any function.  If \(f \in F(U)\), then for every \(i\), \(f(U_i)\)
  is a subset of the totally ordered subset \(f(U)\) of \(Y\), so
  \(f(U_i)\) is also totally ordered, and
  \(f \vert_{U_i} \in F(U_i)\).  Conversely, if
  \(f \vert_{U_i} \in F(U_i)\) for all \(i\), and if
  \(y, y' \in F(U)\), then there are \(x,x' \in U\) such that
  \(f(x) = y\) and \(f(x') = y'\); as \(X\) is totally ordered, we can
  assume that \(x \leq x'\); let \(i \in I\) be such that
  \(x' \in U_i\); as \(U_i\) is a segment, \(x \in U_i\), so
  \(y, y' \in F(U_i)\); as \(F(U_i)\) is totally ordered, either
  \(y \leq y'\) or \(y' \leq y\); therefore, \(f(U)\) is totally
  ordered, and \(f \in F(U)\).  It follows that \(F\) is a subsheaf of
  the sheaf of \(Y\)-valued functions on \(X\).
\end{solution}

\begin{solution}[\ref{exe:oh59gz4i}]
  \label{sol:h64ep1j0}
  Let \(E\) denote the disjoint union of the \(E_x\), and identify
  each \(E_x\) with its canonical image in \(E\), so the \(E_x\) form
  a partition of \(E\); let \(p : E \to X\) be the function which maps
  any element of \(E_x\) to \(x\).  For any subset \(Y\) of X, there
  is a canonical identification of \(\prod_{x \in Y} E_x\) with
  \(\Gamma(Y,p)\), and for any \(x \in X\), \(p^{-1}(x) = E_x\).
  Therefore, the statement in the exercise follows from
  \Cref{thm:srxrngd8}.  Conversely, given this statement we can
  recover that theorem by putting \(E_x = p^{-1}(x)\) for all
  \(x \in X\); \(E\) is then canonically identified with the disjoint
  union of the \(E_x\).  Therefore, the statement in the exercise is a
  reformulation of the theorem.
\end{solution}

\begin{solution}[\ref{exe:eghnwow2}]
  \label{sol:eiv4acyz}
  If \(Y\) were empty, it would be isomorphic to the segment
  \(\emptyset\) of \(X\), contrary to the assumption that \(Y\) is not
  isomorphic to any segment of \(X\); so \(Y \neq \emptyset\).  Fix an
  arbitrary element \(y_0\) of \(Y\), for instance, its minimum.  For
  every element \(x\) of \(X\), define a function
  \(f_x : Y^{(-\infty, x)} \to Y\) by setting
  \begin{displaymath}
    f_x(s) =
    \begin{cases}
      \min \, (Y \setminus s((-\infty, x))) & \text{if} ~ s ~ \text{is
                                              not surjective} \\
      y_0 & \text{if} ~ s ~ \text{is surjective}
    \end{cases}
  \end{displaymath}
  for every function \(s : (-\infty, x) \to Y\).  \Cref{thm:5kg5ewo1}
  gives a unique function \(u : X \to Y\) such that
  \(u(x) = f_x(u \vert_{(-\infty, x)})\) for all \(x \in X\).

  Let \(S\) be the set of segments \(X'\) of \(X\) such that \(u\)
  maps \(X'\) isomorphically onto some segment of \(Y\).  The
  hypothesis that \(Y\) is not isomorphic to any segment of \(X\)
  implies that \(u(X') \neq Y\) for all \(X' \in S\).  We will verify
  that \(S\) satisfies the two conditions of \Cref{thm:y3go6w2h}.

  Suppose \(x \in X\) and \((-\infty, x) \in S\).  As
  \(u((-\infty, x))\) is a proper segment of \(Y\),
  \begin{equation}
    \label{eq:81bkcp3j}
    u(x) = f_x(u \vert_{(-\infty, x)}) =
    \min \, (Y \setminus u((-\infty, x))),
  \end{equation}
  and by \Cref{thm:htdzbbwv}, \(u((-\infty, x)) = (-\infty, u(x))\).
  Therefore,
  \begin{displaymath}
    u((-\infty, x]) = (-\infty, u(x)].
  \end{displaymath}
  If \(x' < x\), then \(u(x')\) belongs to
  \(u((-\infty, x)) = (-\infty, u(x))\), hence \(u(x') < u(x)\); as
  \(u\) is an isomorphism on \((-\infty, x)\), it follows that it is
  strictly increasing on \((-\infty, x]\).  Thus, \(u\) maps
  \((-\infty, x]\) isomorphically onto the segment \((-\infty, u(x)]\)
  of \(Y\), hence \((-\infty, x] \in S\).

  Suppose \((X_i)_{i \in I}\) is a family of elements of \(S\), and
  let \(X'\) denote their union.  Then \(u(X')\) is the union of the
  segments \(u(X_i)\), and is hence a segment.  If \(x,x'\) are
  elements of \(X'\) such that \(x < x'\), then there is an index
  \(i\) such that \(x' \in X_i\); as \(X_i\) is a segment,
  \(x \in X_i\), so as \(u\) is an isomorphism on \(X_i\),
  \(u(x) < u(x')\); thus, \(u\) is strictly increasing on \(X'\).
  Therefore, \(u\) maps \(X'\) isomorphically onto the segment
  \(u(X')\) of \(Y\), and \(X' \in S\).

  By \Cref{thm:y3go6w2h}, every segment of \(X\) belongs to \(S\).
  Therefore, for any \(x \in X\), the segment \((-\infty, x)\) belongs
  to \(S\), so \cref{eq:81bkcp3j} implies that \(u(x)\) is the minimum
  of \(Y \setminus u((-\infty, x))\).  Also, \(X \in S\), hence \(u\)
  induces an isomorphism from \(X\) onto a segment of \(Y\).
\end{solution}

\begin{solution}[\ref{exe:vgms7ihx}]
  \label{sol:cjc2vy2q}
  Suppose \(Q\) is a totally ordered subset of \(P\).  The union \(U\)
  of the sets \(A\) as \((A,f)\) runs over \(Q\) is a segment of \(X\)
  because each \(A\) is a segment.  As \(Q\) is totally ordered, there
  is a unique function \(u : U \to Y\) such that \(u \vert_A = f\) for
  all \((A,f) \in Q\).  The subset \(u(U)\) of \(Y\) is the union of
  the segments \(u(A)\) as \((A,f)\) runs over \(Q\), and is hence a
  segment.  If \(x,x'\) are elements of \(U\) such that \(x < x'\),
  then there is an element \((A,f)\) of \(Q\) such that \(x' \in A\);
  as \(A\) is a segment, \(x \in A\), so as \(f\) is an isomorphism on
  \(A\), \(f(x) < f(x')\); thus, \(u\) is strictly increasing on
  \(U\).  Therefore, \(u\) maps \(U\) isomorphically onto the segment
  \(u(U)\) of \(Y\), hence \((U,u) \in P\).  It is obvious that
  \((U,u)\) is an upper bound of \(Q\).  Thus, every totally ordered
  subset of \(P\) is bounded above.  By Zorn's lemma, \(P\) has a
  maximal element \((M, g)\).

  It suffices to show that either \(M = X\) or \(g(M) = X\).  Suppose
  that neither of these equalities is true.  \Cref{thm:htdzbbwv} then
  gives \(a \in X\) and \(b \in Y\) such that \(M = (-\infty, a)\) and
  \(g(M) = (-\infty, b)\).  Define a function
  \(h : (-\infty, a] \to Y\) by setting
  \begin{displaymath}
    h(x) =
    \begin{cases}
      g(x) & \text{if} ~ x < a \\
      b & \text{if} ~ x = b.
    \end{cases}
  \end{displaymath}
  It is strictly increasing, and its image equals \((-\infty, b]\).
  Therefore, the pair \(((-\infty, a], h)\) belongs to \(P\).  It is
  \(>\) than the maximal element \((M, g)\) of \(P\), a contradiction.
\end{solution}

\begin{solution}[\ref{exe:9d868h6m}]
  \label{sol:yfgula0x}
  If \(a\) is an upper bound of a maximal totally ordered subset \(M\)
  of \(X\), then the set \(M \cup \{ a \}\) is totally ordered because
  \(x \leq a\) for all \(x \in M\); the maximality of \(M\) implies
  that \(a \in M\); thus, \(a\) is the maximum of \(M\), and is also
  the unique upper bound of \(M\) in \(X\); if \(b \in X\) and
  \(a \leq b\), then \(b\) is an upper bound of \(M\), so \(b = a\);
  therefore, \(a\) is a maximal element of \(X\).

  Let \(T\) denote the set of all totally ordered subsets of \(X\).
  For any \(\lambda \in \Lambda\), define a function
  \(f_\lambda : T^{(-\infty, \lambda)} \to T\) by
  \begin{displaymath}
    f_\lambda(s) =
    \begin{cases}
      (\cup_{\mu < \lambda} s(\mu)) \cup \{ h(\lambda) \}
      & \text{if the subset} ~
        (\cup_{\mu < \lambda} s(\mu)) \cup \{ h(\lambda) \} \\
      & \text{of} ~ X ~ \text{is totally ordered} \\
      \emptyset
      & \text{otherwise}
    \end{cases}
  \end{displaymath}
  for every function \(s : (-\infty, \lambda) \to T\).  By
  \Cref{thm:5kg5ewo1}, there is a unique function
  \(u : \Lambda \to T\) such that
  \(u(\lambda) = f_\lambda(u \vert_{(-\infty, \lambda)})\) for all
  \(\lambda \in \Lambda\).  Denote \(u(\lambda)\) by \(M_\lambda\) for
  any \(\lambda \in \Lambda\), and let
  \(M = \cup_{\lambda \in \Lambda} M_\lambda\).  For any
  \(\lambda \in \Lambda\), \(M_\lambda\) equals
  \((\cup_{\mu < \lambda} M_\mu) \cup \{ h(\lambda) \}\) if this
  subset of \(X\) is totally ordered, and is empty otherwise; in
  particular, if \(M_\lambda\) is non-empty, it contains \(M_\mu\) for
  all \(\mu \leq \lambda\).

  If \(x, y \in M\), there are elements \(\lambda, \mu\) of
  \(\Lambda\) such that \(x \in M_\lambda\) and \(y \in M_\mu\); as
  \(\Lambda\) is totally ordered, we can assume that
  \(\mu \leq \lambda\); as \(M_\lambda\) contains \(x\), it is
  non-empty, and hence contains \(M_\nu\) for all
  \(\nu \leq \lambda\); in particular, \(M_\mu \subset M_\lambda\), so
  \(x, y \in M_\lambda\); as \(M_\lambda\) is totally ordered, either
  \(x \leq y\) or \(y \leq x\); it follows that \(M\) is a totally
  ordered subset of \(X\).  Suppose \(N\) is a totally ordered subset
  of \(X\) that contains \(M\); let \(x \in N\), and let
  \(\lambda = h^{-1}(x)\); then the set
  \((\cup_{\mu < \lambda} M_\mu) \cup \{ h(\lambda) \}\) is contained
  in \(N\), and is hence a totally ordered subset of \(X\); therefore,
  it equals \(M_\lambda\), hence \(x = h(\lambda) \in M_\lambda\); it
  follows that \(N = M\).  Thus, \(M\) is a maximal totally ordered
  subset of \(X\).
\end{solution}

\begin{solution}[\ref{exe:5qa65h5w}]
  \label{sol:s0vlfhkj}
  Let \(B\) denote the set consisting of the natural numbers \(0\) and
  \(1\).  For any \(\lambda \in \Lambda\), define a function
  \(f_\lambda : B^{(-\infty, \lambda)} \to B\) by setting
  \begin{displaymath}
    f_\lambda(s) =
    \begin{cases}
      1
      & \text{if} ~ h(\mu) < h(\lambda) ~ \text{for all}
        ~ \mu < \lambda ~ \text{such that} ~ s(\mu) = 1 \\
      0
      & \text{otherwise}
    \end{cases}
  \end{displaymath}
  for every function \(s : (-\infty, \lambda) \to B\).  By
  \Cref{thm:5kg5ewo1}, there is a unique function
  \(u : \Lambda \to B\) such that
  \(u(\lambda) = f_\lambda(u \vert_{(-\infty, \lambda)})\) for all
  \(\lambda \in \Lambda\).  Let \(M\) denote the set of elements of of
  \(X\) of the form \(h(\lambda)\) with \(\lambda \in \Lambda\) and
  \(u(\lambda) = 1\).  Thus, an element \(x\) of \(X\) belongs to
  \(M\) if and only if the element \(\lambda = h^{-1}(x)\) of
  \(\Lambda\) has the property that \(h(\mu) < h(\lambda)\) for all
  \(\mu < \lambda\) such that \(u(\mu) = 1\); equivalently,
  \(x \in M\) if and only if \(w < x\) for all \(w \in M\) such that
  \(h^{-1}(w) < h^{-1}(x)\).

  If \(\lambda,\mu \in h^{-1}(M)\) and \(\mu < \lambda\), then
  \(h(\lambda) \in M\) and \(u(\mu) = 1\), so \(h(\mu) < h(\lambda)\).
  Therefore, \(h\) induces a strictly increasing bijection from
  \(h^{-1}(M)\) onto \(M\).  Since \(h^{-1}(M)\) is totally ordered,
  \Cref{exe:v1tw1ijd} implies that \(h\) induces an isomorphism from
  \(h^{-1}(M)\) onto \(M\).  As \(h^{-1}(M)\) is a well-ordered subset
  of \(\Lambda\), it follows that \(M\) is a well-ordered subset of
  \(X\).  The hypothesis of the exercise implies that \(M\) is bounded
  above in \(X\).  If \(a\) is any upper bound of \(M\) in \(X\), and
  \(\lambda = h^{-1}(a)\), then for all
  \(\mu < \lambda\) such that \(u(\mu) = 1\), \(h(\mu) \in M\), so
  \(h(\mu) \leq a = h(\lambda)\); as \(\mu \neq \lambda\),
  \(h(\mu) \neq h(\lambda)\), so \(h(\mu) < h(\lambda)\); it follows
  that \(a = h(\lambda) \in M\); thus, \(a\) is the maximum of \(M\),
  and is also the unique upper bound of \(M\) in \(X\); if \(b \in X\)
  and \(a \leq b\), then \(b\) is an upper bound of \(M\) in \(X\), so
  \(b = a\); therefore, \(a\) is a maximal element of \(X\).
\end{solution}

\begin{solution}[\ref{exe:x9ewqbv6}]
  \label{sol:l1p9lj7v}
  For any segment \(Y\) of \(X\), let \(G(Y)\) be the set of sections
  \(u \in F(Y)\) with the property that
  \(u \vert_{(-\infty, x]} = f_x(u \vert_{(-\infty, x)})\) for all
  \(x \in Y\).  It is obvious that if \((Y_i)_{i \in I}\) is a family
  of segments of \(X\), \(Y\) their union, and \(u\) an element of
  \(F(Y)\), then \(u \in G(Y)\) if and only if
  \(u \vert_{Y_i} \in G(Y_i)\) for all \(i\); therefore, \(G\) is a
  subsheaf of \(F\).

  Suppose \(Y\) is a segment and \(u,v \in G(Y)\).  Then the set
  \(Y'\) of points \(x \in Y\) such that the germs \(u_x\) and \(v_x\)
  are equal is inductive: if \((-\infty, a) \subset Y'\) for some
  \(a \in Y\), then
  \(u \vert_{(-\infty, a)} = v \vert_{(-\infty, a)}\) because \(F\) is
  a sheaf; therefore,
  \begin{displaymath}
    u \vert_{(-\infty, a]} = f_x(u \vert_{(-\infty, a)}) =
    f_x(v \vert_{(-\infty, a)}) = v \vert_{(-\infty, a]};
  \end{displaymath}
  as \((-\infty, a]\) is an open neighbourhood of \(a\),
  \(u_a = v_a\), hence \(a \in Y'\).  Therefore, by transfinite
  induction (\Cref{thm:5vxgawq0}), \(Y' = Y\).  As \(F\) is a sheaf,
  it follows that \(u = v\).  Thus, \(G(Y)\) has at most one element
  for any segment \(Y\).

  Let \(S\) be the set of segments \(Y\) of \(X\) such that
  \(G(Y) \neq \emptyset\).  We will check that \(S\) satisfies the
  conditions of \Cref{thm:y3go6w2h}.

  Suppose \(a \in X\) and \(Y = (-\infty, a)\) belongs to \(S\).  Let
  \(u\) be the unique element of \(G(Y)\).  We then have the element
  \(v = f_a(u)\) of \(F(Z)\), where \(Z = (-\infty, a]\).  As
  \(v \vert_Y = u\),
  \begin{displaymath}
    v \vert_{(-\infty, x]} = u \vert_{(-\infty, x]} =
    f_x(u \vert_{(-\infty, x)}) = f_x(v \vert_{(-\infty, x)})
  \end{displaymath}
  for all \(x \in Y\), and
  \begin{displaymath}
    v \vert_{(-\infty, a]} = v = f_a(u) = f_a(v \vert_{(-\infty, a)}).
  \end{displaymath}
  Therefore, \(v \in G(Z)\), and \(Z \in S\).

  Suppose, on the other hand, that \(Y\) is the union of a family
  \((Y_i)_{i \in I}\) of elements of \(S\).  For each \(i\), let
  \(u_i\) be the unique element of \(G(Y_i)\).  For all \(i,j\), the
  the restrictions \(u_{ij}\) and \(u_{ji}\) of \(u_i\) and \(u_j\),
  respectively, to the segment \(Y_{ij} = Y_i \cap Y_j\) belong to
  \(G(Y_{ij})\); as \(G(Y_{ij})\) has at most one element,
  \(u_{ij} = u_{ji}\).  Since \(G\) is a sheaf, there is therefore a
  unique element \(u\) of \(G(Y)\) which restricts to \(u_i\) on each
  \(Y_i\).  It follows that \(Y \in S\).

  \Cref{thm:y3go6w2h} implies that every segment of \(X\) belongs to
  \(S\).  In particular, \(X \in S\).

  We can recover \Cref{thm:srxrngd8} from this exercise by taking
  \(F\) to be the sheaf of sections of the function \(p : E \to X\)
  that appears in the statement of that theorem.  Given a function
  \(f_x : \Gamma((-\infty, x)) \to p^{-1}(x)\) for every \(x \in X\),
  we get a function \(g_x\) from \(F((-\infty,x))\) to
  \(F((-\infty, x])\) for every \(x \in X\) by setting
  \begin{displaymath}
    g_x(s)(w) =
    \begin{cases}
      s(w) & \text{if} ~ w < x \\
      f_x(s) & \text{if} ~ w = x
    \end{cases}
  \end{displaymath}
  for all \(s \in F((-\infty, x))\) and \(w \leq x\).  It is obvious
  that \(g_x(s) \vert_{(-\infty, x)} = s\) for all
  \(s \in F((-\infty, x))\).  Therefore, there exists a unique element
  \(u \in F(X)\) such that
  \(u \vert_{(-\infty, x]} = g_x(u \vert_{(-\infty, x)})\) for all
  \(x \in X\).  It is the unique section \(s\) of \(p\) with the
  property that \(u(x) = f_x(u \vert_{(-\infty, x)})\) for all
  \(x \in X\).  So this exercise does generalise \Cref{thm:srxrngd8}.
\end{solution}

\begin{solution}[\ref{exe:k8tmok20}]
  \label{sol:zipx00sf}
  One of the hypotheses on \(X\) is that any well-ordered subset \(A\)
  of \(X\) has an upper bound \(z\); as \(X\) does not have a maximal
  element, there exists an element \(b\) of \(X\) such that \(z < b\);
  thus \(a \leq z < b\) for all \(a \in A\), so \(b\) is a strict
  upper bound of \(A\).

  For every well-ordered subset \(A\) of \(X\), the set \(A^*\) of
  strict upper bounds of \(A\) is a non-empty subset of \(X\), hence
  \(h^{-1}(A^*)\) is a non-empty subset of \(\Lambda\); as \(\Lambda\)
  is well-ordered, \(h^{-1}(A^*)\) has a minimum \(\varphi(A)\); thus,
  \(\varphi(A)\) is the unique element of \(\Lambda\) such that
  \(h(\varphi(A)) \in A^*\) and \(\varphi(A) \leq h^{-1}(b)\) for all
  \(b \in A^*\).  If \(A,B\) are well-ordered subsets of \(X\) such
  that \(A \subset B\), then \(B^* \subset A^*\), hence
  \(\varphi(A) \leq \varphi(B)\).

  Make the set \(\Lambda\) a topological space by giving it the
  topology whose open sets are the segments of \(\Lambda\)
  (\Cref{exe:rpm18mof}(\ref{item:xzh0qsyg})).  For every segment \(U\)
  of \(\Lambda\), let \(F(U)\) be the set of strictly increasing
  functions from \(U\) to \(X\).  By
  \Cref{exe:rpm18mof}(\ref{item:ef4d89pd}), \(F\) is a subsheaf of the
  sheaf of \(X\)-valued functions on \(\Lambda\).  For every
  \(\lambda \in \Lambda\) and \(s \in F((-\infty, \lambda))\), define
  a function \(f_\lambda(s) : (-\infty, \lambda] \to X\) by setting
  \begin{displaymath}
    f_\lambda(s)(\mu) =
    \begin{cases}
      s(\mu) & \text{if} ~ \mu < \lambda \\
      h(\varphi(s((-\infty, \lambda)))) & \text{if} ~ \mu = \lambda
    \end{cases}
  \end{displaymath}
  for all \(\mu \in (-\infty, \lambda]\); this definition makes sense
  because \(s\) is an isomorphism from the well-ordered set
  \((-\infty, \lambda)\) onto the ordered set
  \(s((-\infty, \lambda))\), so \(s((-\infty, \lambda))\) is a
  well-ordered subset of \(X\); as
  \begin{displaymath}
    h(\varphi(s((-\infty, \lambda)))) \in (s((-\infty, \lambda)))^*,
  \end{displaymath}
  \(f_\lambda(s)\) is strictly increasing; it is obvious that
  \(f_\lambda(s) \vert_{(-\infty, \lambda)} = s\); we thus get a
  function \(f_\lambda\) from \(F((-\infty, \lambda))\) to
  \(F((-\infty, \lambda])\) that satisfies the conditions of
  \Cref{exe:x9ewqbv6}.  There is therefore a unique element
  \(u \in F(\Lambda)\) such that
  \(u \vert_{(-\infty, \lambda]} = f_\lambda(u \vert_{(-\infty,
    \lambda)})\) for all \(\lambda \in \Lambda\).  By the definition
  of \(F\), \(u\) is a strictly increasing function from \(\Lambda\)
  to \(X\).  If \(\lambda \in \Lambda\) and \(b\) is a strict upper
  bound of \(u((-\infty, \lambda))\) in \(X\), then
  \(h^{-1}(u(\lambda)) = \varphi(u((-\infty, \lambda))) \leq
  h^{-1}(b)\).

  Suppose \(\lambda, \mu \in \Lambda\) and \(\mu < \lambda\).  Then
  \(u((-\infty, \mu))) \subset u((-\infty, \lambda))\), hence
  \begin{displaymath}
    h^{-1}(u(\mu)) = \varphi(u((-\infty, \mu))) \leq
    \varphi(u((-\infty, \lambda))) = h^{-1}(u(\lambda)).
  \end{displaymath}
  On the other hand, as \(u\) is strictly increasing,
  \(u(\mu) < u(\lambda)\), so
  \(h^{-1}(u(\mu)) \neq h^{-1}(u(\lambda))\).  Therefore,
  \(h^{-1}(u(\mu)) < h^{-1}(u(\lambda))\). Thus, \(h^{-1} \circ u\) is
  a strictly increasing function from \(\Lambda\) to \(\Lambda\).
  \Cref{thm:hw1jxum0} implies that \(\lambda \leq h^{-1}(u(\lambda))\)
  for all \(\lambda \in \Lambda\).

  Let \(M = u(\Lambda)\).  As \(u\) is an isomorphism from \(\Lambda\)
  onto \(M\), \(M\) is a well-ordered subset of \(X\).  Suppose \(M\)
  has a strict upper bound \(b\) in \(X\).  Let
  \(\lambda = h^{-1}(b)\).  Then \(b\) is a strict upper bound of
  \(u((-\infty, \lambda))\) also, hence
  \(h^{-1}(u(\lambda)) \leq h^{-1}(b) = \lambda\).  On the other hand,
  \(\lambda \leq h^{-1}(u(\lambda))\) by the conclusion of the
  previous paragraph.  It follows that
  \(h^{-1}(u(\lambda)) = \lambda\), that is,
  \(u(\lambda) = h(\lambda) = b\), which is a contradiction because
  \(u(\lambda) < b\).

  An alternative definition of \(u\), using \Cref{thm:5kg5ewo1}
  instead of \Cref{exe:x9ewqbv6}, is as follows.  As the well-ordered
  subset \(\emptyset\) of \(X\) has an upper bound in \(X\), the set
  \(X\) is non-empty; fix an element \(x_0 \in X\).  For every
  \(\lambda \in \Lambda\), define a function
  \(f_\lambda : X^{(-\infty, \lambda)} \to X\) by setting
  \begin{displaymath}
    f_\lambda(s) =
    \begin{cases}
      h(\varphi(s((-\infty, \lambda))))
      & \text{if} ~ s ~ \text{is strictly increasing} \\
      x_0
      &
        \text{otherwise}
    \end{cases}
  \end{displaymath}
  for every function \(s : (-\infty, \lambda) \to X\).  Transfinite
  recursion (\Cref{thm:5kg5ewo1}) gives a unique function
  \(u : \Lambda \to X\) such that
  \(u(\lambda) = f_\lambda(u \vert_{(-\infty, \lambda)})\) for all
  \(\lambda \in \Lambda\).  Let \(\Sigma\) be the set of all
  \(\lambda \in \Lambda\) such that \(u\) is strictly increasing on
  \((-\infty, \lambda)\).  For all \(\lambda \in \Sigma\),
  \(u(\lambda) = h(\varphi(u((-\infty, \lambda))))\).  If
  \(\lambda \in \Lambda\) and \((-\infty, \lambda) \subset \Sigma\),
  then for all \(\mu, \nu \in \Lambda\) such that
  \(\nu < \mu < \lambda\), \(u(\nu) < u(\mu)\) because
  \(\mu \in (-\infty, \lambda) \subset \Sigma\) and
  \(u(\mu) = h(\varphi(u((-\infty, \mu))))\) belongs to
  \((u((-\infty, \mu)))^*\); therefore, \(u\) is strictly increasing
  on \((-\infty, \lambda)\), so \(\lambda \in \Sigma\).  Thus,
  \(\Sigma\) is an inductive subset of \(\Lambda\).  Transfinite
  induction (\Cref{thm:5vxgawq0}) implies that \(\Sigma = \Lambda\).
  It follows that \(u(\lambda) = h(\varphi(u((-\infty, \lambda))))\)
  for all \(\lambda \in \Lambda\).  In particular,
  \(u(\lambda) \in (u((-\infty, \lambda)))^*\) for all
  \(\lambda \in \Lambda\), hence \(u\) is strictly increasing.  The
  rest of the solution is as before.
\end{solution}

\begin{solution}[\ref{exe:8yw1mw3a}]
  \label{sol:iyf7ad62}
  There is a subset of \(X\) that is superinductive for \((a, f)\),
  for instance, \(X\) itself.  Let \(A\) be the intersection of all
  the subsets of \(X\) that are superinductive for \((a, f)\).  It is
  the smallest subset of \(X\) that is superinductive for \((a, f)\).
  In particular, it is minimally superinductive for \((a, f)\).  If
  \(Y\) is a subset of \(X\) that is minimally superinductive for
  \((a, f)\), then \(A \subset Y\), hence \(A = Y\).  Thus, \(A\) is
  the unique subset of \(X\) that is minimally superinductive for
  \((a, f)\).

  Suppose \(x \leq f(x)\) for all \(x \in X\).  Then the subset
  \([a, \infty)\) of \(X\) is superinductive for \((a, f)\): it
  contains \(a\); if \(a \leq x\), then as \(x \leq f(x)\),
  \(a \leq f(x)\); and if \(Z\) is a nonempty totally ordered subset
  of \([a, \infty)\), and \(b\) the supremum of \(Z\) in \(X\), then
  as \(a \leq z\) and \(z \leq b\) for an arbitrary element \(z\) of
  the nonempty set \(Z\), \(a \leq b\).  Therefore, \(A\) is contained
  in \([a, \infty)\).  As \(a \in A\), \(a\) is the minimum of \(A\).
\end{solution}

\begin{solution}[\ref{exe:9l6q0c66}]
  \label{sol:st5xn271}
  For any \(x \in X\), let
  \begin{displaymath}
    A_x = \{ y \in X \,\vert\, (x, y) \in R \},
    \quad
    B_x = \{ y \in X \,\vert\, (y, x) \in R \}.
  \end{displaymath}
  For the first part of the exercise, we have to show that if
  \(B_x = X\), then \(A_x = X\).  So suppose \(B_x = X\).  As \(X\) is
  minimally superinductive for \((a, f)\), to show that \(A_x = X\),
  it is enough to show that \(A_x\) is superinductive for \((a, f)\).
  Property (\ref{item:u0r5igra}) of \(R\) implies that
  \((x, a) \in R\), so \(a \in A_x\).  Let \(y \in A_x\); then
  \((x, y) \in R\); as \(B_x = X\), \(y \in B_x\), so
  \((y, x) \in R\); by property (\ref{item:3jc58w0r}) of \(R\),
  \((x, f(y)) \in R\); therefore, \(f(y) \in A_x\); thus,
  \(f(A_x) \subset A_x\).  Let \(Z\) be a nonempty totally ordered
  subset of \(A_x\), and \(b\) the supremum of \(Z\) in \(X\); as
  \(Z \subset A_x\), \((x, y) \in R\) for all \(y \in Z\); by property
  (\ref{item:fh1pwe19}) of \(R\), \((x, b) \in R\); therefore,
  \(b \in A_x\).  Therefore, \(A_x\) is superinductive for \((a, f)\),
  and hence equals \(X\).

  To prove that \(R = X \times X\), it is enough to show that \(X\)
  equals the set
  \begin{displaymath}
    B =
    \{ x \in X \,\vert\, (y, x) \in R ~ \text{for all} ~ y \in X \} =
    \{ x \in X \,\vert\, B_x = X \}.
  \end{displaymath}
  As before, we only need to verify that \(B\) is superinductive for
  \((a, f)\).  By property (\ref{item:u0r5igra}) of \(R\),
  \(a \in B\).  Suppose \(x \in B\); let \(y \in X\); as \(x \in B\),
  \(B_x = X\), so \(y \in B_x\), that is, \((y, x) \in R\); as
  \(B_x = X\), by the first part of the exercise, \(A_x = X\), so
  \(y \in A_x\), that is, \((x, y) \in R\); therefore, by property
  (\ref{item:3jc58w0r}) of \(R\), \((y, f(x)) \in R\); as this is true
  for all \(y \in X\), \(f(x) \in B\); it follows that
  \(f(B) \subset B\).  Let \(Z\) be a nonempty totally ordered subset
  of \(B\), and \(b\) the supremum of \(Z\) in \(X\); let \(y \in X\);
  as \(Z \subset B\), for all \(z \in Z\), \(B_z = X\), so
  \(y \in B_z\), hence \((y, z) \in R\); property
  (\ref{item:fh1pwe19}) implies that \((y, b) \in R\); as this holds
  for all \(y \in X\), \(b \in B\).  Therefore, \(B\) is
  superinductive for \((a,f)\), hence \(R = X \times X\).
\end{solution}

\begin{solution}[\ref{exe:yf8euudc}]
  \label{sol:k2kxymc8}
  Let \(R\) be the set of all elements \((x, y)\) of \(X \times X\)
  such that either \(f(x) \leq y\) or \(y \leq x\).  We will check
  that \(R\) has the properties listed in \Cref{exe:9l6q0c66}.  As
  \(x \leq f(x)\) for all \(x \in X\), and \(X\) is minimally
  inductive for \((a, f)\), by \Cref{exe:8yw1mw3a}, \(a\) is the
  minimum of \(X\); therefore, for all \(x \in X\), \(a \leq x\),
  hence \((x, a) \in R\).  Suppose \(x, y \in X\) are such that
  \((x,y)\) and \((y, x)\) belong to \(R\); we have to check that
  \((x, f(y)) \in R\), that is, that either \(f(x) \leq f(y)\) or
  \(f(y) \leq x\); as \((x, y) \in R\), either \(f(x) \leq y\) or
  \(y \leq x\); as \((y, x) \in R\), either \(f(y) \leq x\) or
  \(x \leq y\); if \(f(x) \leq y\), then as \(y \leq f(y)\),
  \(f(x) \leq f(y)\), so \((x, f(y)) \in R\); if \(f(y) \leq x\), it
  is obvious that \((x,f(y)) \in R\); lastly, if \(y \leq x\) and
  \(x \leq y\), then \(x = y\), so \(f(x) = f(y)\), hence
  \((x,f(y)) \in R\); it follows that \((x, f(y)) \in R\) in all the
  cases.  Let \(x\) be an element of \(X\), \(Y\) a nonempty totally
  ordered subset of \(X\), and \(b\) the supremum of \(Y\) in \(X\),
  and suppose that \((x, y) \in R\) for all \(y \in Y\); we have to
  show that \((x, b) \in R\), that is, that either \(f(x) \leq b\) or
  \(b \leq x\); for all \(y \in Y\), \((x, y) \in R\), hence
  \(f(x) \leq y\) or \(y \leq x\); if \(y \leq x\) for all
  \(y \in Y\), then as \(b\) is the supremum of \(Y\) in \(X\),
  \(b \leq x\), so \((x, b) \in R\); on the other hand, if
  \(y \nleq x\) for some \(y \in Y\), then \(f(x) \leq y\); as \(b\)
  is an upper bound of \(Y\), \(y \leq b\); therefore,
  \(f(x) \leq b\), hence \((x, b) \in R\); thus, \((x, b) \in R\) in
  all the cases.  It follows that \(R\) has the properties listed in
  \Cref{exe:9l6q0c66}, hence \(R = X \times X\).  Therefore, for all
  \(x, y \in X\), either \(f(x) \leq y\) or \(y \leq x\).  In
  particular, as \(x \leq f(x)\) for all \(x \in X\), \(X\) is totally
  ordered.
\end{solution}

\begin{solution}[\ref{exe:htz5ftfy}]
  \label{sol:h2e8o5az}
  Suppose \(x \leq y \leq f(x)\).  By \Cref{exe:yf8euudc}, either
  \(f(x) \leq y\) or \(y \leq x\).  If \(f(x) \leq y\), then as
  \(y \leq f(x)\), we have \(y = f(x)\).  On the other hand, if
  \(y \leq x\), then as \(x \leq y\), we have \(y = x\).

  Suppose \(x < y\).  By \Cref{exe:yf8euudc}, either \(f(x) \leq y\)
  or \(y \leq x\).  If \(y \leq x\), then as \(x < y\), we get
  \(y < y\), contradicting the irreflexivity of \(<\).  Therefore,
  \(f(x) \leq y\).

  Suppose \(x \leq y\).  If \(x = y\), then \(f(x) = f(y)\).  On the
  other hand, if \(x \neq y\), then \(x < y\), so by the previous
  part, \(f(x) \leq y\); as \(y \leq f(y)\), it follows that
  \(f(x) \leq f(y)\).  Thus, \(f(x) \leq f(y)\) in all the cases,
  hence \(f\) is increasing.
\end{solution}

\begin{solution}[\ref{exe:ycuba4tr}]
  \label{sol:yyedjd3p}
  Suppose \(b \in A\) and \(f(b) = b\).  Let \(B\) denote the set of
  elements \(x\) of \(A\) such that \(x \leq b\).  We will check that
  \(B\) is superinductive for \((a, f)\).  \Cref{exe:8yw1mw3a} implies
  that \(a\) is the minimum of \(A\), so \(a \leq b\), hence
  \(a \in B\).  Suppose \(x \in B\); then \(x \leq b\); by
  \Cref{exe:htz5ftfy}, \(f\) is increasing on \(A\), so
  \(f(x) \leq f(b)\); as \(b\) is a fixed point of \(f\), it follows
  that \(f(x) \leq b\); as \(A\) is superinductive for \((a, f)\),
  \(f(x) \in A\); hence \(f(x) \in B\); thus, \(f(B) \subset B\).  Let
  \(Z\) be a nonempty totally ordered subset of \(B\), and \(c\) the
  supremum of \(Z\) in \(X\); as \(Z \subset B\), \(z \leq b\) for all
  \(z \in Z\); therefore, as \(b\) is the supremum of \(Z\) in \(X\),
  \(c \leq b\); thus, \(c \in B\).  It follows that the subset \(B\)
  of \(A\) is superinductive for \((a, f)\).  As \(A\) is minimally
  superinductive for \((a,f)\), we have \(B = A\).  Thus, \(x \leq b\)
  for all \(x \in A\), that is, \(b\) is the maximum of \(A\).

  It is obvious that if \(b\) is the maximum of \(A\), it is the
  supremum of \(A\).

  Suppose \(b\) is the supremum of \(A\).  The subset \(A\) of \(A\)
  is nonempty because it contains \(a\), and it is totally ordered by
  \Cref{exe:yf8euudc}.  As \(b\) is the supremum of \(A\) in \(X\),
  and \(A\) is superinductive for \((a, f)\), it follows that
  \(b \in A\).  Since \(f(A) \subset A\), \(f(b) \in A\).  Therefore,
  as \(b\) is the supremum of \(A\) in \(X\), \(f(b) \leq b\).  On the
  other hand, by the hypothesis on \(f\), \(b \leq f(b)\).  It follows
  that \(f(b) = b\).

  If \(b_1\) and \(b_2\) are fixed points of \(f\) in \(A\), then each
  of them is the maximum of \(A\), hence \(b_1 = b_2\).  Thus, \(f\)
  has at most one fixed point in \(A\).
\end{solution}

\begin{solution}[\ref{exe:al7fffh1}]
  \label{sol:ibjwlmqf}
  Let \(x \in X\), and let \(X_x = [x, \infty)\).  As \(y \leq f(y)\)
  for all \(y \in X\), \(f\) induces a function \(f_x : X_x \to X_x\).
  The hypothesis on \(f\) implies that \(y \leq f_x(y)\) for all
  \(y \in X_x\).  By \Cref{exe:8yw1mw3a}, there is a unique subset
  \(A_x\) of \(X_x\) that is minimally superinductive for
  \((x, f_x)\).  Also, by \Cref{exe:yf8euudc}, \(A_x\) is totally
  ordered.  The hypothesis on \(X\) gives a supremum \(b\) of \(A_x\)
  in \(X\).  As \(x \in A_x\), \(x \leq b\), so \(b \in X_x\).  Thus,
  \(b\) is the supremum of \(A_x\) in \(X_x\).  By
  \Cref{exe:ycuba4tr}, \(b \in A_x\) and \(f_x(b) = b\).  Thus,
  \(x \leq b\) and \(f(b) = b\).

  The hypothesis on \(X\) implies that the totally ordered subset
  \(\emptyset\) of \(X\) has a supremum \(a\) in \(X\).  It is obvious
  that \(a\) is the minimum of \(X\).  Putting \(a\) for \(x\) in the
  first part of the exercise gives an element \(b\) of \(X\) such that
  \(f(b) = b\).  Thus, \(f\) has a fixed point in \(X\).
\end{solution}

\begin{solution}[\ref{exe:313bq61a}]
  \label{sol:eh1t1jcm}
  Let \(Y\) be a subset of \(X\) that does not have an infimum in
  \(X\).  Let \(L\) be the set of lower bounds of \(Y\) in \(X\):
  \begin{displaymath}
    L = \{ l \in X \,\vert\, l \leq x ~ \text{for all} ~ x \in Y \}.
  \end{displaymath}
  The infimum of \(Y\) is just the maximum of \(L\), so \(L\) does not
  have a maximum.  It follows that \(L \cap Y = \emptyset\); for, if
  \(x \in L \cap Y\), then \(x \in Y\), so \(l \leq x\) for all
  \(l \in L\); as \(x \in L\), this means that \(x\) is the maximum of
  \(L\), a contradiction.  We will verify that \(L\) is superinductive
  for \((a, f)\).  As \(X\) is minimally superinductive for
  \((a, f)\), by \Cref{exe:8yw1mw3a}, \(a\) is the minimum of \(X\),
  so \(a \leq x\) for all \(x \in Y\), hence \(a \in L\).  If \(l\) is
  an element of \(L\), then as \(L \cap Y = \emptyset\),
  \(l \notin Y\); as \(l \leq x\) for all \(x \in Y\), this implies
  that \(l < x\) for all \(x \in Y\); by \Cref{exe:htz5ftfy},
  \(f(l) \leq x\) for all \(x \in Y\); therefore, \(f(l) \in L\),
  hence \(f(L) \subset L\).  Suppose \(Z\) is a nonempty subset of
  \(X\), and \(b\) the supremum of \(Z\) in \(X\); then for all
  \(x \in B\), \(z \leq x\) for all \(z \in Z\), hence \(b \leq x\);
  so \(b \in L\).  Therefore, \(L\) is superinductive for \((a, f)\).
  As \(X\) is minimally superinductive for \((a, f)\), it follows that
  \(L = X\).  Thus, \(Y = X \cap Y = L \cap Y = \emptyset\).

  We will now show that \(X\) is well-ordered.  Let \(Y\) be a
  nonempty subset of \(X\).  We have to show that \(Y\) has a minimum.
  By the first part of the exercise, \(Y\) has an infimum \(b\).  If
  \(b \in Y\), then it must be the minimum of \(Y\), so it suffices to
  show that \(b \in Y\).  Suppose \(b \notin Y\).  Then for all
  \(x \in Y\), as \(b \leq x\), we have \(b < x\).  By
  \Cref{exe:htz5ftfy}, \(f(b) \leq x\) for all \(x \in Y\).  Thus,
  \(f(b)\) is a lower bound of \(Y\); so as \(b\) is the infimum of
  \(Y\), \(f(b) \leq b\).  As \(b \leq f(b)\) by the hypothesis on
  \(f\), it follows that \(f(b) = b\).  \Cref{exe:ycuba4tr} implies
  that \(b\) is the maximum of \(X\).  This is a contradiction because
  we have seen above that \(b < x\) if \(x\) is an arbitrary element
  of the nonempty set \(Y\).  Therefore, \(b\) belongs to \(Y\), and
  is hence the minimum of \(Y\).
\end{solution}

\begin{solution}[\ref{exe:m0vdltb2}]
  \label{sol:7pp5s6s1}
  Let \(U\) denote the union of the elements of \(\mathcal{A}\).  It
  is the supremum of \(\mathcal{A}\) in \(\mathcal{P}\).  As
  \(\mathcal{A}\) is minimally superinductive for \((\emptyset, f)\),
  it follows from \Cref{exe:ycuba4tr} that \(U \in \mathcal{A}\) and
  \(f(U) = U\).  But the definition of \(f\) implies that \(X\) is its
  only fixed point.  Therefore, \(U = X\), so \(X \in \mathcal{A}\).

  Let \(h : \mathcal{A}' \to X\) be the restriction of
  \(g : \mathcal{P}' \to X\).  Define a function
  \(u : X \to \mathcal{A}'\) by setting, for every \(x \in X\),
  \(u(x)\) to be the union of the elements of \(\mathcal{A}\) that do
  not contain \(x\); thus, \(u(x)\) is the supremum in \(\mathcal{P}\)
  of the set of elements of \(\mathcal{A}\) that do not contain \(x\);
  as \(\emptyset \in \mathcal{A}\), this set is a nonempty subset of
  the totally ordered set \(\mathcal{A}\) (\Cref{exe:yf8euudc}); that
  \(\mathcal{A}\) is superinductive for \((\emptyset, f)\) implies
  that \(u(x) \in \mathcal{A}\); it is obvious that \(x \notin u(x)\),
  so \(u(x) \neq X\); therefore, \(u(x)\) does belong to
  \(\mathcal{A}'\).

  We will first show that \(h(u(x)) = x\) for all \(x \in X\).  Let
  \(x \in X\), and let \(Y = u(x)\).  We have to verify that
  \(g(Y) = x\).  As \(u(x) \neq X\), \(Y \neq X\), so
  \(f(Y) = Y \cup \{ g(Y) \}\); \(g(Y) \notin Y\), so \(Y < f(Y)\);
  thus, \(f(Y) \in \mathcal{A}\) and \(f(Y) \nleq Y\); by definition,
  \(Z \leq Y\) for all \(Z \in \mathcal{A}\) such that \(x \notin Z\);
  therefore, \(x \in f(Y)\).  On the other hand, \(x \notin u(x)\), so
  \(x \notin Y\).  Thus, \(x \in f(Y) \setminus Y\).  As
  \(f(Y) \setminus Y = \{ g(Y) \}\), it follows that \(x = g(Y)\).
  Therefore, \(h(u(x)) = x\) for all \(x \in X\).

  Now we will check that \(u(h(Y)) = Y\) for all
  \(Y \in \mathcal{A}'\).  Let \(Y \in \mathcal{A}'\), and let
  \(x = h(Y)\) and \(Y' = u(x)\).  We have to show that \(Y = Y'\).
  By the definition of \(h\), \(x = h(Y) = g(Y)\) does not belong to
  \(Y\); as \(Z \leq Y'\) for all the elements of \(\mathcal{A}\) that
  do not contain \(x\), we therefore have \(Y \leq Y'\).  On the other
  hand, as \(f(Y) = Y \cup \{ g(Y) \} = Y \cup \{ x \}\),
  \(x \in f(Y)\); as \(x \notin Y'\), this implies that
  \(f(Y) \nleq Y'\); the set \(\mathcal{A}\) is minimally
  superinductive for \((\emptyset, f)\), so by \Cref{exe:yf8euudc},
  \(Y' \leq Y\).  It follows that \(Y = Y'\).  Therefore,
  \(u(h(Y)) = Y\) for all \(Y \in \mathcal{A}'\).

  Thus, \(h : \mathcal{A}' \to X\) is a bijection.  By transporting
  the order on \(\mathcal{A}'\) through this bijection, we get a
  unique order on \(X\) such that \(h\) is an isomorphism of ordered
  sets.  \Cref{exe:313bq61a} implies that \(\mathcal{A}\) is
  well-ordered, so \(\mathcal{A}'\) is well-ordered, and hence \(X\)
  is also well-ordered.  So we have proved that the set \(X\) has a
  well-order.
\end{solution}

\begin{solution}[\ref{exe:rw9lhmv9}]
  \label{sol:dcd6z7iz}
  For every \(i \in I\), denote the given order on \(X_i\) by
  \(\leq_i\), and let \(\leq_I\) denote the order on \(I\).  Define a
  relation \(\leq\) on \(X\) by setting \(x \leq y\) if either
  \(p(x) <_I p(y)\), or \(p(x) = p(y)\) and \(x \leq_{p(x)} y\) in the
  ordered set \(X_{p(x)}\).  The reflexivity of this relation is
  obvious.  If \(x, y \in X\) are such that \(x \leq y\), then the
  definition of \(\leq\) implies that \(p(x) \leq_I p(y)\); therefore,
  if also \(y \leq x\), then \(p(y) \leq_I p(x)\), hence the
  anti-symmetry of \(\leq_I\) implies that \(p(x) = p(y)\); thus,
  \(x \leq_{p(x)} y\) and \(y \leq_{p(x)} x\); the anti-symmetry of
  \(\leq_{p(x)}\) implies that \(x = y\); therefore, \(\leq\) is
  anti-symmetric.  If \(x, y, z \in X\) are such that \(x \leq y\) and
  \(y \leq z\), then \(p(x) \leq_I p(y)\) and \(p(y) \leq_I p(z)\); if
  either \(p(x) \neq p(y)\) or \(p(y) \neq p(z)\), then
  \(p(x) <_I p(z)\), hence \(x \leq z\); on the other hand, if
  \(p(x) = p(y) = p(z)\), then \(x \leq_{p(x)} y\) and
  \(y \leq_{p(x)} z\), so the transitivity of \(\leq_{p(x)}\) implies
  that \(x \leq_{p(x)} z\), hence \(x \leq z\); therefore, \(\leq\) is
  transitive.  It follows that \(\leq\) is an order on \(X\).  If
  \(i \in I\) and \(x, y \in X_i\), then \(p(x) = p(y) = i\), so
  \(x \leq y\) if and only if \(x \leq_i y\); thus, \(\leq\) induces
  \(\leq_i\) on \(X_i\); on the other hand, if \(x, y \in X\) and
  \(p(x) <_I p(y)\), then the definition of \(\leq\) implies that
  \(x \leq y\); so \(\leq\) has the required properties.  If \(\leq'\)
  is another order on \(X\) that has these properties, then for any
  \(x, y \in X\), we have \(x \leq y\) if and only if either
  \(p(x) <_I p(y)\), or \(p(x) = p(y)\) and \(x \leq_{p(x)} y\);
  therefore, \(\leq'\) equals \(\leq\); thus, \(\leq\) is the unique
  order on \(X\) that satisfies the stated conditions.

  Suppose now that all the orders \(\leq_i\) and \(\leq_I\) are
  well-orders.  We have to show that \(\leq\) is a well-order on
  \(X\).  Let \(Y\) be a nonempty subset of \(X\).  Then the set \(J\)
  of the indices \(i \in I\) such that \(Y \cap X_i \neq \emptyset\)
  is nonempty.  As \(\leq_I\) is a well-order, \(J\) has a minimum
  \(k\) with respect to \(\leq_I\).  As \(\leq_k\) is a well-order,
  the nonempty subset \(Y \cap X_k\) of \(X_k\) has a minimum \(a\)
  with respect to \(\leq_k\).  We will check that \(a\) is the minimum
  of \(Y\) with respect to \(\leq\).  Let \(x \in Y\).  Then
  \(x \in Y \cap X_{p(x)}\), so \(p(x) \in J\), hence
  \(k \leq_I p(x)\).  If \(k < p(x)\), then as \(p(a) = k\), \(p(a) <
  p(x)\), so \(a \leq x\).  On the other hand, if \(k = p(x)\), then
  \(a, x \in Y \cap X_k\); as \(a\) is the minimum of \(Y \cap X_k\)
  with respect to \(\leq_k\), \(a \leq_k x\); thus, \(p(a) = p(x)\),
  and \(a \leq_{p(a)} x\), hence \(a \leq x\).  Therefore, in all the
  cases, \(a \leq x\).  It follows that \(a\) is the minimum of \(Y\)
  with respect to \(\leq\).  This verifies that \(\leq\) is a
  well-order on \(X\).
\end{solution}

\begin{solution}[\ref{exe:mfln3otu}]
  \label{sol:dnerb46y}
  As any ordered set is a segment of itself, \(s \sqsubseteq s\) for
  every element \(s = (Y, \leq)\) of \(S\), so \(\sqsubseteq\) is
  reflexive.  If \(s = (Y, \leq)\) and \(s' = (Y', \leq')\) are two
  elements of \(S\) such that \(s \sqsubseteq s'\) and
  \(s' \sqsubseteq s\), then \(Y \subset Y'\) and \(Y' \subset Y\), so
  \(Y = Y'\); as \(\leq\) and \(\leq'\) induce each other, they are
  equal; therefore, \(s = s'\); thus, \(\sqsubseteq\) is
  anti-symmetric.  If \(s = (Y, \leq)\), \(s' = (Y', \leq')\), and
  \(s'' = (Y'', \leq'')\) are elements of \(S\) such that
  \(s \sqsubseteq s'\) and \(s' \sqsubseteq s''\), then
  \(Y \subset Y'\) and \(Y' \subset Y''\), hence \(Y \subset Y''\); as
  \(\leq''\) induces \(\leq'\) on \(Y'\), and \(\leq'\) induces
  \(\leq\) on \(Y\), \(\leq''\) induces \(\leq\) on \(Y\); because
  \(Y\) is a segment of \(Y'\) with respect to \(\leq'\), \(Y'\) is a
  segment of \(Y''\) with respect to \(\leq''\), and \(\leq''\)
  induces \(\leq'\) on \(Y'\), it follows that \(Y\) is a segment of
  \(Y''\) with respect to \(\leq''\); therefore,
  \(s \sqsubseteq s''\), hence \(\sqsubseteq\) is transitive.  This
  shows that \(\sqsubseteq\) is an order on \(S\).

  Let \(T = (s_i)_{i \in I}\) be a family of elements of \(S\) which
  is totally ordered with respect to \(\sqsubseteq\).  Write
  \(s_i = (Y_i, \leq_i)\) for every \(i\).  Let \(Y\) be the union of
  the \(Y_i\).  Define a relation \(\leq\) on \(Y\) as follows.  Let
  \(x, y \in Y\).  As the family \(T\) is totally ordered with respect
  to \(\sqsubseteq\), there exists an element \(i\) of \(I\) such that
  \(x, y \in Y_i\); define \(x \leq y\) if \(x \leq_i y\) in \(Y_i\).
  To check that this definition is independent of the choice of \(i\),
  let \(j\) be another element of \(I\) such that \(x, y \in Y_j\).
  We have to prove that \(x \leq_j y\) in \(Y_j\).  As the family
  \(T\) is totally ordered with respect to \(\sqsubseteq\), either
  \(s_i \sqsubseteq s_j\) or \(s_j \sqsubseteq s_i\).  If
  \(s_i \sqsubseteq s_j\), then \(Y_i \subset Y_j\) and \(\leq_i\) is
  induced by \(\leq_j\), so as \(x, y \in Y_i\) and \(x \leq_i y\), we
  have \(x \leq_j y\).  On the other hand, if \(s_j \sqsubseteq s_i\),
  then \(Y_j \subset Y_i\) and \(\leq_j\) is induced by \(\leq_i\), so
  as \(x, y \in Y_j\) and \(x \leq_i y\), we have \(x \leq_j y\).
  Therefore, \(\leq\) is a well-defined relation on \(Y\).  It has the
  property that for all \(i \in I\) and \(x, y \in Y_i\), we have
  \(x \leq_i y\) if and only if \(x \leq y\).

  We will first check that \(\leq\) is an order on \(Y\).  The
  reflexivity and anti-symmetry of \(\leq\) follow immediately from
  those of the relations \(\leq_i\).  As for the transitivity of
  \(\leq\), if \(x, y, z \in Y\) are such that \(x \leq y\) and
  \(y \leq z\), then as the family \(T\) is totally ordered, there is
  an element \(i\) of \(I\) such that \(x, y, z \in Y_i\); the
  definition of \(\leq\) implies that \(x \leq_i y\) and
  \(y \leq_i z\), so the transitivity of \(\leq_i\) gives
  \(x \leq_i z\), hence \(x \leq z\).  Therefore, \(\leq\) is an order
  on \(Y\).  For any \(i \in I\) and \(x, y \in Y_i\), we have
  \(x \leq_i y\) if and only if \(x \leq y\), so \(\leq\) induces
  \(\leq_i\) on \(Y_i\).  If \(\leq'\) is another order on \(Y\) that
  induces \(\leq_i\) on each \(Y_i\), then for all \(x, y \in Y\),
  there is an index \(i \in I\) such that \(x, y \in Y_i\); as
  \(\leq_i\) is induced by \(\leq'\), we have \(x \leq_i y\) if and
  only if \(x \leq' y\); on the other hand, we have already seen that
  \(x \leq_i y\) if and only if \(x \leq y\); therefore, \(x \leq' y\)
  if and only if \(x \leq y\); it follows that \(\leq'\) equals
  \(\leq\).  Thus, \(\leq\) is the unique order on \(Y\) that induces
  \(\leq_i\) on every \(Y_i\).

  We will now show that \(\leq\) is a well-order on \(Y\).  Let \(A\)
  be a nonempty subset of \(Y\).  Then there is an element \(i\) of
  \(I\) such that \(A \cap Y_i \neq \emptyset\).  As \(\leq_i\) is a
  well-order on \(Y_i\), \(A \cap Y_i\) has a minimum \(a\) with
  respect to \(\leq_i\).  We will verify that \(a\) is the minimum of
  \(A\) with respect to \(\leq\).  Let \(x \in A\).  We have to check
  that \(a \leq x\).  Let \(j\) be an element of \(I\) such that
  \(x \in Y_j\).  As the family \(T\) is totally ordered, either
  \(s_i \sqsubseteq s_j\) or \(s_j \sqsubseteq s_i\).  If
  \(s_i \sqsubseteq s_j\), then \(Y_i \subset Y_j\), so
  \(a, x \in Y_j\); if \(a <_j x\), then \(a \leq_j x\), so
  \(a \leq x\); on the other hand, if \(a \nless_j x\), then as
  \(\leq_j\) is a total order on \(Y_j\), \(x \leq_j a\); as
  \(a \in Y_i\), and \(Y_i\) is a segment of \(Y_j\) with respect to
  \(\leq_j\), we get \(x \in Y_i\); thus, \(x \in A \cap Y_i\); as
  \(a\) is the minimum of \(A \cap Y_i\) with respect to \(\leq_i\),
  \(a \leq_i x\); since \(\leq_i\) is induced by \(\leq_j\),
  \(a \leq_j x\); the anti-symmetry of \(\leq_j\) implies that
  \(a = x\); in particular, \(a \leq x\).  If \(s_j \sqsubseteq s_i\),
  then \(Y_j \subset Y_i\), so \(a, x \in A \cap Y_i\); as \(a\) is
  the minimum of \(A \cap Y_i\) with respect to \(\leq_i\),
  \(a \leq_i x\), hence \(a \leq x\).  Therefore, in both the cases,
  \(a \leq x\).  Thus, \(a\) is the minimum of \(A\).  It follows that
  \(\leq\) is a well-order on \(Y\).  We thus have an element
  \(s = (Y, \leq)\) of \(S\).

  The definitions of \(Y\) and \(\leq\) imply that for any
  \(i \in I\), \(Y_i \subset Y\), and \(\leq_i\) is induced by
  \(\leq\).  We will show that \(Y_i\) is a segment of \(Y\) with
  respect to \(\leq\).  Let \(x \in Y_i\) and \(y \in Y\) be such that
  \(y \leq x\).  We have to show that \(y \in Y_i\).  There is an
  element \(j\) of \(I\) such that \(y \in Y_j\).  If
  \(s_i \sqsubseteq s_j\), \(Y_i\) is a segment of \(Y_j\) with
  respect to \(\leq_j\); also, as \(y \leq x\) and \(x, y \in A_j\),
  \(y \leq_j x\); therefore, \(y \in Y_i\).  On the other hand, if
  \(s_j \sqsubseteq s_i\), then \(Y_j\) is a subset of \(Y_i\), so
  \(y \in Y_i\).  It follows that \(Y_i\) is a segment of \(Y\) with
  respect to \(\leq\).  Thus, \(s_i \sqsubseteq s\) for all
  \(i \in I\), hence \(s\) is an upper bound of the family \(T\) in
  \(S\).

  We will now verify that \(s\) is in fact the supremum of the family
  \(T\) in \(S\).  Let \(s' = (Y', \leq')\) be an arbitrary upper
  bound of \(T\) in \(S\).  We have to show that \(s \sqsubseteq s'\).
  For all \(i \in I\), \(s_i \sqsubseteq s'\), so \(Y_i \subset Y'\);
  as \(Y\) is the union of the \(Y_i\) as \(i\) runs over \(I\), it
  follows that \(Y \subset Y'\).  Let \(x, y \in Y\); then there is an
  element \(i\) of \(I\) such that \(x, y \in Y_i\); we have already
  seen that \(x \leq_i y\) if and only if \(x \leq y\); on the other
  hand, as \(s_i \sqsubseteq s'\), \(Y_i\) is a subset of \(Y'\) and
  \(\leq_i\) is induced by \(\leq'\), so \(x \leq_i y\) if and only if
  \(x \leq' y\); it follows that \(x \leq y\) if and only if
  \(x \leq' y\); therefore, \(\leq\) is induced by \(\leq'\).  Let
  \(x \in Y\) and \(y \in Y'\) be such that \(y \leq' x\); then there
  is an element \(i\) of \(I\) such that \(x \in Y_i\); as
  \(s_i \sqsubseteq s'\), \(Y_i\) is a segment of \(Y'\) with respect
  to \(\leq'\); so, as \(x \in Y_i\), \(y \in Y'\), and \(y \leq' x\),
  we have \(y \in Y_i\); since \(Y_i \subset Y\), this implies that
  \(y \in Y\); therefore, \(Y\) is a segment of \(Y'\) with respect to
  \(\leq'\).  Thus, \(s \sqsubseteq s'\).  Therefore, \(s\) is the
  supremum of the family \(T\) in \(S\).
\end{solution}

\begin{solution}[\ref{exe:fgbbb6v2}]
  \label{sol:v2ix8vst}
  Let \(\mathcal{P}\) denote the power set of \(X\), and
  \(\mathcal{P}'\) the set of elements of \(\mathcal{P}\) that are
  distinct from \(X\).  The axiom of choice gives a function
  \(g : \mathcal{P}' \to X\) with the property that
  \(g(Y) \in X \setminus Y\) for all \(Y \in \mathcal{P}'\).  Let
  \(S\) be the ordered set from \Cref{exe:mfln3otu}.  We define a
  function \(f : S \to S\) as follows.  Let \(s = (Y, \leq)\) be an
  element of \(S\).  If \(Y = X\), let \(f(s) = s\).  If \(Y \neq X\),
  let \(f(s) = (Y', \leq')\), where \(Y' = Y \cup \{ g(Y) \}\), and
  \(\leq'\) is the unique order on \(Y'\) which induces \(\leq\) on
  \(Y\), and satisfies the condition that \(x \leq g(Y)\) for all
  \(x \in Y\); the existence and uniqueness of \(\leq'\) and the fact
  that it is a well-order on \(Y'\) follow from \Cref{exe:rw9lhmv9}.

  By the definition of \(f\), \(s \sqsubseteq f(s)\) for all
  \(s \in S\).  \Cref{exe:mfln3otu} implies that every totally ordered
  subset of \(S\) has a supremum in \(S\).  Therefore, by the
  Bourbaki-Witt fixed point theorem (\Cref{exe:al7fffh1}), \(f\) has a
  fixed point \(s_0 = (Y_0, \leq_0)\).  The definition of \(f\)
  implies that for every element \(s = (Y, \leq)\) of \(S\) such that
  \(Y \neq X\), if we write \(f(s) = (Y', \leq')\), then
  \(Y \neq Y'\), hence \(s \neq f(s)\).  Therefore, \(Y_0 = X\), so
  \(\leq_0\) is a well-order on \(X\).
\end{solution}

\begin{solution}[\ref{exe:64qxlchj}]
  \label{sol:ejhxq7pq}
  Let \(\mathcal{T}\) be a totally ordered subset of \(\mathcal{S}\).
  Let \(A\) be the union of the elements of \(\mathcal{T}\).  Suppose
  \(x, y \in A\).  As \(\mathcal{T}\) is totally ordered, there exists
  an element \(Y\) of \(\mathcal{T}\) such that \(x, y \in Y\).  As
  \(Y\) is a totally ordered subset of \(X\), either \(x \leq y\) or
  \(y \leq x\).  Therefore, \(A\) is totally ordered, so
  \(A \in \mathcal{S}\).  It is obvious that \(Y \leq A\) for all
  \(Y \in \mathcal{T}\), so \(A\) is an upper bound of \(\mathcal{T}\)
  in \(\mathcal{S}\).  Suppose \(B\) is another upper bound of
  \(\mathcal{T}\) in \(\mathcal{S}\).  Then every element of
  \(\mathcal{T}\) is a subset of \(B\), so the union \(A\) of the
  elements of \(\mathcal{T}\) is also a subset of \(B\).  Thus,
  \(A \leq B\).  It follows that \(A\) is the supremum of
  \(\mathcal{T}\) in \(\mathcal{S}\).  Therefore, every totally
  ordered subset of \(\mathcal{S}\) has a supremum in \(\mathcal{S}\).
\end{solution}

\begin{solution}[\ref{exe:h4ww3mr2}]
  \label{sol:pgr3pn5c}
  Let \(\mathcal{S}\) be the ordered set from \Cref{exe:64qxlchj}, and
  \(\mathcal{M}\) the set of maximal elements of \(\mathcal{S}\).  The
  axiom of choice gives a function
  \(g : \mathcal{S} \setminus \mathcal{M} \to \mathcal{S}\) such that
  \(Y < g(Y)\) for all \(Y \in \mathcal{S} \setminus \mathcal{M}\).
  Define a function \(f : \mathcal{S} \to \mathcal{S}\) by setting
  \begin{displaymath}
    f(Y) =
    \begin{cases}
      Y & \text{if} ~ Y \in \mathcal{M} \\
      g(Y) & \text{if} ~ Y \notin \mathcal{M}
    \end{cases}
  \end{displaymath}
  for all \(Y \in \mathcal{S}\).  Then \(Y \leq f(Y)\) for all
  \(Y \in \mathcal{S}\).  As every totally ordered subset of
  \(\mathcal{S}\) has a supremum in \(\mathcal{S}\), by the
  Bourbaki-Witt fixed point theorem (\Cref{exe:al7fffh1}), \(f\) has a
  fixed point \(M\) in \(\mathcal{S}\).  For all
  \(Y \in \mathcal{S} \setminus \mathcal{M}\), \(Y < g(Y)\), so
  \(Y < f(Y)\).  Therefore, \(M \in \mathcal{M}\).  Thus, \(M\) is a
  maximal totally ordered subset of \(X\).  The hypothesis on \(X\)
  gives an upper bound \(a\) of \(M\) in \(X\).  It follows from part
  (\ref{item:ebwyq3lr}) of \Cref{exe:9d868h6m} that \(a\) is a maximal
  element of \(X\).
\end{solution}

\begin{solution}[\ref{exe:51daga9q}]
  \label{sol:5u09m13a}
  Let \(S\) be the ordered set in \Cref{exe:mfln3otu}.  For any subset
  \(Y\) of \(X\), let \(\leq_Y\) be the order on \(Y\) induced by the
  given order on \(X\).  Let \(u : \mathcal{S} \to S\) be the function
  that maps any element \(Y\) of \(\mathcal{S}\) to the element
  \((Y, \leq_Y)\) of \(S\).  It is obvious that \(u\) is injective,
  and that for all \(Y, Z \in \mathcal{S}\), we have
  \(Y \sqsubseteq Z\) if and only if \(u(Y) \sqsubseteq u(Z)\).  It
  follows that \(\sqsubseteq\) is an order on \(\mathcal{S}\), and
  that \(u\) is an isomorphism of ordered sets from \(\mathcal{S}\)
  onto the subset \(u(\mathcal{S})\) of \(S\).  If
  \(\mathcal{T} = (Y_i)_{i \in I}\) is a family of elements of
  \(\mathcal{S}\) that is totally ordered with respect to
  \(\sqsubseteq\), and \(Y\) the union of the \(Y_i\), then \(\leq_Y\)
  induces \(\leq_{Y_i}\) on each \(Y_i\); therefore, by
  \Cref{exe:mfln3otu}, \(\leq\) is a well-order on \(Y\) (so
  \(Y \in \mathcal{S}\)), and \(u(Y)\) is the supremum of the totally
  ordered family \(T = (u(Y_i))_{i \in I}\) in \(S\); as \(u\) is an
  isomorphism onto \(u(\mathcal{S})\), this implies that \(Y\) is the
  supremum of the family \(\mathcal{T}\) in \(\mathcal{S}\).

  Suppose \(Y\) is an element of \(\mathcal{S}\), and let \(a\) be a
  strict upper bound of \(M\).  Let \(\bar{Y} = Y \cup \{ a \}\).
  Then the order \(\leq_{\bar{Y}}\) on \(\bar{Y}\) induces \(\leq_Y\)
  on \(Y\), and satisfies the condition that \(x \leq a\) for all
  \(x \in Y\).  By \Cref{exe:rw9lhmv9}, \(\leq_{\bar{Y}}\) is the
  unique order on \(\bar{Y}\) with this property, and is a well-order.
  Therefore, \(\bar{Y} \in \mathcal{S}\).  If \(x \in Y\) and
  \(w \in \bar{Y}\) are such that \(w < x\), then \(w \neq a\) because
  \(x < a\); therefore, \(w \in Y\), hence \(Y\) is a segment of
  \(\bar{Y}\).  Thus, \(Y \sqsubseteq \bar{Y}\).  As
  \(Y \neq \bar{Y}\), this implies that \(Y\) is not a maximal element
  of \(\mathcal{S}\).

  It follows that if \(a\) is an upper bound of a maximal element
  \(M\) of \(\mathcal{S}\), then \(a\) is not a strict upper bound of
  \(M\).  This means that \(a\) belongs to \(M\), and is hence the
  maximum and the unique upper bound of \(M\) in \(X\).  If
  \(b \in X\) and \(a \leq b\), then \(b\) is an upper bound of \(M\),
  so \(b = a\).  Therefore, \(a\) is a maximal element of \(X\).

  Suppose that every well-ordered subset of \(X\) is bounded above.
  Let \(\mathcal{M}\) be the set of maximal elements of
  \(\mathcal{S}\).  The axiom of choice gives a function
  \(g : \mathcal{S} \setminus \mathcal{M} \to \mathcal{S}\) such that
  \(Y \sqsubset g(Y)\) for all
  \(Y \in \mathcal{S} \setminus \mathcal{M}\).  Define a function
  \(f : \mathcal{S} \to \mathcal{S}\) by setting
  \begin{displaymath}
    f(Y) =
    \begin{cases}
      Y & \text{if} ~ Y \in \mathcal{M} \\
      g(Y) & \text{if} ~ Y \in \mathcal{S} \setminus \mathcal{M}
    \end{cases}
  \end{displaymath}
  for all \(Y \in \mathcal{S}\).  Then \(Y \sqsubseteq f(Y)\) for all
  \(Y \in \mathcal{S}\).  By part (\ref{item:j6hjwyxg}) of the
  exercise, every totally ordered subset of \(\mathcal{S}\) has a
  supremum.  Therefore, by the Bourbaki-Witt fixed point theorem
  (\Cref{exe:al7fffh1}), \(f\) has a fixed point \(M\).  As
  \(Z \sqsubset f(Z)\) for all
  \(Z \in \mathcal{S} \setminus \mathcal{M}\), \(M\) is a maximal
  element of \(\mathcal{S}\).  The hypothesis on \(X\) gives an upper
  bound \(a\) of \(M\).  By part (\ref{item:41f56yw7}) of the
  exercise, \(a\) is a maximal element of \(X\).
\end{solution}

\begin{solution}[\ref{exe:ylp1zcc8}]
  \label{sol:6szduuuo}
  Let \(Z\) be the set of elements \(t\) of \(Y\) such that \(w < t\).
  If \(t \in Z\), then \(w < t\), so by \Cref{thm:q8h11s4b},
  \(w^+ \leq t\); thus, \(w^+\) is a lower bound of \(Z\).  Let \(x\)
  be any lower bound of \(Z\); as \(w\) has a successor, by the same
  theorem, \(w < w^+\); since \(w^+ \in Y\), it follows that
  \(w^+ \in Z\); therefore, \(x \leq w^+\).  This verifies that
  \(w^+\) is the infimum of \(Z\).
\end{solution}

\begin{solution}[\ref{exe:ddc4bdfk}]
  \label{sol:knjkxl0v}
  Let \(Y\) be the set of elements \(\tau\) of \(X\) such that
  \(\xi \in \beta\).  It is nonempty because \(\xi^+ \in Y\).  Let
  \(\theta\) be the intersection of the elements of \(Y\).  We have to
  show that \(\theta = \xi^+\).  As \(\xi^+ \in Y\), the definition of
  \(\theta\) implies that \(\theta\) is a subset of \(\xi^+\).
  Conversely, if \(\tau \in Y\), then \(\xi \in \tau\), hence by
  \Cref{thm:8odofa55}, \(\xi^+ \subset \tau\); it follows that
  \(\xi^+\) is a subset of \(\theta\).  This verifies that
  \(\theta = \xi^+\).
\end{solution}

\begin{solution}[\ref{exe:9pk4eh7k}]
  \label{sol:s4fkzn20}
  Suppose there is a set \(X\) such that every ordinal is an element
  of \(X\).  Let \(Y\) denote the set of all elements of \(X\) that
  are ordinals.  If \(\alpha\) is an element of \(Y\), and
  \(\beta \in \alpha\), then by \Cref{thm:5qmhh3to}, \(\beta\) is an
  ordinal, so \(\beta \in Y\); therefore, \(Y\) is a transitive set.
  By the definition of \(Y\), every element of \(Y\) is an ordinal.
  Therefore, by \Cref{thm:22wi7nhu}, \(Y\) is itself an ordinal.  As
  every ordinal is an element of \(Y\), it follows that \(Y \in Y\).
  This contradicts the part of \Cref{thm:7cc2cckf} which says that
  \(\alpha \notin \alpha\) for every ordinal \(\alpha\).  Therefore,
  there is no set \(X\) such that every ordinal is an element of
  \(X\).
\end{solution}

\begin{solution}[\ref{exe:7tvca54w}]
  \label{sol:45jmdyf7}
  Suppose \(\alpha^- \in \beta^-\).  If \(\alpha\) is a limit ordinal,
  then by \Cref{thm:8odofa55}, \(\alpha = \alpha^-\); so as
  \(\alpha^- \in \beta^-\), we have \(\alpha \in \beta^-\); since
  \(\beta^-\) is a subset of \(\beta\), it follows that
  \(\alpha \in \beta\).  If both \(\alpha\) and \(\beta\) are
  successor ordinals, then \(\alpha \in \beta\) by
  \Cref{thm:8odofa55}.  If \(\alpha\) is a successor ordinal and
  \(\beta\) a limit ordinal, then \(\alpha = (\alpha^-)^+\) and
  \(\beta = \beta^-\); as \(\alpha^- \in \beta^-\),
  \(\alpha^- \in \beta\); since \(\alpha = (\alpha^-)^+\), by
  \Cref{thm:8odofa55}, \(\alpha\) is a subset of \(\beta\); as
  \(\alpha\) is a successor ordinal and \(\beta\) a limit ordinal,
  \(\alpha \neq \beta\); therefore, \(\alpha\) is a proper subset of
  \(\beta\); so by \Cref{thm:lebsg6gf}, \(\alpha \in \beta\).  Thus,
  in all the cases, \(\alpha \in \beta\).

  Suppose \(\alpha \in \beta\).  If both \(\alpha\) and \(\beta\) are
  successor ordinals, then by \Cref{thm:8odofa55},
  \(\alpha^- \in \beta^-\).  If both \(\alpha\) and \(\beta\) are
  limit ordinals, then by the same theorem, \(\alpha^- = \alpha\) and
  \(\beta^- = \beta\), so as \(\alpha \in \beta\), we have
  \(\alpha^- \in \beta^-\).  If \(\alpha\) is a successor ordinal and
  \(\beta\) a limit ordinal, then by \Cref{thm:8odofa55},
  \(\alpha = \xi^+\), where \(\xi = \alpha^-\), and
  \(\beta = \beta^-\); as \(\alpha \in \beta\), we have
  \(\xi^+ \in \beta\); since \(\xi \in \xi^+\) and \(\beta\) is a
  transitive set, we get \(\xi \in \beta\); now since
  \(\xi = \alpha^-\) and \(\beta = \beta^-\), this implies that
  \(\alpha^- \in \beta^-\).  Lastly, if \(\alpha\) is a limit ordinal
  and \(\beta\) a successor ordinal, then \(\alpha = \alpha^-\) and
  \(\beta = \xi^+\), where \(\xi = \beta^-\); so the hypothesis that
  \(\alpha \in \beta\) give \(\alpha \in \xi^+\); therefore, either
  \(\alpha \in \xi\) or \(\alpha = \xi\); if \(\alpha \in \xi\), then
  as \(\alpha = \alpha^-\) and \(\xi = \beta^-\), we get
  \(\alpha^- \in \beta^-\); on the other hand, if \(\alpha = \xi\),
  then we get \(\alpha^- = \beta^-\) and \(\beta = \xi^+ = \alpha^+\).
  Thus in all the cases, either \(\alpha^- \in \beta^-\), or
  \(\alpha\) is a limit ordinal and \(\beta = \alpha^+\).


\end{solution}

\bibsection

\end{document}

%%% End of file
