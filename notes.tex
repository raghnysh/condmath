%%% Notes on condensed mathematics

\documentclass{article}

\usepackage{notes}

\addbibresource{condmath.bib}

\title{Notes on condensed mathematics}

\hypersetup{pdftitle={Notes on condensed mathematics}}

\hypersetup{pdfsubject={Mathematics}}

\hypersetup{pdfkeywords={condensed mathematics, topological algebra,
    profinite spaces, category theory, abelian categories, derived
    categories, infinity categories, topos theory}}

\author{Raghavendra Nyshadham\thanks{\cczero} \\
  {\normalsize\nolinkurl{rn@raghnysh.com}}}

\hypersetup{pdfauthor={Raghavendra Nyshadham (rn@raghnysh.com)}}

\date{2024-01-14}

\begin{document}

\begin{titlingpage}
  \maketitle

  \begin{abstract}
    This document is a set of rough notes on condensed mathematics.
    The document is a work in progress.  The files of the current
    version of the document are available at
    \url{https://github.com/raghnysh/condmath}.

    \mscnumbers{20E18, 18E10, 22A99, 18F10, 18G80, 18N60}

    \ghtopics{algebraic-geometry, category-theory, topology,
      homological-algebra}
  \end{abstract}
\end{titlingpage}

\tableofcontents

\section{Introduction}
\label{sec:113nrd0o}

This document is a set of rough notes on condensed mathematics, a
subject that was developed in the references \textcite{bib:872u2noz},
\textcite{bib:iy49ytm3}, and \textcite{bib:7k2n4jtn}.  The document is
a work in progress.  The files of the current version of the document
are available at \url{https://github.com/raghnysh/condmath}.

\section{Set theory}
\label{sec:pafnta4o}

This section contains some preliminaries on set theory: ordinals,
transfinite induction, cardinals, etc.  The description of these
topics here is informal because nothing is gained for these notes by
making the development more rigorous.

\subsection{Ordered sets}
\label{sec:5f3q5o8v}

This subsection is a recapitulation of some basic notions about orders
on sets.  It is essentially a summary of \S1 of Chapter~III in
\textcite{bib:lmhdqwpw}.

The \firstterm{negation} of a relation \(R\) on a set \(X\) is the
relation \(R^\neg\) on \(X\) that is defined by setting, for all
\(x, y \in X\), \(x R^\neg y\) if \(x\) is not related to \(y\) with
respect to \(R\), that is, if the statement \(x R y\) is false.  The
\firstterm{opposite} of \(R\) is the relation \(R^{\mathrm{op}}\) on
\(X\) that is defined by setting \(x R^{\mathrm{op}} y\) if \(y R x\).
The \firstterm{strictification} of \(R\) is the relation \(R^\neq\) on
\(X\) that is defined by setting \(x R^\neq y\) if \(x R y\) and
\(x \neq y\).  It is obvious that negation and forming the opposite
are involutive while strictification is idempotent, that is,
\begin{displaymath}
  (R^\neg)^\neg = R, \quad
  (R^{\mathrm{op}})^{\mathrm{op}} = R, \quad
  (R^\neq)^\neq = R^\neq,
\end{displaymath}
and that taking the opposite commutes with the other two operations:
\begin{displaymath}
  (R^{\mathrm{op}})^\neg = (R^\neg)^{\mathrm{op}}, \quad
  (R^{\mathrm{op}})^\neq = (R^\neq)^{\mathrm{op}}.
\end{displaymath}

An \firstterm{order} on \(X\) is a relation \(R\) on it which is:
\begin{enumerate}
\item \firstterm{reflexive}: \(x R x\) for every element \(x\) of
  \(X\);
\item \firstterm{anti-symmetric}: if \(x, y \in X\), and \(x R y\) and
\(y R x\), then \(x = y\); and
\item \firstterm{transitive}: if \(x, y, z \in X\), and \(x R y\) and
  \(y R z\), then \(x R z\).
\end{enumerate}
An \firstterm{ordered set} is a set together with an order on it.

Usually the symbol \(\leq\) (or a similar symbol such as \(\preceq\)
or \(\sqsubseteq\)) is used as notation for an order.  If \(\leq\) is
used to denote an order, the opposite of \(\leq\) is denoted by
\(\geq\).  The strictification of \(\leq\) is denoted by \(<\), and
that of \(\geq\) by \(>\).  The negations of \(\leq, \geq, <, >\) are
denoted by \(\nleq, \ngeq, \nless, \ngtr\), respectively.  The
corresponding notations when another symbol is used instead of
\(\leq\) are analogous to these notations.

The restriction of an order \(\leq\) on \(X\) to any subset \(Y\) of
\(X\) is an order on \(Y\), which is said to be \firstterm{induced} by
the order on \(X\).  A family \((X_i)_{i \in I}\) of sets and an order
\(\leq_i\) on each \(X_i\) together induce on the set
\(X = \prod_{i \in I}X_i\) an order \(\leq\) called the
\firstterm{product} order, which is defined by setting for any two
elements \(x= (x_i)\) and \(y = (y_i)\) of \(X\), \(x \leq y\) if
\(x_i \leq y_i\) for all \(i\).  The product order specialises to an
order on the set \(Y^X\) of all functions from a set \(X\) to a set
\(Y\), in which any two functions \(f,g\) satisfy the condition
\(f \leq g\) if and only if \(f(x) \leq g(x)\) for all \(x \in X\).

A function \(f\) from an ordered set \(X\) to an ordered set \(Y\) is
called \firstterm{increasing} (respectively, \firstterm{decreasing})
if \(f(x) \leq f(x')\) (respectively, \(f(x) \geq f(x')\)) for all
\(x, x' \in X\) such that \(x \leq x'\).  It is called
\firstterm{strictly increasing} (respectively, \firstterm{strictly
  decreasing}) if \(f(x) < f(x')\) (respectively, \(f(x) > f(x')\))
for all \(x, x' \in X\) such that \(x < x'\).  These definitions imply
that an injective increasing (respectively, decreasing) function is
strictly increasing (respectively, strictly decreasing).

An element \(a\) of an ordered set \(X\) is called \firstterm{minimal}
(respectively, \firstterm{maximal}) if it is the only element \(x\) of
\(X\) such that \(x \leq a\) (respectively, \(x \geq a\)).  It is
called a \firstterm{minimum} (respectively, \firstterm{maximum}) of
\(X\) if \(a \leq x\) (respectively, \(a \geq x\)) for all
\(x \in X\).  Anti-symmetry implies that \(X\) has at most one
minimum, and at most one maximum.  A minimum (respectively, maximum)
of \(X\) is a minimal element (respectively, maximal element) of
\(X\).  In fact, if \(a\) is a minimum (respectively, maximum) of
\(X\), then it is the unique minimal element (respectively, maximal
element) of \(X\).

An element \(a\) of \(X\) is called a \firstterm{lower bound}
(respectively, \firstterm{upper bound}) of a subset \(Y\) of \(X\) if
\(a \leq y\) (respectively, \(a \geq y\)) for all \(y \in Y\).  A
subset of \(X\) is said to be \firstterm{bounded below} (respectively,
\firstterm{bounded above}) if it has a lower bound (respectively,
upper bound).  A maximum of the set of lower bounds of \(Y\) is called
the \firstterm{infimum} of \(Y\), and a minimum of the set of upper
bounds of \(Y\) is called the \firstterm{supremum} of \(Y\).  The
infimum and supremum of \(Y\) are unique when they exist, and are
respectively denoted by \(\inf Y\) and \(\sup Y\).

An ordered set \(X\) is said to be \firstterm{directed} (respectively,
\firstterm{codirected}) if every subset of \(X\) with two elements is
bounded above (respectively, bounded below).  A maximal element of a
directed set is its maximum, and a minimal element of a codirected set
is its minimum.

A \firstterm{lattice} is an ordered set \(X\) with the property that
every subset of \(X\) with two elements has an infimum and a supremum.
The product of a family of lattices is again a lattice.

An order on a set \(X\) is said to be \firstterm{total} if for any two
elements \(x, y\) of \(X\), either \(x \leq y\) or \(y \leq x\).  A
\firstterm{totally ordered set} is an ordered set whose order is
total.  Every strictly increasing function from a totally ordered set
to any ordered set is injective.

For any two elements \(a, b\) of an ordered set \(X\), any of the sets
\begin{align*}
  [a, b]& = \{ x \in X \,\vert\, a \leq x \leq b \},&
  [a, b)& = \{ x \in X \,\vert\, a \leq x < b \}, \\
  (a, b]& = \{ x \in X \,\vert\, a < x \leq b \},&
  (a, b)& = \{ x \in X \,\vert\, a < x < b \},
\end{align*}
or the sets
\begin{align*}
    [a, \infty)& = \{ x \in X \,\vert\, a \leq x \},&
  (a, \infty)& = \{ x \in X \,\vert\, a < x \}, \\
  (-\infty, b]& = \{ x \in X \,\vert\, x \leq b \},&
  (-\infty, b)& = \{ x \in X \,\vert\, x < b \},
\end{align*}
or the set \(X\) itself, is called an \firstterm{interval}.  A subset
\(Y\) of \(X\) is called a \firstterm{segment} if for every
\(a \in Y\), the interval \((-\infty, a]\) is contained in \(Y\).

\bibsection

\end{document}

%%% End of file
