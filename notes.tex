%%% Notes on condensed mathematics

\documentclass{article}

\usepackage{notes}

\addbibresource{condmath.bib}

\title{Notes on condensed mathematics}

\hypersetup{pdftitle={Notes on condensed mathematics}}

\hypersetup{pdfsubject={Mathematics}}

\hypersetup{pdfkeywords={condensed mathematics, topological algebra,
    profinite spaces, category theory, abelian categories, derived
    categories, infinity categories, topos theory}}

\author{Raghavendra Nyshadham\thanks{\cczero} \\
  {\normalsize\nolinkurl{rn@raghnysh.com}}}

\hypersetup{pdfauthor={Raghavendra Nyshadham (rn@raghnysh.com)}}

\date{2024-01-14}

\begin{document}

\begin{titlingpage}
  \maketitle

  \begin{abstract}
    This document is a set of rough notes on condensed mathematics.
    The document is a work in progress.  The files of the current
    version of the document are available at
    \url{https://github.com/raghnysh/condmath}.

    \mscnumbers{20E18, 18E10, 22A99, 18F10, 18G80, 18N60}

    \ghtopics{algebraic-geometry, category-theory, topology,
      homological-algebra}
  \end{abstract}
\end{titlingpage}

\tableofcontents

\section{Introduction}
\label{sec:113nrd0o}

This document is a set of rough notes on condensed mathematics, a
subject that was developed in the references \textcite{bib:872u2noz},
\textcite{bib:iy49ytm3}, and \textcite{bib:7k2n4jtn}.  The document is
a work in progress.  The files of the current version of the document
are available at \url{https://github.com/raghnysh/condmath}.

\section{Set theory}
\label{sec:pafnta4o}

This section contains some preliminaries on set theory: ordinals,
transfinite induction, cardinals, etc.  The description of these
topics here is informal because nothing is gained for these notes by
making the development more rigorous.

\subsection{Ordered sets}
\label{sec:5f3q5o8v}

This subsection is a recapitulation of some basic notions about orders
on sets.  It is essentially a summary of \S1 of Chapter~III in
\textcite{bib:lmhdqwpw}.

The \firstterm{negation} of a relation \(R\) on a set \(X\) is the
relation \(R^\neg\) on \(X\) that is defined by setting, for all
\(x, y \in X\), \(x R^\neg y\) if \(x\) is not related to \(y\) with
respect to \(R\), that is, if the statement \(x R y\) is false.  The
\firstterm{opposite} of \(R\) is the relation \(R^{\mathrm{op}}\) on
\(X\) that is defined by setting \(x R^{\mathrm{op}} y\) if \(y R x\).
The \firstterm{irreflexive kernel} of \(R\) is the relation \(R^\neq\)
on \(X\) that is defined by setting \(x R^\neq y\) if \(x R y\) and
\(x \neq y\).  The \firstterm{reflexive closure} of \(R\) is the
relation \(R^=\) on \(X\) that is defined by setting \(x R^= y\) if
\(x R y\) or \(x = y\); it is the smallest reflexive relation on \(X\)
that contains \(R\).

It is obvious that negation and forming the opposite are involutive
while the irreflexive kernel and reflexive closure are idempotent,
that is,
\begin{displaymath}
  (R^\neg)^\neg = R, \quad
  (R^{\mathrm{op}})^{\mathrm{op}} = R, \quad
  (R^\neq)^\neq = R^\neq, \quad
  (R^=)^= = R^=,
\end{displaymath}
and that taking the opposite commutes with the other three operations:
\begin{displaymath}
  (R^{\mathrm{op}})^\neg = (R^\neg)^{\mathrm{op}}, \quad
  (R^{\mathrm{op}})^\neq = (R^\neq)^{\mathrm{op}}, \quad
  (R^{\mathrm{op}})^= = (R^=)^{\mathrm{op}}.
\end{displaymath}

An \firstterm{order} on \(X\) is a relation \(R\) on it which is:
\begin{enumerate}
\item \firstterm{reflexive}: \(x R x\) for every element \(x\) of
  \(X\);
\item \firstterm{anti-symmetric}: if \(x, y \in X\), and \(x R y\) and
\(y R x\), then \(x = y\); and
\item \firstterm{transitive}: if \(x, y, z \in X\), and \(x R y\) and
  \(y R z\), then \(x R z\).
\end{enumerate}
An \firstterm{ordered set} is a set together with an order on it.

Usually the symbol \(\leq\) (or a similar symbol such as \(\preceq\)
or \(\sqsubseteq\)) is used as notation for an order.  If \(\leq\) is
used to denote an order, the opposite of \(\leq\) is denoted by
\(\geq\).  The irreflexive kernel of \(\leq\) is denoted by \(<\), and
that of \(\geq\) by \(>\).  The negations of \(\leq, \geq, <, >\) are
denoted by \(\nleq, \ngeq, \nless, \ngtr\), respectively.  The
corresponding notations when another symbol is used instead of
\(\leq\) are analogous to these notations.

Taking the irreflexive kernel is a bijection from the set of orders on
\(X\) onto the set of all relations \(<\) on \(X\) which are:
\begin{enumerate}
\item \firstterm{irreflexive}: there is no element \(x\) of \(X\) such
  that \(x < x\);
\item \emph{transitive}: if \(x,y,z \in X\), and \(x < y\) and
  \(y < z\), then \(x < z\).
\end{enumerate}
The inverse of this bijection maps any relation with these two
properties to its reflexive closure.

The restriction of an order \(\leq\) on \(X\) to any subset \(Y\) of
\(X\) is an order on \(Y\), which is said to be \firstterm{induced} by
the order on \(X\).  A family \((X_i)_{i \in I}\) of sets and an order
\(\leq_i\) on each \(X_i\) together induce on the set
\(X = \prod_{i \in I}X_i\) an order \(\leq\) called the
\firstterm{product} order, which is defined by setting for any two
elements \(x= (x_i)\) and \(y = (y_i)\) of \(X\), \(x \leq y\) if
\(x_i \leq y_i\) for all \(i\).  The product order specialises to an
order on the set \(Y^X\) of all functions from a set \(X\) to a set
\(Y\), in which any two functions \(f,g\) satisfy the condition
\(f \leq g\) if and only if \(f(x) \leq g(x)\) for all \(x \in X\).

A function \(f\) from an ordered set \(X\) to an ordered set \(Y\) is
called \firstterm{increasing} (respectively, \firstterm{decreasing})
if \(f(x) \leq f(x')\) (respectively, \(f(x) \geq f(x')\)) for all
\(x, x' \in X\) such that \(x \leq x'\).  It is called
\firstterm{strictly increasing} (respectively, \firstterm{strictly
  decreasing}) if \(f(x) < f(x')\) (respectively, \(f(x) > f(x')\))
for all \(x, x' \in X\) such that \(x < x'\).  These definitions imply
that an injective increasing (respectively, decreasing) function is
strictly increasing (respectively, strictly decreasing).  An
\firstterm{isomorphism} from \(X\) to \(Y\) is an increasing bijection
from \(X\) onto \(Y\) whose inverse is also increasing.  An
isomorphism from \(X\) to itself is called an \firstterm{automorphism}
of \(X\).

An element \(a\) of an ordered set \(X\) is called \firstterm{minimal}
(respectively, \firstterm{maximal}) if it is the only element \(x\) of
\(X\) such that \(x \leq a\) (respectively, \(x \geq a\)).  It is
called a \firstterm{minimum} (respectively, \firstterm{maximum}) of
\(X\) if \(a \leq x\) (respectively, \(a \geq x\)) for all
\(x \in X\).  Anti-symmetry implies that \(X\) has at most one
minimum, and at most one maximum.  A minimum (respectively, maximum)
of \(X\) is a minimal element (respectively, maximal element) of
\(X\).  In fact, if \(a\) is a minimum (respectively, maximum) of
\(X\), then it is the unique minimal element (respectively, maximal
element) of \(X\).

An element \(a\) of \(X\) is called a \firstterm{lower bound}
(respectively, \firstterm{upper bound}) of a subset \(Y\) of \(X\) if
\(a \leq y\) (respectively, \(a \geq y\)) for all \(y \in Y\).  A
subset of \(X\) is said to be \firstterm{bounded below} (respectively,
\firstterm{bounded above}) if it has a lower bound (respectively,
upper bound).  A maximum of the set of lower bounds of \(Y\) is called
the \firstterm{infimum} of \(Y\), and a minimum of the set of upper
bounds of \(Y\) is called the \firstterm{supremum} of \(Y\).  The
infimum and supremum of \(Y\) are unique when they exist, and are
respectively denoted by \(\inf Y\) and \(\sup Y\).

An ordered set \(X\) is said to be \firstterm{directed} (respectively,
\firstterm{codirected}) if every subset of \(X\) with two elements is
bounded above (respectively, bounded below).  A maximal element of a
directed set is its maximum, and a minimal element of a codirected set
is its minimum.

A \firstterm{lattice} is an ordered set \(X\) with the property that
every subset of \(X\) with two elements has an infimum and a supremum.
The product of a family of lattices is again a lattice.

An order on a set \(X\) is said to be \firstterm{total} if for any two
elements \(x, y\) of \(X\), either \(x \leq y\) or \(y \leq x\).  An
order on \(X\) is total if and only if it has the property of
\firstterm{trichotomy}: for any two elements \(x, y\) of \(X\),
exactly one of the conditions \(x < y\), \(x = y\), \(x > y\) is true.
A \firstterm{totally ordered set} is an ordered set whose order is
total.  Every strictly increasing surjection from a totally ordered
set \(X\) onto any ordered set \(Y\) is an isomorphism; in particular,
any increasing bijection from \(X\) onto \(Y\) is an isomorphism.
Zorn's lemma, which is equivalent to the axiom of choice, states that
an ordered set in which every totally ordered subset is bounded above
has a maximal element.

For any two elements \(a, b\) of an ordered set \(X\), any of the sets
\begin{align*}
  [a, b]& = \{ x \in X \,\vert\, a \leq x \leq b \},&
  [a, b)& = \{ x \in X \,\vert\, a \leq x < b \}, \\
  (a, b]& = \{ x \in X \,\vert\, a < x \leq b \},&
  (a, b)& = \{ x \in X \,\vert\, a < x < b \},
\end{align*}
or the sets
\begin{align*}
    [a, \infty)& = \{ x \in X \,\vert\, a \leq x \},&
  (a, \infty)& = \{ x \in X \,\vert\, a < x \}, \\
  (-\infty, b]& = \{ x \in X \,\vert\, x \leq b \},&
  (-\infty, b)& = \{ x \in X \,\vert\, x < b \},
\end{align*}
or the set \(X\) itself, is called an \firstterm{interval}.  A subset
\(Y\) of \(X\) is called a \firstterm{segment} if for every
\(a \in Y\), the interval \((-\infty, a]\) is contained in \(Y\).  The
union and the intersection of a family of segments of \(X\) are
evidently segments, as are the empty set and \(X\).

\subsection{Well-ordered sets}
\label{sec:62tryw49}

An order on a set \(X\) is called a \firstterm{well-order} if every
nonempty subset of \(X\) has a minimum.  A well-order is total: if
\(x,y\) are two elements of \(X\), and if, for instance, \(x\) is the
minimum of the set \(\{ x, y \}\), then \(x \leq y\).  A well-order on
\(X\) induces a well-order on every subset of \(X\).  Zermelo's
theorem, which is equivalent to the axiom of choice, states that every
set has a well-order.

A \firstterm{well-ordered set} is an ordered set whose order is a
well-order.  Every subset of a well-ordered set that is bounded above
has a supremum.

\begin{theorem}
  \label{thm:htdzbbwv}
  Every segment of a well-ordered set \(X\) that is a proper subset of
  \(X\) is an interval \((-\infty, a)\) for a unique element \(a\) of
  \(X\).
\end{theorem}

The complement \(Z\) of a proper segment \(Y\) of \(X\), being
nonempty, has a minimum \(a\).  For any element \(y\) of \(Y\), the
interval \((-\infty,y]\) is contained in \(Y\), so the element \(a\)
of \(X\) that lies outside \(Y\) is not in that interval; the
trichotomy property implies that \(a > y\).  Conversely, any element
\(y\) of \(X\) that is \(<\) the minimum \(a\) of \(Z\) lies outside
\(Z\), and is hence in \(Y\).  Therefore, \(Y = (-\infty,a)\).

\begin{theorem}
  \label{thm:b4yv5x3a}
  The function \(x \mapsto (-\infty, x)\) is an isomorphism from any
  well-ordered set \(X\) onto the set of proper segments of \(X\)
  ordered by inclusion.  In particular, the set of segments of \(X\)
  is well-ordered by inclusion.
\end{theorem}

The transitivity of \(<\) and \Cref{thm:htdzbbwv} imply that the
stated map is an increasing bijection from the totally ordered set
\(X\) onto the ordered set \(S\) of proper segments of \(X\), and is
hence an isomorphism.  It follows that \(S\) is well-ordered by
inclusion.  As every element of \(S\) is contained in \(X\), the set
of all segments of \(X\) is also well-ordered.

\begin{theorem}
  \label{thm:36vndfrs}
  Any segment \(Y\) of a well-ordered set \(X\) satisfies exactly one
  of the following conditions:
  \begin{enumerate}
  \item \(Y\) is an interval \((-\infty, a]\) for a unique
    \(a \in X\).
  \item \(Y\) is the union of the intervals \((-\infty, y)\) as \(y\)
    runs over \(Y\).
  \end{enumerate}
\end{theorem}

In case \(Y\) has a maximum \(a\), it is contained in the interval
\((-\infty, a]\); as \(a\) belongs to the segment \(Y\),
\((-\infty, a]\) is contained in \(Y\); therefore,
\(Y = (-\infty, a]\); anti-symmetry implies that \(a\) is the unique
element of \(X\) for which this equality is true.  On the other hand,
if \(Y\) does not have a maximum, for every element \(t\) of \(Y\),
there is an element \(y\) of \(Y\) such that \(y > t\), and because
\(Y\) is a segment, \((-\infty, y)\) is contained in \(Y\); thus,
\(Y\) is the union of the segments \((-\infty, y)\) as \(y\) runs over
\(Y\).

\begin{theorem}
  \label{thm:y3go6w2h}
  Let \(S\) be a set of segments of a well-ordered set \(X\) such
  that:
  \begin{enumerate}
  \item if \(x \in X\) and \((-\infty, x) \in S\), then
    \((-\infty, x] \in S\); and
  \item the union of any family of elements of \(S\) belongs to \(S\).
  \end{enumerate}
  Then every segment of \(X\) belongs to \(S\).
\end{theorem}

If the set \(T\) of segments of \(X\) that do not belong to \(S\) is
non-empty, then \Cref{thm:b4yv5x3a} implies that \(T\) has a minimum
\(Y\) under inclusion.  By \Cref{thm:36vndfrs}, this segment \(Y\) is
either \((-\infty, a]\) for some \(a \in X\), or the union of the
segments \((-\infty, y)\) as \(y\) runs over \(Y\).  In the first
case, the segment \((-\infty, a)\) is a proper subset of the minimum
\(Y\) of \(T\), and hence belongs to \(S\); the first hypothesis on
\(S\) implies that \(Y \in S\), a contradiction.  In the second case,
the segments \((-\infty, y)\) for \(y \in Y\) are proper subsets of
\(Y\), and hence belong to \(S\); the second hypothesis on \(S\)
implies that \(Y \in S\), again a contradiction.

An \firstterm{inductive} subset of an ordered set \(X\) is a subset
\(Y\) of \(X\) with the property that any element \(x\) of \(X\) such
that \((-\infty, x) \subset Y\) belongs to \(Y\).

\begin{theorem}[Principle of transfinite induction]
  \label{thm:5vxgawq0}
  The only inductive subset of a well-ordered set \(X\) is \(X\)
  itself.
\end{theorem}

If the complement \(Z\) of an inductive subset \(Y\) of \(X\) is
nonempty, then \(Z\) has a minimum \(a\).  Trichotomy implies that
every element \(x\) of \((-\infty, a)\) belongs to \(Y\); as \(Y\) is
inductive, \(a \in Y\), a contradiction.

Here is another proof, one that uses \Cref{thm:y3go6w2h}.  If an
inductive subset \(Y\) of \(X\) contains the interval \((-\infty, x)\)
for some \(x \in X\), it contains \(x\) also, hence
\((-\infty, x] \subset Y\).  On the other hand, the union of any
family of segments contained in \(Y\) is obviously a subset of \(Y\).
Therefore, every segment of \(X\) is contained in \(Y\).  In
particular, \(X \subset Y\).

Given a function of sets \(p : E \to X\), a \firstterm{section} of
\(p\) on a subset \(Y\) of \(X\) is a function \(s : Y \to E\) such
that \(p(s(x)) = x\) for all \(x \in Y\).  The set of all sections of
\(p\) on \(Y\) is denoted below by the symbol \(\Gamma(Y, p)\).

\begin{theorem}[Definition by transfinite recursion]
  \label{thm:srxrngd8}
  Let \(X\) be a well-ordered set, and \(p : E \to X\) a function from
  a set \(E\) to \(X\).  Suppose that for every element \(x\) of
  \(X\), \(f_x\) is a function from \(\Gamma((-\infty, x), p)\) to
  \(p^{-1}(x)\).  Then there exists a unique section \(s\) of \(p\) on
  \(X\) such that \(s(x) = f_x(s \vert_{(-\infty, x)})\) for all
  \(x \in X\).
\end{theorem}

Call a section \(s\) of \(p\) on a segment \(Y\) of \(X\)
\emph{inductive} if \(s(x) = f_x(s \vert_{(-\infty, x)})\) for all
\(x \in Y\).  The set of points where two inductive sections \(s\) and
\(t\) of \(p\) on \(Y\) agree is an inductive subset of the
well-ordered set \(Y\), so transfinite induction (\Cref{thm:5vxgawq0})
implies that \(s = t\).  The restriction of an inductive section \(s\)
of \(p\) on \(Y\) to a segment contained in \(Y\) is also inductive.

Let \(S\) be the set of segments \(Y\) of \(X\) such that there is an
inductive section of \(p\) on \(Y\).  We will show that \(S\)
satisfies the conditions of \Cref{thm:y3go6w2h}.

Suppose \(a \in X\) and \(Y = (-\infty, a)\) belongs to \(S\).  Let
\(s\) be an inductive section of \(p\) on \(Y\).  Define a section
\(t\) of \(p\) on \(Z = (-\infty, a]\) as follows:
\begin{displaymath}
  t(x) =
  \begin{cases}
    s(x) & \text{if} ~ x \in Y \\
    f_a(s) & \text{if} ~ x = a.
  \end{cases}
\end{displaymath}
As \(t\) extends \(s\),
\begin{displaymath}
  t(x) = s(x) = f_x(s \vert_{(-\infty, x)}) =
  f_x(t \vert_{(-\infty, x)})
\end{displaymath}
for all \(x \in Y\), and
\begin{displaymath}
  t(a) = f_a(s) = f_a(t \vert_{(-\infty, a)}).
\end{displaymath}
Therefore, \(t\) is inductive, and \(Z \in S\).

Suppose, on the other hand, that \(Y\) is the union of a family
\((Y_i)_{i \in I}\) of elements of \(S\).  On each \(Y_i\), there is
an inductive section \(s_i\) of \(p\).  For all \(i, j \in I\), the
segment \(Y_{ij} = Y_i \cap Y_j\) belongs to \(S\) because the
restriction \(s_{ij}\) of \(s_i\) to \(Y_{ij}\) is inductive.  As the
restriction \(s_{ji}\) of \(s_{Y_j}\) to \(Y_{ij}\) is also inductive,
the uniqueness statement above implies that \(s_{ij} = s_{ji}\).
There is, therefore, a unique function \(s : Y \to E\) which restricts
to \(s_i\) on each \(Y_i\).  It is obviously an inductive section of
\(p\), hence \(Y \in S\).

\Cref{thm:y3go6w2h} implies that every segment of \(X\) belongs to
\(S\).  In particular, \(X \in S\).

\begin{theorem}[Definition by transfinite recursion]
  \label{thm:5kg5ewo1}
  Let \(X\) be a well-ordered set, and \(T\) any set.  Suppose that
  for every element \(x\) of \(X\), \(f_x\) is a function from
  \(T^{(-\infty, x)}\) to \(T\).  Then there exists a unique function
  \(u : X \to T\) such that \(u(x) = f_x(u \vert_{(-\infty, x)})\) for
  all \(x \in X\).
\end{theorem}

Any fibre of the first projection \(p : X \times T \to X\) is
canonically identified with \(T\), and a section of \(p\) on a subset
\(Y\) of \(X\) is canonically identified with a function from \(Y\) to
\(T\).  Therefore, the present statement follows from
\Cref{thm:srxrngd8}.

\begin{theorem}
  \label{thm:hw1jxum0}
  If \(f\) is a strictly increasing function from a well-ordered set
  \(X\) to itself, then \(x \leq f(x)\) for all \(x \in X\).
\end{theorem}

If the set \(M\) of \(x \in X\) such that \(x > f(x)\) is non-empty,
it has a minimum \(a\).  Let \(b = f(a)\).  As \(a \in M\), \(a > b\);
as \(f\) is strictly increasing, \(f(a) > f(b)\), that is,
\(b > f(b)\).  Thus, \(b \in M\) and \(b < a\), a contradiction.

\begin{theorem}
  \label{thm:yvftewxp}
  The identity function is the only automorphism of any well-ordered
  set.
\end{theorem}

Let \(f\) be an automorphism of a well-ordered set \(X\), and \(g\)
its inverse.  Being increasing bijections, they are strictly
increasing.  \Cref{thm:hw1jxum0} implies that for any \(x \in X\),
\(x \leq f(x)\) and \(x \leq g(x)\); putting \(f(x)\) in the place of
\(x\) in the second of these inequalities gives
\(f(x) \leq g(f(x)) = x\); by anti-symmetry, \(f(x) = x\).

\begin{theorem}
  \label{thm:uvocd4lt}
  There is at most one isomorphism from a well-ordered set to another.
\end{theorem}

If \(f\) and \(g\) are two isomorphisms from a well-ordered set \(X\)
to another well-ordered set \(Y\), then \(g \circ f^{-1}\) is an
automorphism of \(X\).  It must be the identity map of \(X\) by
\Cref{thm:yvftewxp}, so \(f = g\).

\begin{theorem}
  \label{thm:q14zcdkw}
  If \(f\) is an isomorphism from a well-ordered set \(X\) onto a
  segment \(Y\) of \(X\), then \(Y = X\) and \(f\) is the identity
  function of \(X\).
\end{theorem}

Let \(f' : X \to X\) be the composite of \(f\) and the inclusion map
from \(Y\) to \(X\).  It is increasing and injective, hence strictly
increasing.  \Cref{thm:hw1jxum0} implies that for any \(x \in X\),
\(x \leq f'(x) = f(x)\); as \(Y = f(X)\) is a segment of \(X\),
\(x \in Y\).  Therefore, \(Y = X\).  It follows from
\Cref{thm:yvftewxp} that \(f\) is the identity map of \(X\).

\begin{theorem}
  \label{thm:7z7bwher}
  Let \(X\) and \(Y\) be well-ordered sets, and suppose that \(f\) is
  an isomorphism from \(X\) onto a segment \(Y'\) of \(Y\), and \(g\)
  is an isomorphism from \(Y\) onto a segment \(X'\) of \(X\).  Then
  \(X' = X\), \(Y' = Y\), and \(g = f^{-1}\).
\end{theorem}

The composite \(g \circ f\) is an isomorphism from \(X\) onto the
segment \(X'\) of \(X\).  By \Cref{thm:q14zcdkw}, \(X' = X\) and
\(g \circ f\) is the identity function of \(X\).  Similarly,
\(Y' = Y\) and \(f \circ g\) is the identity function of \(Y\).

\begin{theorem}
  \label{thm:zi5pzuap}
  Let \(X, Y\) be two well-ordered sets, and \(f, g\) two increasing
  functions from \(X\) to \(Y\) such that \(f(X)\) is a segment of
  \(Y\) and \(g\) is strictly increasing.  Then \(f(x) \leq g(x)\) for
  all \(x \in X\).
\end{theorem}

If the set \(M\) of \(x \in X\) such that \(f(x) > g(x)\) is
non-empty, it has a minimum \(a\).  The segment \(f(X)\) of \(Y\)
contains \(f(a)\), hence it contains the interval \((-\infty, f(a)]\).
Therefore, the element \(g(a)\) of this interval belongs to \(f(X)\):
\(g(a) = f(z)\) for some \(z \in X\).  As \(f\) is increasing and
\(f(z) < f(a)\), trichotomy gives \(z < a\), so \(z \notin M\):
\(f(z) \leq g(z)\).  The hypothesis that \(g\) is strictly increasing
and the relation \(z < a\) imply that \(g(z) < g(a)\).  It follows
that
\begin{displaymath}
  f(z) \leq g(z) < g(a) = f(z),
\end{displaymath}
a contradiction.

\begin{theorem}
  \label{thm:vy95iuyi}
  For any two well-ordered sets \(X\) and \(Y\), either \(X\) is
  isomorphic to a segment of \(Y\), or \(Y\) is isomorphic to a
  segment of \(X\).
\end{theorem}

Let \(X_0\) be the set of elements \(x\) of \(X\) for which there
exists an element \(y\) of \(Y\) such that the interval
\((-\infty, x)\) of \(X\) is isomorphic to the interval
\((-\infty, y)\) of \(Y\).  If \(y, y' \in Y\) and \((-\infty, y)\) is
isomorphic to \((-\infty, y'\), then \(y = y'\); for if, for instance,
\(y < y'\), then \((-\infty, y')\) is isomorphic to its segment
\((-\infty, y)\), so by \Cref{thm:q14zcdkw},
\((-\infty, y) = (-\infty, y')\), a contradiction.  Thus, for every
\(x \in X_0\), there is a unique element \(y\) of \(Y\) such that
\((-\infty, x)\) is isomorphic to \((-\infty, y)\).  We thus have a
function \(f : X_0 \to Y\) such that \((-\infty, x)\) is isomorphic to
\((-\infty, f(x))\).  Its image \(Y_0\) consists of the points
\(y \in Y\) for which there is an \(x \in X\) such that
\((-\infty, x)\) is isomorphic to \((-\infty, y)\).  Let
\(g : X_0 \to Y_0\) be the function induced by \(f\).

Suppose \(x \in X_0\) and \(x' < x\).  Let \(u\) be an isomorphism
from \((-\infty, x)\) onto \((-\infty, g(x))\).  Then \(u\) and
\(u^{-1}\) are strictly increasing, so \(u((-\infty, x'))\) equals
\((-\infty, u(x'))\).  Thus, \(u\) restricts to an isomorphism from
\((-\infty, x')\) onto \((-\infty, u(x'))\).  This means that
\(x' \in X_0\) and \(g(x') = u(x') < g(x)\).  Therefore, \(X_0\) is a
segment of \(X\), and \(g\) is strictly increasing.  As \(g\) is
surjective as well and \(X_0\) is totally ordered, \(g\) is an
isomorphism.  In particular, \(Y_0\) is a segment of \(Y\).  Thus,
\(g\) is an isomorphism from the segment \(X_0\) of \(X\) onto the
segment \(Y_0\) of \(Y\).

It therefore suffices to prove that either \(X_0 = X\) or \(Y_0 = Y\).
If \(X_0 \neq X\) and \(Y_0 \neq Y\), \Cref{thm:htdzbbwv} implies that
\(X_0 = (-\infty, a)\) and \(Y_0 = (-\infty, b)\) for some \(a \in X\)
and \(b \in Y\).  This means that \(g\) is an isomorphism from
\((-\infty, a)\) onto \((-\infty, b)\).  Therefore,
\(a \in X_0 = (-\infty, a)\), a contradiction.

\begin{theorem}
  \label{thm:xhbclgwz}
  Every subset of a well-ordered set \(X\) is isomorphic to a segment
  of \(X\).
\end{theorem}

Suppose \(Y\) is a subset of \(X\) that is not isomorphic to any
segment of \(X\).  By \Cref{thm:vy95iuyi}, there is an isomorphism
\(f\) from \(X\) onto a proper segment \(Y'\) of \(Y\).
\Cref{thm:htdzbbwv} gives an element \(b\) of \(Y\) such that
\begin{displaymath}
  Y' = \{ y \in Y \,\vert\, y < b \}.
\end{displaymath}
Let \(g : X \to X\) denote the composite of \(f\) and the inclusion
from \(Y'\) to \(X\).  It is strictly increasing, so
\Cref{thm:hw1jxum0} implies that \(b \leq g(b)\), a contradiction
because \(g(b) \in Y'\).

Suppose \((X_i)_{i \in I}\) is a family of ordered sets indexed by a
well-ordered set \(I\).  Define a relation \(<\) on the product set
\(X = \prod_{i \in I} X_i\) by setting \(x < y\) if \(x \neq y\) and
\(x_j < y_j\), where \(x = (x_i)\), \(y = (y_i)\), and \(j\) is the
minimum of the set of all \(i \in I\) satisfying \(x_i \neq y_i\).
The reflexive closure of \(<\) is an order on \(X\), which is called
the \firstterm{lexicographic} order.  It is a total order if every
\(X_i\) is totally ordered.

\bibsection

\end{document}

%%% End of file
